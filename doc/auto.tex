%==============================================================================
%==============================================================================
\documentclass[12pt]{report}
\usepackage{epsf}
\usepackage{graphicx}
\usepackage{include/harvard}
\usepackage{alltt}
\usepackage{longtable}
\usepackage{float}
\usepackage{moreverb}
\usepackage{amsmath,amssymb}
\usepackage{url}
\floatstyle{plain}
\restylefloat{table}
\restylefloat{figure}
\pagestyle{plain}
\topmargin=0pt
\textwidth=6.75in
\textheight=9in
\evensidemargin=-15pt
\oddsidemargin=-15pt
%==============================================================================
\newcommand{\autoversion}{0.9.6}

\newcommand{\ds}{\displaystyle}
\newcommand{\beq}{\begin{equation}} 
\newcommand{\eeq}{\end{equation}}
\def\norm#1{\parallel#1\parallel}
\def\abs#1{\mid#1\mid}
\def\eps{\epsilon}
\def\R{{\rm R}}
\def\Rn{{\rm R}^n}

\newcommand{\AUTO}{{\cal AUTO}~}
\newcommand{\AUTOc}{\AUTO 2000~}
\newcommand{\AUTOold}{\AUTO 97~}
\newcommand{\AUTOolder}{\AUTO 94~}

\newcommand{\python}{{\cal Python}}

\newcommand{\emp}{\emph}

% A font for AUTO parameters
%\newcommand{\parf}[1]{\filef{#1}}
\newcommand{\parf}[1]{{\tt #1}}

% A font for UNIX commands
%\newcommand{\commandf}[1]{{\it #1}}
\newcommand{\commandf}[1]{{\tt #1}}

% A font for environment variables
%\newcommand{\envf}[1]{\filef{#1}}
\newcommand{\envf}[1]{{\it #1}}

% A font for function names
%\newcommand{\funcf}[1]{\filef{#1}}
\newcommand{\funcf}[1]{{\bf #1}}

% A font for filenames
%\newcommand{\filef}[1]{{\tt #1}}
\newcommand{\filef}[1]{{\sf #1}}

% A font for ftp sites and web pages
%\newcommand{\webf}[1]{\commandf{#1}}
\newcommand{\webf}[1]{{\sl #1}}

\bibliographystyle{include/agsm}
%==============================================================================
\def\norm#1{\parallel#1\parallel}
\def\abs#1{\mid#1\mid}
\def\eps{\epsilon}
\def\R{{\rm R}}
\def\Rn{{\rm R}^n}
%==============================================================================
%==============================================================================
\begin{document}
\bibliographystyle{include/agsm} 
%
\title{
\LARGE {\cal AUTO}-07P :\\ 
\Large CONTINUATION AND BIFURCATION SOFTWARE \\
\Large FOR ORDINARY DIFFERENTIAL EQUATIONS\\ 
\author{
\Large Eusebius J. Doedel\\
\Large Concordia University\\
\Large Montreal, Canada\\
\\
with contributions by\\
\\
Alan R. Champneys (Bristol),
Thomas Fairgrieve (Toronto),
\\
Yuri Kuznetsov (Utrecht),
Bart Oldeman,
Randy Paffenroth (Pasadena),
\\
Bj\"orn Sandstede (Surrey),
and 
Xianjun Wang.\\
}}
\date{April 2006}
\maketitle
\tableofcontents
%
%
\newpage
~
\vskip.50truein
\subsection*{Preface}
This is a guide to the software package {\cal AUTO}
for continuation and bifurcation problems in ordinary differential 
equations.
Earlier versions of {\cal AUTO} were described in 
\citename{Do:81} \citeyear{Do:81},
\citename{DoKe:86} \citeyear{DoKe:86},
\citename{DoWa:95a} \citeyear{DoWa:95a},
\citename{WaDo:95b} \citeyear{WaDo:95b},
\citename{AUTO:97} \citeyear{AUTO:97},
\citename{AUTO:2000} \citeyear{AUTO:2000},
For a description of the basic algorithms see
\citename{DoKeKe:91a} \citeyear{DoKeKe:91a},
\citename{DoKeKe:91b} \citeyear{DoKeKe:91b}.
{\cal AUTO} incorporates the {\cal HomCont} algorithms of
\citename{ChKu:94} \citeyear{ChKu:94},
\citename{ChKuSa:95} \citeyear{ChKuSa:95}
for the bifurcation analysis of homoclinic orbits.
The Floquet multiplier algorithms were written by
\citename{Fa:94} \citeyear{Fa:94},
\citename{FaJe:91} \citeyear{FaJe:91}.
A GUI was written by
\citename{XJW:94} \citeyear{XJW:94}.
The Python CLUI is the work of Randy Paffenroth.

\vskip1.00truein
\subsection*{Acknowledgments}
The first author is much indebted to H.~B.~Keller 
of the California Institute of Technology for his inspiration,
encouragement and support.
He is also thankful to {\cal AUTO} users and research collaborators who have 
directly or indirectly contributed to its development,
in particular, 
Jean Pierre Kern\'evez, UTC, Compi\`egne, France;
Don Aronson, University of Minnesota, Minneapolis; and 
Hans Othmer, University of Utah.
Some material in this document related to the computation of connecting orbits
was developed with Mark Friedman, University of Alabama, Huntsville.
Also acknowledged is the work of Nguyen Thanh Long,
Concordia University, Montreal, on the graphics program {\cal PLAUT}.
Special thanks are due to Sheila Shull, California Institute of Technology,
for her cheerful assistance in the distribution of {\cal AUTO} over a long period
of time.
Over the years, the development of {\cal AUTO} has been supported by
various agencies through the California Institute of Technology, and
by research grants from NSERC (Canada).

The development of {\cal HomCont} has benefitted from help and advice from, 
among others, 
W.-J. Beyn, Universit\"{a}t Bielefeld,
M.~J. Friedman, University of Alabama,
A. Rucklidge, University of Cambridge, 
M. Koper, University of Utrecht and 
C.~J. Budd, University of Bristol. 
Financial support for this collaboration was also received from the U.K.
Engineering and Physical Science Research Council and the Nuffield Foundation.

%==============================================================================
%==============================================================================
\chapter{Installing {\cal AUTO}.} \label{ch:Installing_AUTO}
%==============================================================================
%==============================================================================
\section{ Installation.} \label{sec:Installation}
The {\cal AUTO} files {\tt auto.ps.gz, auto.tar.gz} and {\tt README} 
are available via FTP from 
directory {\tt pub/doedel/auto} at {\it ftp.cs.concordia.ca}.
The {\tt README} file contains the instructions for printing this manual.
Below it is assumed that you are using the Linux {\it bash} shell 
and that the file {\tt auto.tar.gz} is in your main directory.

While in your main directory, enter the commands
{\it gunzip auto.tar.gz},
followed by
{\it tar xvfo auto.tar}.
This will result in a directory {\tt auto}, 
with one subdirectory, {\tt auto/07}. 
Type {\it cd auto/07}  
to change directory to {\tt auto/07}.
Then type {\it make}.  
Upon compilation, type 
{\it make clean}  
to remove unnecessary files.
Also add the following two commands:
{\it export AUTO\_DIR=/mnt/auto/07}
(correcting the path, if necessary), and
{\it PATH="\$PATH:\$AUTO\_DIR/cmds:\$AUTO\_DIR/bin:."}
to your {\tt .bashrc} file.

The Graphical User Interface (GUI) requires the {\cal X-Window} system
and {\cal Motif}.
It may be necessary to enter their correct pathname in the Makefile
in {\tt auto/07/gui}.
Note that the GUI is not strictly necessary, since {\cal AUTO} can be run very effectively using the Linux Command Language User Interface (CLUI).
Moreover, long or complicated sequences of {\cal AUTO} calculations can
be programmed using the alternative Python CLUI. 
To compile {\cal AUTO} without the GUI, type {\it make cmd} in directory 
{\tt auto/07}.

For timing purposes, the file {\tt auto/07/src/autlib1.f}
contains references to the function {\it etime}.
If this function is not automatically supplied by your f77 compiler then it can be 
replaced by an appropriate alternative call, or it can be disabled by replacing
the two occurrences of the string {\it T=etime(timaray)} with {\it T=0.}
To recompile {\tt autlib1.f}, type {\it @C 1} in directory {\tt auto/07/src}.

To enable the PostScript conversion command {\it @ps}, 
type {\it make} in directory.  {\tt auto/07/tek2ps}.
To generate the on-line manual, type {\it make} in {\tt auto/07/doc}.

To prepare {\cal AUTO} for transfer to another machine, type {\it make superclean}
in directory {\tt auto/07} before creating the {\it tar}-file. 
This will remove all executable, object, and other non-essential files, and
thereby reduce the size of the package.

Some {\cal EISPACK} routines used by {\cal AUTO} for computing eigenvalues and
Floquet multipliers are included in the package
(\citename{EISPACK:76} \citeyear{EISPACK:76}).


\section{ Restrictions on Problem Size.} \label{sec:Restrictions}
There are size restrictions in the file {\tt auto/07/include/auto.h}
on the following {\cal AUTO}-constants~:
the effective problem dimension {\tt NDIM},
the number of collocation points {\tt NCOL},
the number of mesh intervals {\tt NTST},
the effective number of boundary conditions {\tt NBC},
the effective number of integral conditions {\tt NINT},
the effective number of equation parameters {\tt NPAR},
the number of stored branch points {\tt NBIF} for algebraic problems,
and the number of user output points {\tt NUZR}. 
See Chapter~\ref{ch:AUTO_constants} 
for the significance these constants. 
Their maxima are denoted by the  corresponding constant followed by an{\tt X}.
For example, {\tt NDIMX} in {\tt auto.h} denotes the maximum value 
of {\tt NDIM}.
If any of these maxima is exceeded in an {\cal AUTO}-run then a message 
will be printed.
The exception is the the maximum value of {\tt NPAR},
which, if exceeded, may lead to unreported errors.
Upon installation {\tt NPARX}=36; it should never be decreased below that value;
see also Section~\ref{sec:Restrictions_on_PAR}.
Size restrictions can be changed by editing {\tt auto.h}.
This must be followed by recompilation by typing {\it make} 
in directory {\tt auto/07/src}.
{\it It is strongly recommended that {\tt NCOLX=4} be used, and that 
the value of {\tt NDIMX} and {\tt NTSTX} be chosen as small as
possible for the intended application of {\cal AUTO}}.

Note that in certain cases the {\it effective dimension} may be greater
than the user dimension.
For example, for the continuation of folds,
the effective dimension is 2{\tt NDIM}+1 for algebraic equations,
and 2{\tt NDIM} for ordinary differential equations, respectively.
Similarly, for the continuation of Hopf bifurcations,
the effective dimension is 3{\tt NDIM}+2.
 
 
\section{Compatibility with Earlier Versions.} \label{sec:Compatibility}
The {\cal AUTO} output files are now called 
{\tt b.xxx} (the bifurcation-diagram-file),
{\tt s.xxx} (the solution-file),
and
{\tt d.xxx} (the diagnostics-file).
There are also minor changes in the formatting of these files 
compared to recent versions of {\cal AUTO}, such as {\cal AUTO97} 
and {\cal AUTO2000}.
The main change compared to {\cal AUTO97} is that there is now a
programmable Python CLUI.
 
%==============================================================================
%==============================================================================
\chapter{ Overview of Capabilities.} \label{ch:Overview}
%==============================================================================
%==============================================================================
\section{ Summary.} \label{sec:Summary}
{\cal AUTO} can do a limited bifurcation analysis of algebraic systems
\begin{equation} \label{1} 
  f( u , p ) = 0 ,  \qquad  f(\cdot,\cdot) , u \in \Rn.
\end{equation}
The main algorithms in {\cal AUTO}, however, are aimed at the continuation
of solutions of systems of ordinary differential equation (ODEs) of the form
\begin{equation} \label{2} 
 u'(t) = f\bigl( u(t) , p \bigr) , 
  \qquad  f(\cdot,\cdot) , u(\cdot) \in \Rn,
\end{equation}
subject to boundary (including initial) conditions and integral constraints.
Above, $p$ denotes one or more free parameters,

These boundary value algorithms also allow {\cal AUTO} to do certain stationary 
solution and wave calculations for the partial differential equation (PDE)
\begin{equation} \label{3} 
  u_t = D u_{xx} + f( u , p ), 
  \qquad  f(\cdot,\cdot) , u(\cdot) \in \Rn,
\end{equation}
where $D$ denotes a diagonal matrix of diffusion constants.

The basic algorithms used in {\cal AUTO},
as well as related algorithms, can be found in 
\citename{HBK:77} \citeyear{HBK:77},
\citename{HBK:86} \citeyear{HBK:86},
\citename{DoKeKe:91a} \citeyear{DoKeKe:91a},
\citename{DoKeKe:91b} \citeyear{DoKeKe:91b}.

Below, the basic capabilities of {\cal AUTO} are specified in more detail.
Some representative demos are also indicated.
 
\section{ Algebraic Systems.} \label{sec:algebraic_systems}
Specifically, for (\ref{1}) {\cal AUTO} can~:~
 
\begin{itemize}
\item[-]
  Compute solution families.\\  (Demo {\tt ab}; Run~1.) 
\item[-]
  Locate branch points and automatically compute
  bifurcating families. \\ (Demo {\tt pp2}; Run~1.)
\item[-]
  Locate Hopf bifurcation points and continue these in two
  parameters. \\ (Demo {\tt ab}; Runs~1 and 5.)
\item[-]
  Locate folds (limit points) and continue these 
  in two parameters. \\(Demo {\tt ab}; Runs~1,3,4.)
\item[-]
  Do each of the above for fixed points
  of the discrete dynamical system 
  $u^{(k+1)}= f( u^{(k)}, p )$ \\ (Demo {\tt dd2}.)
\item[-]
  Find extrema of an objective function along solution families
  and successively continue such extrema in more parameters.
  \\ (Demo {\tt opt}.)
\end{itemize}


\section{ Ordinary Differential Equations.} \label{sec:ODEs}
For the ODE (\ref{2}) the program can~:~
 
\begin{itemize}
\item[-]
  Compute families of stable and unstable periodic
  solutions and
  compute the Floquet multipliers, that determine stability, along
  these families.
  Starting data for the computation of periodic orbits are
  generated automatically at Hopf bifurcation points. \\
  (Demo {\tt ab}; Run~2.)
\item[-]
  Locate folds, branch points, period doubling bifurcations,
  and bifurcations to tori, along families of periodic solutions. 
  Branch switching is possible at branch points and at period 
  doubling bifurcations.  \\
  (Demos {\tt tor}, {\tt lor}.)
\item[-]  Continue folds and period-doubling bifurcations, 
  in two parameters. \\ (Demos {\tt plp}, {\tt pp3}.)
  The continuation of orbits of fixed period is also
  possible. This is the simplest way to compute curves of
  homoclinic orbits, if the period is sufficiently large.
  \\ (Demo {\tt pp2}.)
\item[-]  Do each of the above for {\it rotations}, i.e., when some of the
  solution components are periodic modulo a phase gain of a
  multiple of $2 \pi$. \\
  (Demo {\tt pen}.)
\item[-]  Follow curves of homoclinic orbits and detect and continue
  various codimension-2 bifurcations, using the {\cal HomCont} algorithms of 
  \citename{ChKu:94} \citeyear{ChKu:94},
  \citename{ChKuSa:95} \citeyear{ChKuSa:95}.\\
  (Demos  {\tt san}, {\tt mnt}, {\tt kpr}, {\tt cir},
  {\tt she}, {\tt rev}.)
\item[-]  Locate extrema of an integral objective functional along a family 
  of periodic solutions and successively continue such extrema 
  in more parameters. \\
  (Demo {\tt ops}.)
\item[-]
  Compute curves of solutions to (\ref{2}) on $[0,1]$, subject to general
  nonlinear boundary and integral conditions.
  The boundary conditions need not be separated, i.e., they may
  involve both $u(0)$ and $u(1)$ simultaneously.
  The side conditions may also depend on parameters.
  The number of boundary conditions plus the number of integral
  conditions need not equal the dimension of the ODE, 
  provided there is a corresponding number of additional
  parameter variables. \\
  (Demos {\tt exp}, {\tt int}.)
\item[-]
  Determine folds and branch points along
  solution families to the above boundary value problem.
  Branch switching is possible at branch points.
  Curves of folds can be computed in two parameters.\\
  (Demos {\tt bvp}, {\tt int}.)
\end{itemize}
 


\section{ Parabolic PDEs.} \label{sec:Parabolic_PDEs}
For (\ref{3}) the program can~:~
 
\begin{itemize}
\item[-]
  Trace out families of spatially homogeneous solutions.
  This amounts to a bifurcation analysis of the algebraic
  system (\ref{1}). However, {\cal AUTO} uses a related system instead,
  in order to enable the detection of bifurcations to wave train
  solutions of given wave speed. More precisely, bifurcations
  to wave trains are detected as Hopf bifurcations along fixed
  point families of the related ODE
  \begin{equation} \label{4} \begin{array}{cl}
  & u'(z) = v(z) ,\\
  & v'(z) =-D^{-1}  \bigl[ c~v(z) + f\bigl( u(z) , p \bigr) \bigr], \\
  \end{array} \end{equation}
  where $z = x - ct$ , with the wave speed $c$ specified by the user.\\
  (Demo {\tt wav}; Run~2.) 
\item[-]
  Trace out families of periodic wave solutions to (\ref{3}) that emanate
  from a Hopf bifurcation point of Equation~\ref{4}.
  The wave speed $c$ is  fixed along such a family, but
  the wave length $L$, i.e., the period of periodic solutions 
  to (\ref{4}),
  will normally vary. If the wave length $L$ becomes large,
  i.e., if a homoclinic orbit of Equation~\ref{4} is approached,
  then the wave tends to a solitary wave solution of (\ref{3}). \\
  (Demo {\tt wav}; Run~3.) 
\item[-]
  Trace out families of waves of fixed wave length $L$ in two parameters. 
  The wave speed $c$ may be chosen as one of these parameters.
  If $L$ is large then such a continuation gives a family
  of approximate solitary wave solutions to (\ref{3}).\\
  (Demo {\tt wav}; Run~4.) 
\item[-]
  Do time evolution calculations for (\ref{3}), given periodic
  initial data on the interval $[0,L]$.
  The initial data must be specified on $[0,1]$ and
  $L$ must be set separately because of internal scaling.
  The initial data may be given analytically or
  obtained from a previous computation of wave trains, solitary
  waves, or from a previous evolution calculation.
  Conversely, if an evolution calculation results in a
  stationary wave then this wave can be used as starting data
  for a wave continuation calculation.\\
  (Demo {\tt wav}; Run~5.)
\item[-]
  Do time evolution calculations for (\ref{3}) subject to user-specified
  boundary conditions.
  As above, the initial data must be specified on $[0,1]$ and the space
  interval length $L$ must be specified separately.
  Time evolution computations of (\ref{3}) are adaptive in space and
  in time. Discretization in time is not very accurate~: only
  implicit Euler. Indeed, time integration of (\ref{3}) has only been
  included as a convenience and it is not very efficient.
  (Demos {\tt pd1}, {\tt pd2}.)
\item[-]
  Compute curves of stationary solutions to (\ref{3}) subject to user-specified
  boundary conditions.
  The initial data may be given analytically, obtained from a previous 
  stationary solution computation, or from a time evolution calculation.\\
  (Demos {\tt pd1}, {\tt pd2}.)
\end{itemize}
 
In connection with periodic waves,
note that (\ref{4}) is just a special case of (\ref{2}) and
that its fixed point analysis is a special case of (\ref{1}).
One advantage of the built-in capacity of {\cal AUTO} to deal with
problem (\ref{3}) is that the user need only specify $f$, $D$, and $c$.
Another advantage is the compatibility of output data for
restart purposes. This allows switching back and forth between
evolution calculations and wave computations.

\section{ Discretization.} \label{sec:Discretization}
  {\cal AUTO} discretizes ODE boundary value problems
  (which includes periodic solutions) by the method of orthogonal 
  collocation using piecewise polynomials with 2-7 collocation points 
  per mesh interval (\citename{dBSw:73} \citeyear{dBSw:73}).
  The mesh automatically adapts to the solution to equidistribute
  the local discretization error (\citename{RuCr:78} \citeyear{RuCr:78}).
  The number of mesh intervals and the number of collocation points
  remain constant during any given run, although they may be changed 
  at restart points.
  The implementation is {\cal AUTO}-specific. In particular, the choice of
  local polynomial basis
  and the algorithm for solving the linearized collocation systems
  were specifically designed for use in numerical bifurcation analysis.
  
%==============================================================================
%==============================================================================
\chapter{ User-Supplied Files.} \label{ch:User_supplied_files}
%==============================================================================
%==============================================================================
The user must prepare the two files described below.
This can be done with the GUI described in Chapter~\ref{ch:GUI}, 
or independently.

\section{ The Equations-File {\tt xxx.f}} 
A source file {\tt xxx.f} containing the Fortran routines
{\tt FUNC}, {\tt STPNT}, {\tt BCND}, {\tt ICND}, {\tt FOPT}, and {\tt PVLS}.
Here {\tt xxx} stands for a user-selected name. 
If any of these routines is irrelevant 
to the problem then its body need not be completed.
Examples are in {\tt auto/07/demos}, where, e.g.,
the file {\tt ab/ab.f} defines a two-dimensional dynamical system,
and the file {\tt exp/exp.f} defines a boundary value problem.
The simplest way to create a new equations-file is to copy 
an appropriate demo file.
For a fully documented equations-file see {\tt auto/07/gui/aut.f}.
In GUI mode, this file can be directly loaded with the GUI-button 
{\it Equations/New}; see Section~\ref{sec:GUI_Menu_bar}.
 

\section{ The Constants-File {\tt c.xxx}} 
{\cal AUTO}-constants for {\tt xxx.f} are normally expected 
in a corresponding file {\tt c.xxx}.
Specific examples include {\tt ab/r.ab}
and {\tt exp/r.exp} in {\tt auto/07/demos}.
See Chapter~\ref{ch:AUTO_constants}
for the significance of each constant.

\newpage
\section{ User-Supplied Routines.} \label{sec: User_supplied_routines}
The purpose of each of the user-supplied routines in
the file {\tt xxx.f} is described below.
  
\begin{itemize}
\item[-] {\tt FUNC}~:~
  defines the function $f(u,p)$ in (\ref{1}), (\ref{2}), or (\ref{3}).
\item[-] {\tt STPNT}~:~
  This routine is called only if {\tt IRS}=0 
(see Section~\ref{sec:IRS} for {\tt IRS}),
  which typically is the case for the first run.
  It defines a starting solution $(u,p)$ of (\ref{1}) or (\ref{2}).
  The starting solution should not be a branch point.\\ 
  (Demos {\tt ab}, {\tt exp}, {\tt frc}, {\tt lor}.)  
\item[-] {\tt BCND}~:~ 
  A routine {\tt BCND} that defines the boundary conditions. \\
  (Demo {\tt exp}, {\tt kar}.)
\item[-] {\tt ICND}~:~ 
  A routine {\tt ICND} that defines the integral conditions. \\ 
  (Demos {\tt int}, {\tt lin}.)  
\item[-] {\tt FOPT}~:~ 
  A routine {\tt FOPT} that defines the objective functional. \\ 
  (Demos {\tt opt}, {\tt ops}.)  
\item[-] {\tt PVLS}~:~
  A routine {\tt PVLS} for defining ``solution measures''. \\
  (Demo {\tt pvl}.)
\end{itemize}
 
\section{ User-Supplied Derivatives.} \label{sec:derivatives}
If {\cal AUTO}-constant {\tt JAC} equals 0 
then derivatives need not be specified in 
{\tt FUNC}, {\tt BCND}, {\tt ICND}, and {\tt FOPT}; see Section~{sec:JAC}.
If {\tt JAC=1} then derivatives must be given.
This may be necessary for sensitive 
problems, and is recommended for computations in which {\cal AUTO} 
generates an extended system.
Examples of user-supplied derivatives can be found in
demos  {\tt dd2}, {\tt int}, {\tt plp}, {\tt opt}, and {\tt ops}.

%==============================================================================
%==============================================================================
\chapter{ Running {\cal AUTO} using Linux Commands.} \label{sec:command_mode}
%==============================================================================
%==============================================================================
{\cal AUTO} can be run with the GUI described in Chapter~\ref{ch:GUI} 
or with the commands described below.
The {\cal AUTO} aliases must have been activated; see Section~\ref{sec:Installation}; 
and an equations-file {\tt xxx.f} 
and a corresponding constants-file {\tt c.xxx} 
(see Section~\ref{ch:User_supplied_files})
must be in the current user directory.
\\
{\it Do not run {\cal AUTO} in the directory {\tt auto/07} 
or in any of its subdirectories.}

\section{ Basic commands.} 

\begin{itemize}
\item[\tt @r]:
  Type {\it @r xxx} to run {\cal AUTO}.
  Restart data, if needed, are expected in {\tt s.xxx},
  and {\cal AUTO}-constants in {\tt c.xxx}.
  This is the simplest way to run {\cal AUTO}.
\item[-]
  Type {\it @r xxx yyy} to run {\cal AUTO}
  with equations-file {\tt xxx.f} and restart data-file {\tt s.yyy}.
  {\cal AUTO}-constants must be in {\tt c.xxx}.
\item[-]
  Type {\it @r xxx yyy zzz} to run {\cal AUTO}
  with equations-file {\tt xxx.f}, restart data-file {\tt s.yyy}
  and constants-file {\tt c.zzz}.

\item[\tt @R]~:
  The command {\it @R xxx} is equivalent to the command {\it @r xxx} above.
\item[-]
  Type {\it @R xxx i}  to run {\cal AUTO} with equations-file {\tt xxx.f},
  constants-file {\tt c.xxx.i}
  and, if needed, restart data-file {\tt s.xxx}. 
\item[-]
  Type {\it @R xxx i yyy} to run {\cal AUTO}
  with equations-file {\tt xxx.f}, 
  constants-file {\tt c.xxx.i}
  and restart data-file {\tt s.yyy}.

\item[\tt @sv]~:
  Type {\it @sv xxx} to save the output-files 
  {\tt fort.7}, {\tt fort.8}, {\tt fort.9},
  as {\tt b.xxx}, {\tt s.xxx}, {\tt d.xxx}, respectively.
  Existing files by these names will be deleted.

\item[\tt @ap]~:
  Type {\it @ap xxx} to append the output-files 
  {\tt fort.7}, {\tt fort.8}, {\tt fort.9}, 
  to existing data-files 
  {\tt b.xxx}, {\tt s.xxx}, {\tt d.xxx}, resp.
\item[-]
  Type {\it @ap xxx yyy} 
  to append 
  {\tt b.xxx}, {\tt s.xxx}, {\tt d.xxx}, to
  {\tt b.yyy}, {\tt s.yyy}, {\tt d.yyy}, resp.
\end{itemize}

\section{ Plotting commands.} 

\begin{itemize}

\item[\tt @p]~:
  Type {\it @p xxx} to run the graphics program {\cal PLAUT}
  (See Chapter~\ref{ch:PLAUT})
  for the graphical inspection of the data-files 
  {\tt b.xxx} and {\tt s.xxx}. 
\item[-]
  Type {\it @p} to run the graphics program {\cal PLAUT}
  for the graphical inspection of the output-files 
  {\tt fort.7} and {\tt fort.8}.

\item[\tt @ps]~:
  Type {\it @ps fig.x} to convert a saved {\cal PLAUT} figure {\tt fig.x}
  from compact {\cal PLOT10} format to {\cal PostScript} format.
  The converted file is called {\tt fig.x.ps}. 
  The original file is left unchanged.

\end{itemize}

\section{ File-manipulation.} 

\begin{itemize}

\item[\tt @cp]~:
  Type {\it @cp xxx yyy} 
  to copy the data-files 
  {\tt b.xxx}, {\tt s.xxx}, {\tt d.xxx}, {\tt c.xxx} to
  {\tt b.yyy}, {\tt s.yyy}, {\tt d.yyy}, {\tt c.yyy}, respectively.

\item[\tt @mv]~:
  Type {\it @mv xxx yyy} 
  to move the data-files 
  {\tt b.xxx}, {\tt s.xxx}, {\tt d.xxx}, {\tt c.xxx}, to
  {\tt b.yyy}, {\tt s.yyy}, {\tt d.yyy}, {\tt c.yyy}, respectively.

\item[\tt @df]~:
  Type {\it @df} 
  to delete the output-files 
  {\tt fort.7}, {\tt fort.8}, {\tt fort.9}.

\item[\tt @cl]~:
  Type {\it @cl} 
  to clean the current directory.
  This command will delete  all files of the form
  {\tt fort.*}, {\tt *.o}, and {\tt *.exe}.

\item[\tt @dl]~:
  Type {\it @dl xxx} 
 to delete the data-files 
  {\tt b.xxx}, {\tt s.xxx}, {\tt d.xxx}.
\end{itemize}

\section{ Diagnostics.} 

\begin{itemize}
\item[\tt @lp]~:
  Type {\it @lp} to list the value of the ``limit point function'' 
  in the output-file {\tt fort.9}. This function
  vanishes at a limit point (fold).
  \item[-]
  Type {\it @lp xxx} to list the value of the ``limit point function'' 
  in the data-file {\tt d.xxx}. This function
  vanishes at a limit point (fold).
\item[\tt @bp]~:
  Type {\it @bp} to list the value of the ``branch-point function'' 
  in the output-file {\tt fort.9}. This function
  vanishes at a branch point.
  \item[-]
  Type {\it @bp xxx} to list the value of the ``branch-point function''
  in the data-file {\tt d.xxx}. This function
  vanishes at a branch point.
\item[\tt @hb]~:
  Type {\it @hb} to list the value of the ``Hopf function'' 
  in the output-file {\tt fort.9}. This function
  vanishes at a Hopf bifurcation point.
  \item[-]
  Type {\it @hb xxx} to list the value of the ``Hopf function''
  in the data-file {\tt d.xxx}. This function
  vanishes at a  Hopf bifurcation point.
\item[\tt @sp]~:
  Type {\it @sp} to list the value of the 
  ``secondary-periodic bifurcation function'' 
  in the output-file {\tt fort.9}. This function
  vanishes at period-doubling and torus bifurcations.
  \item[-]
  Type {\it @sp xxx} to list the value of the
   ``secondary-periodic bifurcation function''
  in the data-file {\tt d.xxx}. This function
  vanishes at period-doubling and torus bifurcations.
\item[\tt @it]~:
  Type {\it @it} to list the number of Newton iterations per
  continuation step in {\tt fort.9}. 
  \item[-]
   Type {\it @it xxx} to list the number of Newton iterations per
  continuation step in {\tt d.xxx}. 
\item[\tt @st]~:
  Type {\it @st} to list the number of stable eigenvalues or stable
  Floquet multipliers per continuation step in  {\tt fort.9}. 
\item[\tt @ss]~:
  Type {\it @st} to list the continuation step size for each
  continuation step in  {\tt fort.9}. 
  \item[-]
   Type {\it @st xxx} to list the continuation step size for each
  continuation step in {\tt d.xxx}. 
\item[\tt @ev]~:
  Type {\it @ev} to list the eigenvalues of the Jacobian 
  in {\tt fort.9}. 
  (Algebraic problems.)
  \item[-]
   Type {\it @ev xxx} to list the eigenvalues of the Jacobian 
  in {\tt d.xxx}. 
  (Algebraic problems.)
\item[\tt @fl]~:
  Type {\it @fl} to list the Floquet multipliers
  in the output-file {\tt fort.9}. 
  (Differential equations.)
  \item[-]
   Type {\it @fl xxx} to list the Floquet multipliers 
  in the data-file {\tt d.xxx}. 
  (Differential equations.)
\end{itemize}

\section{ File-editing.} 

\begin{itemize}

\item[\tt @e7]~:
  To use the vi editor to edit the output-file {\tt fort.7}.
\item[\tt @e8]~:
  To use the vi editor to edit the output-file {\tt fort.8}.
\item[\tt @e9]~:
  To use the vi editor to edit the output-file {\tt fort.9}.
\item[\tt @j7]~:
  To use the SGI jot editor to edit the output-file {\tt fort.7}.
\item[\tt @j8]~:
  To use the SGI jot editor to edit the output-file {\tt fort.8}.
\item[\tt @j9]~:
  To use the SGI jot editor to edit the output-file {\tt fort.9}.
  
\end{itemize}

\section{ File-maintenance.} 

\begin{itemize}
\item[\tt @lb]~:
  Type {\it @lb} to run an interactive utility program
  for listing, deleting and relabeling solutions 
  in the output-files {\tt fort.7} and {\tt fort.8}.
  The original files are backed up as
{\tt $\sim$fort.7} and {\tt $\sim$fort.8}. 
  \item[-]
  Type {\it @lb xxx} to list, delete and relabel solutions
  in the data-files {\tt b.xxx} and {\tt s.xxx}.
  The original files are backed up as {\tt $\sim$b.xxx} and {\tt $\sim$s.xxx}. 
\item[-]
  Type {\it @lb xxx yyy} to list, delete and relabel solutions
  in the data-files {\tt b.xxx} and {\tt s.xxx}.
  The modified files are written as {\tt b.yyy} and {\tt s.yyy}. 

\item[\tt @fc]~:
  Type {\it @fc xxx} to convert a user-supplied data file {\tt xxx.dat}
  to {\cal AUTO} format. The converted file is called {\tt s.dat}.
  The original file is left unchanged.
  {\cal AUTO} automatically sets the period in {\tt PAR(11)}.
  Other parameter values must be set in {\tt STPNT}. (When necessary,
  PAR(11) may also be redefined there.) 
  The constants-file file {\tt c.xxx} must be present, as the 
  {\cal AUTO}-constants {\tt NTST} and {\tt NCOL} 
  (Sections~\ref{sec:NTST} and \ref{sec:NCOL}) are used to define the new mesh.
  For examples of using the {\it @fc} command see demos {\tt lor} and {\tt pen}.

\end{itemize}

\section{ HomCont commands.} 

\begin{itemize}
\item[\tt @h]~:
  Use {\tt @h} instead of {\tt @r} when using {\cal HomCont}, i.e., when {\tt IPS}=9
  (see Chapter~\ref{ch:HomCont}).
  Type {\it @h xxx} to run {\cal AUTO}/{\cal HomCont}.
  Restart data, if needed, are expected in {\tt s.xxx},
  {\cal AUTO}-constants in {\tt c.xxx} and {\cal HomCont}-constants in {\tt s.xxx}.
\item[-]
  Type {\it @h xxx yyy} to run {\cal AUTO}/{\cal HomCont}
  with equations-file {\tt xxx.f} and restart data-file {\tt s.yyy}.
  {\cal AUTO}-constants must be in {\tt c.xxx} and {\cal HomCont}-constants in {\tt s.xxx}.
\item[-]
  Type {\it @h xxx yyy zzz} to run {\cal AUTO}/{\cal HomCont}
  with equations-file {\tt xxx.f}, restart data-file {\tt s.yyy}
  and constants-files {\tt c.zzz} and {\tt s.zzz}.

\item[\tt @H]~:
  The command {\it @H xxx} is equivalent to the command {\it @h xxx} above.
\item[-]
  Type {\it @H xxx i} in order to run {\cal AUTO}/{\cal HomCont} with equations-file {\tt xxx.f}
  and constants-files {\tt c.xxx.i} and {\tt s.xxx.i}
  and, if needed, restart data-file {\tt s.xxx}. 
\item[-]
  Type {\it @H xxx i yyy} to run {\cal AUTO}/{\cal HomCont}
  with equations-file {\tt xxx.f}, 
  constants-files {\tt c.xxx.i} and {\tt s.xxx.i},
  and restart data-file {\tt s.yyy}.
\end{itemize}

\section{ Copying a demo.} 

\begin{itemize}

\item[\tt @dm]~:
  Type {\it @dm xxx} 
  to copy all files 
  from {\tt auto/07/demos/xxx}
  to the current user directory.
  Here {\tt xxx} denotes a demo name; e.g., {\tt abc}.
  Note that the {\it @dm} command also copies a {\it Makefile}
  to the current user directory. To avoid the overwriting of
  existing files, always run demos in a clean work directory.
\end{itemize}

\section{ Viewing the manual.} 

\begin{itemize}

\item[\tt @mn]~: Use {\cal Ghostview} to view the PostScript version of this manual.
\end{itemize}

\newpage

%==============================================================================
%==============================================================================
\chapter{ Running {\cal AUTO} using Python Commands.} \label{sec:python_mode}
%==============================================================================
%==============================================================================

%==============================================================================
%==============================================================================
\chapter{ Output Files.} \label{ch:Output_files}
%==============================================================================
%==============================================================================
{\cal AUTO} writes four output-files~:
\begin{itemize}
\item[-] {\tt fort.6}~:~
  A summary of the computation is written in {\tt fort.6}, which usually
  corresponds to the window in which {\cal AUTO} is run. 
  Only special, labeled solution points are noted, namely those listed
  in Table~\ref{tbl:Solution_Types}.
  The letter codes in the Table are used in the screen output.
  The numerical codes are used internally and in
  the {\tt fort.7} and {\tt fort.8} output-files described below.

\begin{table}[htbp]
\begin{center}
\begin{tabular}{| l | r | l |}
\hline
 BP & (1)  & Branch point (algebraic systems) \\
\hline
 LP & (2)  & Fold (algebraic systems) \\
\hline
 HB & (3)  & Hopf bifurcation \\
\hline
  & (4)  & User-specified regular output point \\
\hline
 UZ & (-4)  & Output at user-specified parameter value \\
\hline
 LP & (5)  & Fold (differential equations) \\
\hline
 BP & (6)  & Branch point (differential equations) \\
\hline
 PD & (7)  & Period doubling bifurcation \\
\hline
 TR & (8)  & Torus bifurcation \\
\hline
 EP & (9)  & End point of family; normal termination \\
\hline
 MX & (-9)  & Abnormal termination; no convergence \\
\hline
\end{tabular}
\caption{Solution Types.}
\label{tbl:Solution_Types}
\end{center}
\end{table}
 

\item[-] {\tt fort.7}~:~ 
  The {\tt fort.7} output-file contains the bifurcation diagram.
  Its format is the same as the {\tt fort.6} (screen) output, 
  but the {\tt fort.7} output is more extensive, as every solution point has 
  an output line printed.
\item[-] {\tt fort.8}~:~ 
  The {\tt fort.8} output-file contains complete graphics and restart data
  for selected, labeled solutions. 
  The information per solution is generally much more extensive than
  that in {\tt fort.7}. 
  The {\tt fort.8} output should normally be kept to a minimum.
\item[-] {\tt fort.9}~:~
  Diagnostic messages, convergence history, eigenvalues, and 
  Floquet multipliers are written in {\tt fort.9}.
  It is strongly recommended that this output be habitually inspected.
  The amount of diagnostic data can be controlled via the {\cal AUTO}-constant {\tt IID};
  see Section~\ref{sec:IID}.
\end{itemize}

The user has some control over the {\tt fort.6} (screen) and {\tt fort.7} output 
via the {\cal AUTO}-constant {\tt IPLT} (Section~\ref{sec:IPLT}).
Furthermore, the routine {\tt PVLS} can be used to define ``solution measures''
which can then be printed by ``parameter overspecification'';
see Section~\ref{sec:Parameter_over_specification}.
For an example see demo {\tt pvl}.

The {\cal AUTO}-commands {\it @sv}, {\it @ap}, and {\it @df} can be used 
to manipulate  the output-files {\tt fort.7}, {\tt fort.8},
and {\tt fort.9}.
Furthermore, the {\cal AUTO}-command {\it @lb} can be used to delete and
relabel solutions simultaneously in {\tt fort.7} and {\tt fort.8}.
For details see Section~\ref{sec:command_mode}.

The graphics program {\cal PLAUT} can be used to graphically inspect 
the data in {\tt fort.7} and {\tt fort.8}; see Chapter~\ref{ch:PLAUT}.
 
%==============================================================================
%==============================================================================
\chapter{ The Graphics Program PLAUT.} \label{ch:PLAUT}
%==============================================================================
%==============================================================================
{\cal PLAUT} can be used to extract graphical
information from the {\cal AUTO} output-files {\tt fort.7} and {\tt fort.8},
or from the corresponding data-files {\tt b.xxx} and {\tt s.xxx}.
To invoke {\cal PLAUT}, use the the {\it @p} command defined in 
Section~\ref{sec:command_mode}.
The {\cal PLAUT} window (a Tektronix window) will appear, in which {\cal PLAUT}
commands can be entered.
For examples of using {\cal PLAUT} see the tutorial demo {\tt ab}, in particular,
Sections~\ref{sec:Tutorial_PLAUT} and \ref{sec:Tutorial_plotting_2p}.
See also demo {\tt pp2} in Section~\ref{sec:Demos_pp2}.


\section{ Basic {\cal PLAUT}-Commands.} \label{sec:main_PLAUT_commands}
The principal {\cal PLAUT}-commands are 
\begin{itemize}
\item[\tt bd0]~:
  This command is useful for an initial overview of the bifurcation
  diagram as stored in {\tt fort.7}.
  If you have not previously selected one of the default options 
  {\it d0, d1, d2, d3}, or {\it d4} described below then you will be asked
  whether you want solution labels, grid lines, titles, or labeled axes.

\item[\tt bd]~:
  This command is the same as the {\it bd0} command, except that you will be
  asked to enter the minimum and the maximum of the horizontal and 
  vertical axes.
  This is useful for blowing up portions of a previously displayed
  bifurcation diagram.

\item[\tt ax]~:
  With the {\it ax} command you can select any pair of columns of real
  numbers from {\tt fort.7} as horizontal and vertical axis in the
  bifurcation diagram. (The default is columns 1 and 2).
  To determine what these columns represent, one can look at the
  screen ouput of the corresponding {\cal AUTO} run, or one can inspect the
  column headings in {\tt fort.7}.
  
\item[\tt 2d]~:
  Upon entering the {\it 2d} command, the labels of all solutions stored 
  in {\tt fort.8} will be listed and you can select one or more of these 
  for display. The number of solution components is also listed
  and you will be prompted to select two of these as horizontal and
  vertical axis in the display.
  Note that the first component is typically the independent 
  time or space variable scaled to the interval [0,1].

\item[\tt sav]~:
  To save the displayed plot in a file. You will be asked to enter
  a file name. Each plot must be stored in a separate new file.
  The plot is stored in compact {\cal PLOT10} format, which can be converted to 
  {\cal PostScript} format with the {\cal AUTO}-commands {\tt @ps};
  see Section~\ref{sec:Printing_PLAUT_files}.

\item[\tt cl]~:  To clear the graphics window.

\item[\tt lab]~:
  To list the labels of all solutions stored in {\tt fort.8}.
  Note that {\cal PLAUT} requires all labels to be distinct.
  In case of multiple labels you can use the {\cal AUTO}
  command {\it @lb} to relabel solutions in
  {\tt fort.7} and {\tt fort.8}.

\item[\tt end]~:  To end execution of {\cal PLAUT}.
\end{itemize}


\section{ Default Options.} \label{sec:PLAUT_default}
After entering the commands {\it bd0, bd}, or {\it 2d}, you will be asked whether you 
want solution labels, grid lines, titles, or axes labels.
For quick plotting it is convenient to bypass these selections.
This can be done by the default commands {\it d0, d1, d2, d3}, or {\it d4} below.
These can be entered as a single command 
or they can be entered as prefixes in the {\it bd0} and {\it bd} commands. 
Thus, for example, one can enter the command {\it d1bd0}.  

\begin{itemize}
\item[\tt d0]~:  Use solid curves, showing solution labels and symbols.  
\item[\tt d1]~:  Use solid curves, except use dashed curves for unstable
  solutions and for solutions of unknown stability.
  Show solution labels and symbols.
\item[\tt d2]~:  As {\it d1}, but with grid lines.  
\item[\tt d3]~:  As {\it d1}, except for periodic solutions use 
  solid circles if stable,
  and open circles if unstable or if the stability
  is unknown.
\item[\tt d4]~:  Use solid curves, without labels and symbols.  
\end{itemize}

If no default option {\it d0, d1, d2, d3}, or {\it d4} has been selected 
or if you want to override a default feature,
then the the following commands can be used.
These can be entered as individual commands or as prefixes.
For example, one can enter the command {\it sydpbd0}.

\begin{itemize}
\item[\tt sy]~:  Use symbols for special solution points, for example,
  open square = branch point,
  solid square = Hopf bifurcation.
\item[\tt dp]~:  ``Differential Plot'', i.e., show stability of the 
  solutions. Solid curves represent stable solutions.
  Dashed curves are used for unstable
  solutions and for solutions of unknown stability.
  For periodic solutions use solid/open circles
  to indicate stability/instability (or unknown
  stability).
\item[\tt st]~:  Set up titles and axes labels. 
\item[\tt nu]~:  Normal usage (reset special options). 
\end{itemize}


\section{ Other {\cal PLAUT}-Commands.} \label{sec:Other_PLAUT_commands}
The full {\cal PLAUT} program has several other capabilities, for example,

\begin{itemize}
\item[\tt scr]~:  To change the diagram size.
\item[\tt rss]~:  To change the size of special solution point symbols.
\end{itemize}


\section{ Printing {\cal PLAUT} Files.} \label{sec:Printing_PLAUT_files}
\begin{itemize}
\item[\tt @ps]~:
  Type {\it @ps fig.1} to convert a saved {\cal PLAUT} file {\tt fig.1} 
  to {\cal PostScript} format
  in {\tt fig.1.ps}.
\end{itemize}

 
%==============================================================================
%==============================================================================
\chapter{ The Graphics Program PLAUT04.} \label{ch:PLAUT04}
%==============================================================================
%==============================================================================

%==============================================================================
%==============================================================================
\chapter{ The Graphical User Interface GUI94.} \label{ch:GUI}
%==============================================================================
%==============================================================================
\section{ General Overview.} \label{sec:GUI_Overview}
The {\cal AUTO} graphical user interface (GUI) is a tool
for creating and editing equations-files and constants-files;
see Section~\ref{ch:User_supplied_files}
 for a description of these files.
The GUI can also be used to run {\cal AUTO} and to manipulate and plot
output-files and data-files; 
see Section~\ref{sec:command_mode} for corresponding commands.
To use the GUI for a new equation, change to an empty work directory.
For an existing equations-file, change to its directory.
({\it Do not activate the GUI in the directory {\tt auto/07} 
or in any of its subdirectories.})
Then type 

\centerline { @{\it auto}, }

or its abbreviation @{\it a}.
Here we assume that the {\cal AUTO} aliases have been activated; 
see Section~\ref{sec:Installation}.
The GUI includes a window for editing the equations-file,
and four groups of buttons, namely,
the {\it Menu Bar} at the top of the GUI,
the {\it Define Constants}-buttons at the center-left,
the {\it Load Constants}-buttons at the lower left,
and the {\it Stop- and Exit}-buttons.

{\bf Note :}~
Most GUI buttons are activated by point-and-click action with 
the {\it left} mouse button. 
If a beep sound results then the {\it right} mouse button must be used. 

\subsection{ The Menu bar.}
It contains the main buttons for running {\cal AUTO}
and for manipulating the equations-file, the constants-file,
the output-files, and the data-files.
In a typical application, these buttons are used from left to right.
First the {\it Equations} are defined and, if necessary, {\it Edited},
before being {\it Written}.
Then the {\cal AUTO}-constants are {\it Defined}.
This is followed by the actual {\it Run} of {\cal AUTO}.
The resulting output-files can be {\it Saved} as data-files,
or they can be {\it Appended} to existing data-files.
Data-files can be {\it Plotted} with the graphics program {\cal PLAUT},
and various file operations can be done with the {\it Files}-button.
Auxiliary functions are provided by the {\it Demos-}, {\it Misc-},
and {\it Help}-buttons.
The Menu Bar buttons are described in more detail 
in Section~\ref{sec:GUI_Menu_bar}.


\subsection{ The Define-Constants-buttons.}
These have the same function as
the {\it Define}-button on the  Menu Bar, namely to set and change
{\cal AUTO}-constants.
However, 
for the {\it Define}-button all constants appear in one panel, 
while 
for the Define Constants-buttons they
are grouped by function, 
as in Chapter~\ref{ch:AUTO_constants}, namely
{\it Problem} definition constants,
{\it Discretization} constants,
convergence {\it Tolerances},
continuation {\it Step Size},
diagram {\it Limits},
designation of free {\it Parameters},
constants defining the {\it Computation},
and
constants that specify {\it Output} options.


\subsection{ The Load-Constants-buttons.}
The {\it Previous}-button can be used to load an existing {\cal AUTO}-constants file.
Such a file is also loaded, if it exists,
by the {\it Equations}-button on the {\it Menu Bar}.
The {\it Default}-button can be used
to load  default values of all {\cal AUTO}-constants. 
Custom editing is normally necessary.


\subsection{ The Stop- and Exit-buttons.}
The {\it Stop}-button can be used to abort execution of an {\cal AUTO}-run.
This should be done only in exceptional circumstances.
Output-files, if any, will normally be incomplete and should be deleted.
Use the {\it Exit}-button to end a session.


\section{ The Menu Bar.} \label{sec:GUI_Menu_bar}
\subsection{ Equations-button.}
This pull-down menu contains the items
{\it Old}, to load an existing equations-file,
{\it New}, to load a model equations-file,
and
{\it Demo}, to load a selected demo equations-file.
Equations-file names are of the form {\tt xxx.f}.
The corresponding constants-file {\tt c.xxx} is also loaded if it exists.
The equation name {\tt xxx} remains active until redefined.

\subsection{ Edit-button.}
This pull-down menu contains the items
{\it Cut} and {\it Copy}, 
to be performed on text in the GUI window
highlighted by click-and-drag action of the mouse,
and the item {\it Paste}, which places editor buffer text at the
location of the cursor.



\subsection{ Write-button.}
This pull-down menu contains the item
{\it Write},
to write the loaded files {\tt xxx.f} and {\tt c.xxx},
by the active equation name,
and the item
{\it Write As}
to write these files by a selected new name, which then becomes the active name.


\subsection{ Define-button.}
Clicking this button will display the full {\cal AUTO}-constants panel.
Most of its text fields can be edited,
but some have restricted input values that can be selected with
the right mouse button.
Some text fields will display a subpanel for entering data.
To actually apply changes made in the panel, click the
{\it OK-} or {\it Apply}-button at the bottom of the panel.



\subsection{ Run-button.}
Clicking this button will write the constants-file {\tt c.xxx} and run {\cal AUTO}.
If the equations-file has been edited then it should first be rewritten 
with the {\it Write}-button. 


\subsection{ Save-button.}
This pull-down menu contains the item
{\it Save},
to save the output-files {\tt fort.7}, {\tt fort.8}, {\tt fort.9},
as {\tt b.xxx}, {\tt s.xxx}, {\tt d.xxx}, respectively.
Here {\tt xxx} is the active equation name.
It also contains the item
{\it Save As}, 
to save the output-files under another name. 
Existing data-files with the selected name, if any, will be overwritten.


\subsection{ Append-button.}
This pull-down menu contains the item
{\it Append},
to append the output-files {\tt fort.7}, {\tt fort.8}, {\tt fort.9},
to existing data-files {\tt b.xxx}, {\tt s.xxx}, {\tt d.xxx}, respectively.
Here {\tt xxx} is the active equation name.
It also contains the item
{\it Append To}, 
to append the output-files to other existing data-files.

\subsection{ Plot-button.}
This pull-down menu contains the items
{\it Plot},
to run the plotting program {\cal PLAUT} for the data-files 
{\tt b.xxx} and {\tt s.xxx},
where {\tt xxx} is the active equation name,
and the item
{\it Name}, 
to run {\cal PLAUT} with other data-files.


\subsection{ Files-button.}
This pull-down menu contains 
the item 
{\it Restart}, to redefine the restart file.
Normally, when restarting from a previously computed solution,
the restart data is expected in the file {\tt s.xxx},
where {\tt xxx} is the active equation name.
Use the {\it Restart}-button to read the restart data from another data-file
in the immediately following run.  
The pull-down menu also contains the following items~:
\begin{itemize}
\item[-]{\it Copy},~ to copy  
  {\tt b.xxx}, {\tt s.xxx}, {\tt d.xxx}, {\tt c.xxx},
  to
  {\tt b.yyy}, {\tt s.yyy}, {\tt d.yyy}, {\tt c.yyy}, resp.;

\item[-]{\it Append},~ to append data-files
  {\tt b.xxx}, {\tt s.xxx}, {\tt d.xxx},
  to
  {\tt b.yyy}, {\tt s.yyy}, {\tt d.yyy}, resp.;

\item[-]{\it Move},~ to move 
  {\tt b.xxx}, {\tt s.xxx}, {\tt d.xxx}, {\tt c.xxx},
  to
  {\tt b.yyy}, {\tt s.yyy}, {\tt d.yyy}, {\tt c.yyy}, resp.;

\item[-]{\it Delete},~ to delete data-files
  {\tt b.xxx}, {\tt s.xxx}, {\tt d.xxx};  

\item[-]{\it Clean}, to delete all files of the form 
  {\tt fort.*}, {\tt *.o}, and {\tt *.exe}.  
\end{itemize}


\subsection{ Demos-button.}
This pulldown menu contains the items
{\it Select},
to view and run a selected {\cal AUTO} demo in the demo directory,
and
{\it Reset},
to restore the demo directory to its original state.
Note that demo files can be copied to the user work directory
with the {\it Equations/Demo}-button.


\subsection{ Misc.-button.}
This pulldown menu contains the items
{\it Tek Window}
and
{\it VT102 Window},
for opening windows;
{\it Emacs}
and
{\it Xedit},
for editing files,
and
{\it Print}, for printing the active equations-file {\tt xxx.f}.


\subsection{ Help-button.}
This pulldown menu contains the items
{\it {\cal AUTO}-constants}, for help on {\cal AUTO}-constants,
and
{\it User Manual}, for viewing the user manual; i.e., this document.


\section{ Using the GUI.} \label{sec:Using_the_GUI}
{\cal AUTO}-commands are described in Section~\ref{sec:command_mode} and
illustrated in the demos.
In Table~\ref{tbl:CM_GUI} we list the main {\cal AUTO}-commands 
together with the corresponding GUI button.

\begin{table}[htbp]
\begin{center}
\begin{tabular}{| l | l |}
\hline
{\it @r }  & {\it Run} \\  
\hline
{\it @sv }  & {\it Save}  \\ 
\hline
{\it @ap }  & {\it Append} \\ 
\hline
{\it @p }  & {\it Plot}  \\ 
\hline
{\it @cp }  & {\it Files/Copy}  \\ 
\hline
{\it @mv }  & {\it Files/Move}  \\ 
\hline
{\it @cl }  & {\it Files/Clean} \\ 
\hline
{\it @dl }  & {\it Files/Delete} \\  
\hline
{\it @dm }  & {\it Equations/Demo} \\  
\hline
\end{tabular}
\caption{Command Mode - GUI correspondences.}
\label{tbl:CM_GUI}
\end{center}
\end{table}


The {\cal AUTO}-command {\it @r xxx yyy} is given in the GUI as follows~:
click {\it Files/Restart} and enter {\it yyy} as data.
Then click {\it Run}.
As noted in Section~\ref{sec:command_mode}, 
this will run {\cal AUTO} with the current equations-file
{\tt xxx.f} and the current constants-file {\tt c.xxx}, 
while expecting restart data in {\tt s.yyy}.
The {\cal AUTO}-command {\it @ap xxx yyy} is given in the GUI by
clicking {\it Files/Append}.

\section{ Customizing the GUI.} \label{sec:Customizing_the_GUI}
\subsection{ Print-button.}
The {\it Misc/Print}-button on the Menu Bar can be customized 
by editing the file {\tt GuiConsts.h} in directory {\tt auto/07/include}.

\subsection{ GUI colors.}
GUI colors can be customized by creating an X resource file.
Two model files can be found in directory {\tt auto/07/gui}, namely,
{\tt Xdefaults.1} and {\tt Xdefaults.2}.
To become effective, edit one of these, if desired,
and copy it to {\tt .Xdefaults} in your home directory.
Color names can often be found in the system file {\tt /usr/lib/X11/rgb.txt}.

\subsection{ On-line help.}
The file {\tt auto/07/include/GuiGlobal.h}
contains on-line help on {\cal AUTO}-constants and demos.
The text can be updated, subject to a modifiable maximum length.
On SGI machines this is 10240 bytes,
which can be increased, for example, to 20480 bytes, 
by replacing the line
{\it CC = cc -Wf, -XNl10240 -O}
in {\tt auto/07/gui/Makefile} by
{\it CC = cc -Wf, -XNl20480 -O}
On other machines, the maximum message length is the system defined maximum
string literal length.


%==============================================================================
%==============================================================================
\chapter{ Description of {\cal AUTO}-Constants.} \label{ch:AUTO_constants}
%==============================================================================
%==============================================================================
\section{ The {\cal AUTO}-Constants File.} \label{sec:The_AUTO_constants_file}
As described in Section~\ref{ch:User_supplied_files}, 
if the equations-file is {\tt xxx.f} 
then the constants that define the computation 
are normally expected in the file  {\tt c.xxx}.
The general format of this file is the same for all {\cal AUTO} runs.
For example, the file {\tt c.ab} 
in directory {\tt auto/07/demos/ab} is listed below.
(The tutorial demo {\tt ab} is described in detail in 
Chapter~\ref{ch:Demos:_Tutorial}.)  

\begin{verbatim}
2 1 0 1               NDIM,IPS,IRS,ILP
1   1                 NICP,(ICP(I),I=1,NICP)
50 4 3 1 1 0 0 0      NTST,NCOL,IAD,ISP,ISW,IPLT,NBC,NINT
100 0. 0.15 0. 100.   NMX,RL0,RL1,A0,A1
100 10 2 8 5 3 0      NPR,MXBF,IID,ITMX,ITNW,NWTN,JAC
1.e-6 1.e-6 0.0001    EPSL,EPSU,EPSS
0.01 0.005 0.05 1     DS,DSMIN,DSMAX,IADS
1                     NTHL,((I,THL(I)),I=1,NTHL)
11 0.
0                     NTHU,((I,THU(I)),I=1,NTHU)
0                     NUZR,((I,UZR(I)),I=1,NUZR)
\end{verbatim}

The significance of the {\cal AUTO}-constants, grouped by function, is 
described in the sections below. 
Representative demos that illustrate use of the {\cal AUTO}-constants
are also mentioned.

%=====================================================================
\section{ Problem Constants.} \label{sec:Problem_constants}
\subsection{\tt NDIM} \label{sec:NDIM}
 Dimension of the system of equations as specified in the user-supplied
 routine {\tt FUNC}.

\subsection{\tt NBC}  \label{sec:NBC}
 The number of boundary conditions as specified in the user-supplied
 routine {\tt BCND}. \\
(Demos {\tt exp}, {\tt kar}.)

\subsection{\tt NINT}  \label{sec:NINT}
 The number of integral conditions as specified in the user-supplied
 routine {\tt ICND}. \\ 
(Demos {\tt int}, {\tt lin}, {\tt obv}.)

\subsection{\tt JAC}  \label{sec:JAC}
 Used to indicate whether derivatives are supplied by the user
 or to be obtained by differencing~:
\begin{itemize}
\item[-] {\tt JAC=0}~: 
  No derivatives are given by the user. (Most demos use {\tt JAC}=0.)
\item[-] {\tt JAC=1}~:  
  Derivatives with respect to state- and problem-parameters are given 
  in the user-supplied routines 
  {\tt FUNC}, {\tt BCND}, {\tt ICND} and {\tt FOPT}, where 
  applicable.  This may be necessary for sensitive problems. 
  It is also recommended for computations in which {\cal AUTO} generates 
  an extended system, for example, when {\tt ISW}=2.
  (For {\tt ISW} see Section~\ref{sec:ISW}.) \\
(Demos {\tt int}, {\tt dd2}, {\tt obt}, {\tt plp}, {\tt ops}.)
\end{itemize}
%=====================================================================
\section{ Discretization Constants.} \label{sec:Discretization_constants}
\subsection{\tt NTST}  \label{sec:NTST}
 The number of mesh intervals used for discretization.
 {\tt NTST} remains fixed during any particular run, but can be changed
 when restarting. 
 (For mesh adaption see {\tt IAD} in Section~\ref{sec:IAD}.)
 Recommended value of {\tt NTST} : As small as possible to maintain convergence. \\ 
 (Demos {\tt exp}, {\tt ab}, {\tt spb}.)


\subsection{\tt NCOL}  \label{sec:NCOL}
 The number of Gauss collocation points per mesh interval,
 (2 $\le$ {\tt NCOL} $\le$ 7).
 {\tt NCOL} remains fixed during any given run, but can be changed
 when restarting at a previously computed solution.
 The choice {\tt NCOL}=4, used in most demos, is recommended.
 If {\tt NDIM} is ``large'' and the solutions ``very smooth'' then
 {\tt NCOL}=2 may be appropriate.

\subsection{\tt IAD} \label{sec:IAD}
This constant controls the mesh adaption~: 
\begin{itemize}
\item[-] {\tt IAD=0}~:
  Fixed mesh. Normally, this choice should never be used, as it may result
  in spurious solutions. (Demo {\tt ext}.)
\item[-] {\tt IAD$>$0}~:  
  Adapt the mesh every {\tt IAD} steps along the family.
  Most demos use {\tt IAD=3}, which is the strongly recommended value.
\end{itemize}

When computing  ``trivial'' solutions to a boundary value problem,
for example, when all solution components are constant, then the
mesh adaption may fail under certain circumstances, and overflow may
occur. In such case, try recomputing the solution family with a fixed
mesh {\tt (IAD=0)}. Be sure to set  {\tt IAD} back to {\tt IAD=3} 
for computing eventual non-trivial bifurcating solution families.
%=====================================================================
\section{ Tolerances.} \label{sec:Tolerances}
\subsection{\tt EPSL}  \label{sec:EPSL}
 Relative convergence criterion for equation parameters in the Newton/Chord 
 method. Most demos use {\tt EPSL}=$10^{-6}$ or {\tt EPSL}=$10^{-7}$,
 which is the recommended value range.

\subsection{\tt EPSU}  \label{sec:EPSU}
 Relative convergence criterion for solution components in the Newton/Chord 
 method. Most demos use {\tt EPSU}=$10^{-6}$ or {\tt EPSU}=$10^{-7}$,
 which is the recommended value range.

\subsection{\tt EPSS}  \label{sec:EPSS}
 Relative arclength convergence criterion for the detection of special solutions. 
 Most demos use {\tt EPSS}=$10^{-4}$ or  {\tt EPSS}=$10^{-5}$,
 which is the recommended value range.
 Generally, {\tt EPSS} should be approximately 100 to 1000 times the value
 of {\tt EPSL}, {\tt EPSU}.
 
\subsection{\tt ITMX}  \label{sec:ITMX}
 The maximum number of iterations allowed in the accurate
 location of special solutions, such as bifurcations, folds, 
 and user output points, by M\"uller's method with bracketing.
 The recommended value is {\tt ITMX}=8, used in most demos.

\subsection{\tt NWTN}  \label{sec:NWTN}
 After {\tt NWTN} Newton iterations the Jacobian is frozen, i.e.,
 {\cal AUTO} uses full Newton for the first {\tt NWTN} iterations
 and the Chord method for iterations {\tt NWTN}+1 to {\tt ITNW}.
 The choice {\tt NWTN}=3 is strongly recommended and used in most demos.
 Note that this constant is only effective for ODEs, i.e., for solving
 the piecewise polynomial collocation equations.
 For algebraic systems {\cal AUTO} always uses full Newton.

\subsection{\tt ITNW}  \label{sec:ITNW}
 The maximum number of combined Newton-Chord iterations.
 When this maximum is reached, the step will be retried with 
 half the stepsize.
 This is repeated until convergence, or until the minimum
 stepsize is reached. In the latter case the computation of
 the family is discontinued and a message printed in {\tt fort.9}.
 The recommended value is {\tt ITNW}=5, but {\tt ITNW}=7 may be used for 
 ``difficult'' problems, for example, 
 demos {\tt spb}, {\tt chu}, {\tt plp}, etc.

%=====================================================================
\section{ Continuation Step Size.} \label{sec:step_size}
\subsection{\tt DS}  \label{sec:DS}
 {\cal AUTO} uses pseudo-arclength continuation for following solution families.
 The pseudo-arclength stepsize is the distance between
 the current solution and the next solution on a family.
 By default, this distance includes all state variables
 (or state functions) and all free parameters.
 The constant {\tt DS} defines the pseudo-arclength stepsize to be used for the
 first attempted step along any family. 
 (Note that if {\tt IADS}$>$0 then {\tt DS} will automatically be adapted
 for subsequent steps and for failed steps.)
 {\tt DS} may be chosen positive or negative; changing its sign 
 reverses the direction of computation.
 The relation {\tt DSMIN} $\le$ $\abs {\tt DS}$ $\le$ {\tt DSMAX} must be satisfied.
 The precise choice of {\tt DS} is problem-dependent; the demos use a value 
 that was found appropriate after some experimentation.
 

\subsection{\tt DSMIN}  \label{sec:DSMIN}
 This is minimum allowable absolute value of the pseudo-arclength 
 stepsize. {\tt DSMIN} must be positive.
 It is only effective if the pseudo-arclength step is adaptive,
 i.e., if {\tt IADS}$>$0.
 The choice of {\tt DSMIN} is highly problem-dependent;
 most demos use a value that was found appropriate after some
 experimentation.
 See also the discussion in Section~\ref{sec:Efficiency}.

\subsection{\tt DSMAX}  \label{sec:DSMAX}
 The maximum allowable absolute value of the pseudo-arclength stepsize.
 {\tt DSMAX} must be positive.
 It is only effective if the pseudo-arclength step is adaptive,
 i.e., if {\tt IADS}$>$0.
 The choice of {\tt DSMAX} is highly problem-dependent; 
 most demos use a value that was found appropriate after some
 experimentation.
 See also the discussion in Section~\ref{sec:Efficiency}.

\subsection{\tt IADS}  \label{sec:IADS}
This constant controls the frequency of adaption of the 
pseudo-arclength stepsize.
\begin{itemize}
\item[-] {\tt IADS=0}~: 
  Use fixed pseudo-arclength stepsize, i.e., the stepsize will
  be equal to the specified value of {\tt DS} for every step.
  The computation of a family will be discontinued as soon as
  the maximum number of iterations {\tt ITNW} is reached.
  This choice is not recommended. \\(Demo {\tt tim}.)
\item[-] {\tt IADS$>$0}~:  
 Adapt the pseudo-arclength stepsize after every {\tt IADS} steps.
  If the Newton/Chord iteration converges rapidly then 
  $\abs{\tt DS}$ will be increased, but never beyond {\tt DSMAX}.
  If a step fails then it will be retried with half
  the stepsize. This will be done repeatedly until the
  step is successful or until $\abs{\tt DS}$ reaches {\tt DSMIN}. 
  In the latter case nonconvergence will be signalled.
  The strongly recommended value is {\tt IADS}=1, which is used in 
  almost all demos.
\end{itemize}
  
\subsection{\tt NTHL}  \label{sec:NTHL}
By default, the pseudo-arclength stepsize includes all state variables
(or state functions) and all free parameters.
Under certain circumstances one may want to modify the weight accorded 
to individual parameters in the definition of stepsize.
For this purpose, {\tt NTHL} defines the number of parameters whose weight 
is to be modified.
If {\tt NTHL}=0 then all weights will have default value 1.0~.
If {\tt NTHL}$>$0 then one must enter {\tt NTHL} pairs,
             ~{\it Parameter Index} ~ {\it Weight}~,
with each pair on a separate line.

For example, for the computation of periodic solutions it is 
recommended that the period not be included in the pseudo-arclength 
continuation stepsize, in order to avoid period-induced limitations 
on the stepsize near orbits of infinite period. 
This exclusion can be accomplished by setting {\tt NTHL=1}, with, 
on a separate line, the pair ~ 11 ~ 0.0 ~.
Most demos that compute periodic solutions use this option;
see for example demo {\tt ab}.

\subsection{\tt NTHU}  \label{sec:NTHU}
Under certain circumstances one may want to modify the weight accorded 
to individual state variables (or state functions) in the definition 
of stepsize.
For this purpose, {\tt NTHU} defines the number of states whose weight 
is to be modified.
If {\tt NTHU}=0 then all weights will have default value 1.0~.
If {\tt NTHU}$>$0 then one must enter {\tt NTHU} pairs,
             ~{\it State Index} ~ {\it Weight}~,
with each pair on a separate line.
At present none of the demos use this option.
%=====================================================================
\section{ Diagram Limits.} \label{sec:Diagram_limits}

There are three ways to limit the computation of a family~:

\begin{itemize}
\item[-]
By appropriate choice of the computational window 
defined by the constants {\tt RL0}, {\tt RL1}, {\tt A0}, and {\tt A1}.
One should always check that the starting solution lies within
this computational window, otherwise the computation will stop immediately
at the starting point.

\item[-]
By specifying the maximum number of steps, {\tt NMX}.

\item[-]
By specifying a negative parameter index in the list associated 
with the constant {\tt NUZR}; see Section~\ref{sec:NUZR}. 
\end{itemize}

\subsection{\tt NMX} \label{sec:NMX}
The maximum number of steps to be taken along any family.

\subsection{\tt RL0}  \label{sec:RL0}
 The lower bound on the principal continuation parameter.
 (This is the parameter which appears first in the {\tt ICP} list;
 see Section~\ref{sec:ICP}.). 


\subsection{\tt RL1}  \label{sec:RL1}
 The upper bound on the principal continuation parameter. 

\subsection{\tt A0}  \label{sec:A0}
 The lower bound on the principal solution measure.
 (By default, if {\tt IPLT}=0, the principal solution measure
 is the $L_2$-norm of the state vector or state vector function.
 See the {\cal AUTO}-constant {\tt IPLT} in Section~\ref{sec:IPLT} 
 for choosing another principal solution measure.)

\subsection{\tt A1}  \label{sec:A1}
 The upper bound on the principal solution measure.

%=====================================================================
%=====================================================================
\section{ Free Parameters.} \label{sec:Free_parameters}


\subsection{\tt NICP, ICP}  \label{sec:ICP}
For each equation type and for each continuation calculation there is
a typical (``generic'') number of problem parameters that must be 
allowed to vary, in order for the calculations to be properly posed.
The constant {\tt NICP} indicates how many free parameters have been specified,
while the array {\tt ICP} actually designates these free parameters.
The parameter that appears first in the {\tt ICP} list is called the 
``principal continuation parameter''.
Specific examples and special cases are described below.

%=====================================================================
\subsection{ Fixed points.}
The simplest case is the continuation of a solution family to the system
$ f( u , p ) = 0$,  where $f(\cdot,\cdot), u \in \Rn$, cf. Equation~(\ref{1}).
Such a system arises in the continuation of ODE stationary solutions and 
in the continuation of fixed points of discrete dynamical systems.
There is only one free parameter here, so {\tt NICP}=1.

As a concrete example, consider Run~1 of demo {\tt ab},
where {\tt NICP=1}, with {\tt ICP(1)=1}. 
Thus, in this run {\tt PAR(1)} is designated as the free parameter.

%=====================================================================
\subsection{ Periodic solutions and rotations.}
The continuation of periodic solutions and rotations generically requires 
two parameters, namely, one problem parameter and the period.
Thus, in this case  {\tt NICP}=2.
For example, in Run~2 of demo {\tt ab} we have {\tt NICP}=2,
with {\tt ICP}(1)=1 and {\tt ICP}(2)=11.
Thus, in this run, the free parameters are {\tt PAR(1)} and {\tt PAR(11)}.
(Note that {\cal AUTO} reserves {\tt PAR(11)} for the period.)

Actually, for periodic solutions, one can set {\tt NICP}=1 and only specify 
the index of the free problem parameter, as {\cal AUTO} will automatically 
addd {\tt PAR(11)}.
However, in this case the period will not appear in the screen output 
and in the {\tt fort.7} output-file. 

For fixed period orbits one must set {\tt NICP}=2 and specify two free problem 
parameters.
For example, in Run~7 of demo {\tt pp2}, we have {\tt NICP}=2, with 
{\tt PAR(1)} and {\tt PAR(2)}
specified as free problem parameters.
The period {\tt PAR(11)} is fixed in this run.
If the period is large then such a continuation provides a simple and 
effective method for computing a locus of homoclinic orbits.
%=====================================================================
\subsection{ Folds and Hopf bifurcations.}
The continuation of folds for algebraic problems and the continuation of
Hopf bifurcations requires two free problem parameters, i.e.,  {\tt NICP}=2.
For example, to continue a fold in Run~3 of demo {\tt ab}, we have {\tt NICP}=2, 
with {\tt PAR(1)} and {\tt PAR(3)} specified as free parameters.
Note that one must set {\tt ISW}=2 for computing such loci of special solutions.
Also note that in the continuation of folds the principal continuation parameter
must be the one with respect to which the fold was located.

%=====================================================================
\subsection{ Folds and period-doublings.}
The continuation of folds, for periodic orbits and rotations,
and the continuation of period-doubling bifurcations require two free 
problem parameters plus the free period. Thus, one would normally set {\tt NICP}=3.
For example, in Run~6 of demo {\tt pen}, where a locus of period-doubling
bifurcations is computed for rotations, we have {\tt NICP}=3, 
with {\tt PAR(2)}, {\tt PAR(3)}, and {\tt PAR(11)} specified as free parameters. 
Note that one must set {\tt ISW}=2 for computing such loci of special solutions.
Also note that in the continuation of folds the principal continuation parameter
must be the one with respect to which the fold was located.

Actually, one may set {\tt NICP}=2, and only specify the problem parameters,
as {\cal AUTO} will automatically add the period.
For example, in Run~3 of demo {\tt plp}, where a locus of folds is computed 
for periodic orbits, we have {\tt NICP}=2, with {\tt PAR(4)} and {\tt PAR(1)} specified
as free parameters. 
However, in this case the period will not appear in the screen output 
and in the {\tt fort.7} output-file. 

To continue a locus of folds or period-doublings with fixed period, simply
set {\tt NICP}=3 and specify three problem parameters, not including {\tt PAR(11)}.

%=====================================================================
\subsection{ Boundary value problems.}
The simplest case is that of boundary value problems where 
{\tt NDIM}={\tt NBC} 
and where {\tt NINT}=0.
Then, generically, one free problem parameter is required for computing 
a solution family.
For example, in demo {\tt exp}, we have {\tt NDIM}={\tt NBC}=2, {\tt NINT}=0. 
Thus {\tt NICP}=1.
Indeed, in this demo one free parameter is designated,
namely {\tt PAR(1)}.

More generally, for boundary value problems with integral constraints,
the generic number of free parameters is {\tt NBC} + {\tt NINT}$-${\tt NDIM} +1.
For example, in demo {\tt lin}, we have {\tt NDIM}=2, {\tt NBC}=2, and {\tt NINT}=1.
Thus {\tt NICP}=2. 
Indeed, in this demo two free parameters are designated,
namely {\tt PAR(1)} and {\tt PAR(3)}.

%=====================================================================
\subsection{ Boundary value folds.}
To continue a locus of folds for a general boundary value problem
with integral constraints, set {\tt NICP}={\tt NBC}+{\tt NINT}$-${\tt NDIM}+2, 
and specify this number of parameter indices to designate the free parameters.

%=====================================================================
\subsection{ Optimization problems.}
In algebraic optimization problems one must set {\tt ICP}(1)=10, 
as {\cal AUTO} uses {\tt PAR(10)} as principal continuation parameter
to monitor the value of the objective function.
Furthermore, one must designate one free equation parameter in {\tt ICP}(2). 
Thus, {\tt NICP}=2 in the first run.

Folds with respect to {\tt PAR(10)} correspond to extrema of the objective function.
In a second run one can restart at such a fold, with an additional
free equation parameter specified in {\tt ICP}(3).
Thus, {\tt NICP}=3 in the second run.

The above procedure can be repeated.
For example, folds from the second run can be continued in a third run
with three equation parameters specified in addition to {\tt PAR(10)}.
Thus, {\tt NICP}=4 in the third run.

For a simple example see demo {\tt opt}, where a four-parameter extremum
is located.
Note that {\tt NICP}=5 in each of the four constants-files of this demo, 
with the indices of {\tt PAR(10)} and {\tt PAR(1)-PAR(4)} specified in {\tt ICP}.
Thus, in the first three runs, there are overspecified parameters.
However, {\cal AUTO} will always use the correct number of parameters.
Although the overspecified parameters will be printed, their values will
remain fixed. 

%=====================================================================
\subsection{ Internal free parameters.}
The actual continuation scheme in {\cal AUTO} may use additional free
parameters that are automatically added.
The simplest example is the computation of periodic solutions and rotations,
where {\cal AUTO} automatically adds the period, if not specified.
The computation of loci of folds, Hopf bifurcations, and period-doublings
also requires additional internal continuation parameters.
These will be automatically added, and their indices will be greater
than 10.


%=====================================================================
\subsection{ Parameter overspecification.} \label{sec:Parameter_over_specification}
The number of specified parameter indices is allowed to be be greater 
than the generic number.
In such case there will be ``overspecified'' parameters, whose values
will appear in the screen and {\tt fort.7} output, but which are not
part of the continuation process.
A simple example is provided by demo {\tt opt}, where the first three runs
have overspecified parameters whose values, although constant, are printed.

There is, however, a more useful application of parameter overspecification.
In the user-supplied routine {\tt PVLS} one can define solution measures
and assign these to otherwise unused parameters.
Such parameters can then be overspecified, in order to print them
on the screen and in the {\tt fort.7} output.
It is important to note that such overspecified parameters must appear
at the end of the {\tt ICP} list, as they cannot be used as true continuation
parameters.

For an example of using parameter overspecification for printing user-defined
solution measures, see demo {\tt pvl}.
This is a boundary value problem (Bratu's equation) which has
only one true continuation parameter, namely {\tt PAR(1)}.
Three solution measures are defined in the routine {\tt PVLS}, namely,
the $L_2$-norm of the first solution component,
the minimum of the second component, and
the left boundary value of the second component.
These solution measures are assigned to {\tt PAR(2), PAR(3)}, and {\tt PAR(4)}, respectively.
In the constants-file {\tt c.pvl} we have {\tt NICP}=4, with {\tt PAR(1)-PAR(4)}
specified as parameters.
Thus, in this example, {\tt PAR(2)-PAR(4)} are overspecified.
Note that {\tt PAR(1)} must appear first in the {\tt ICP} list;
the other parameters cannot be used as true continuation parameters.
%=====================================================================
%=====================================================================
\section{ Computation Constants.} \label{sec:Computation_constants}
\subsection{\tt ILP}  \label{sec:ILP}
\begin{itemize}
\item[-] {\tt ILP=0}~: 
  No detection of folds. This choice is recommended.
\item[-] {\tt ILP=1}~: 
  Detection of folds. To be used if subsequent fold continuation is intended.
\end{itemize}
 
\subsection{\tt ISP}  \label{sec:ISP}
This constant controls the detection of branch points,
period-doubling bifurcations, and torus bifurcations. 
\begin{itemize}
\item[-] {\tt ISP=0}~:  
  This setting disables the detection of branch points, period-doubling 
  bifurcations, and torus bifurcations and the computation of 
  Floquet multipliers.
\item[-] {\tt ISP=1}~:  
  Branch points are detected for algebraic equations, but not for
  periodic solutions and boundary value problems.
  Period-doubling bifurcations and torus bifurcations are not located either.
  However, Floquet multipliers are computed.
\item[-] {\tt ISP=2}~: This setting enables the detection of all special 
 solutions.
 For periodic solutions and rotations, the choice {\tt ISP}=2 should be used with
 care, due to potential inaccuracy in the computation of the
 linearized Poincar\'e map and possible rapid variation of the
 Floquet multipliers.
 The linearized Poincar\'e map always has a multiplier $z=1$.
 If this multiplier becomes inaccurate
 then the automatic detection of secondary periodic
 bifurcations will be discontinued and a
 warning message will be printed in {\tt fort.9}.
 See also Section~\ref{sec:Bifurcations}.
\item[-] {\tt ISP=3}~:  
  Branch points will be detected, but {\cal AUTO} will not monitor the 
  Floquet multipliers. Period-doubling and torus bifurcations will go undetected. 
  This option is useful for certain problems with non-generic Floquet behavior.
\end{itemize}

\subsection{\tt ISW}  \label{sec:ISW}
 This constant controls branch switching at branch points for the case
 of differential equations.
 Note that branch switching is automatic for algebraic equations.
\begin{itemize}
\item[-] {\tt ISW=1}~: This is the normal value of {\tt ISW}.
\item[-] {\tt ISW=$-$1}~:
  If {\tt IRS} is the label of a branch point or a period-doubling
  bifurcation then branch switching will be done.
  For period doubling bifurcations it is recommended that {\tt NTST} be increased.
  For examples see Run~2 and Run~3 of demo {\tt lor}, where branch switching
  is done at period-doubling bifurcations, and Run~2 and Run~3 of demo {\tt bvp},
  where branch switching is done at a transcritical branch point.
\item[-] {\tt ISW=2}~:
  If {\tt IRS} is the label of a fold, a Hopf bifurcation point, 
  or a period-doubling or torus bifurcation then a locus of such points will be
  computed. An additional free parameter must be specified for such 
  continuations; see also Section~\ref{sec:Free_parameters}.
\end{itemize}

\subsection{\tt MXBF}  \label{sec:MXBF}
 This constant, which is effective for algebraic problems only,
 sets the maximum number of bifurcations to be treated.
 Additional branch points will be noted, but the corresponding bifurcating
 families will not be computed.
 If {\tt MXBF} is positive then the bifurcating families of the first {\tt MXBF}
  branch points will be traced out in both directions.
 If {\tt MXBF} is negative then the bifurcating families of the first 
 $\abs{{\tt MXBF}}$ branch points will be traced out in only one direction. 

\subsection{\tt IRS}  \label{sec:IRS}
This constant sets the label of the solution where the computation
is to be restarted.
\begin{itemize}
\item[-] {\tt IRS=0}~:  
  This setting is typically used in the first run of a new problem.
  In this case a starting solution must be defined in the user-supplied
  routine {\tt STPNT}.
  For representative examples of analytical starting solutions 
  see demos {\tt ab} and {\tt frc}.
  For starting from unlabeled numerical data see the {\it @fc} command
  (Section~\ref{sec:command_mode}) and demos {\tt lor} and {\tt pen}.
  
\item[-] {\tt IRS$>$0}~: 
  Restart the computation at the previously computed solution with label {\tt IRS}. 
  This solution is normally expected to be in the current data-file 
 {\tt s.xxx}; see also the {\it @r} and {\it @R} commands in 
 Section~\ref{sec:command_mode}.
 Various {\cal AUTO}-constants can be modified when restarting.
\end{itemize}

\subsection{\tt IPS}  \label{sec:IPS}
This constant defines the problem type~:
\begin{itemize}
%=====================================================================
\item[-] {\tt IPS=0}~: 
  An algebraic bifurcation problem.
  Hopf bifurcations will not be detected and stability
  properties will not be indicated in the {\tt fort.7} output-file.
%=====================================================================
\item[-] {\tt IPS=1}~: 
  Stationary solutions of ODEs with detection of Hopf bifurcations.
  The sign of PT, the point number, in {\tt fort.7} is used 
  to indicate stability~: $-$ is stable , + is unstable.\\
 (Demo {\tt ab}.)
%=====================================================================
\item[-] {\tt IPS=$-$1}~:  
  Fixed points of the discrete dynamical system
  $u^{(k+1)}=f(u^{(k)},p ),$ with detection of Hopf bifurcations.
  The sign of PT in {\tt fort.7} indicates stability~: 
  $-$ is stable , + is unstable.  
 (Demo {\tt dd2}.)
%=====================================================================
\item[-] {\tt IPS=$-$2}~: 
  Time integration using implicit Euler. 
  The {\cal AUTO}-constants {\tt DS}, {\tt DSMIN}, {\tt DSMAX}, and {\tt ITNW}, {\tt NWTN} control 
  the stepsize.
  In fact, pseudo-arclength is used for ``continuation in time''. 
  Note that the time discretization is only first order accurate, 
  so that results should be carefully interpreted. 
  Indeed, this option has been included primarily for the detection 
  of stationary solutions, which can then be entered in the user-supplied
  routine {\tt STPNT}.  \\  
 (Demo {\tt ivp}.)
%=====================================================================
\item[-]  {\tt IPS=2}~:
  Computation of periodic solutions. Starting data can be
  a Hopf bifurcation point (Run~2 of demo {\tt ab}),
  a periodic orbit from a previous run (Run~4 of demo {\tt pp2}),
  an analytically known periodic orbit (Run~1 of demo {\tt frc}),
  or a numerically known periodic orbit (Demo {\tt lor}).
  The sign of PT in {\tt fort.7} is used to indicate
  stability~: $-$ is stable , + is unstable or unknown.
%=====================================================================
\item[-] {\tt IPS=4}~: 
  A boundary value problem. Boundary conditions must be
  specified in the user-supplied routine {\tt BCND}
  and integral constraints in {\tt ICND}. The {\cal AUTO}-constants
  {\tt NBC} and {\tt NINT} must be given correct values.
 (Demos {\tt exp}, {\tt int}, {\tt kar}.)
%=====================================================================
\item[-] {\tt IPS=5}~:
  Algebraic optimization problems. The objective function
  must be specified in the user-supplied routine {\tt FOPT}. 
 (Demo {\tt opt}.)
%=====================================================================
\item[-] {\tt IPS=7}~:
  A boundary value problem with computation of Floquet multipliers. 
  This is a very special option; for most boundary value problems 
  one should use {\tt IPS=4}.
  Boundary conditions must be
  specified in the user-supplied routine {\tt BCND}
  and integral constraints in {\tt ICND}. The {\cal AUTO}-constants
  {\tt NBC} and {\tt NINT} must be given correct values.
%=====================================================================
\item[-] {\tt IPS=9}~:
  This option is used in connection with the {\cal HomCont} algorithms
  described in 
  Chapters~\ref{ch:HomCont}-\ref{ch:HomCont_rev}
  for the  detection and continuation of homoclinic bifurcations.\\  
 (Demos {\tt san}, {\tt mtn}, {\tt kpr}, {\tt cir}, {\tt she},
  {\tt rev}.)
%=====================================================================
\item[-] {\tt IPS=11}~: 
  Spatially uniform solutions of a system of parabolic PDEs,
  with detection of traveling wave bifurcations.
  The user need only define the nonlinearity (in routine {\tt FUNC}),
  initialize the wave speed in {\tt PAR(10)}, initialize the diffusion 
  constants in {\tt PAR(15,16,$\cdots$)}, and set a free equation parameter 
  in {\tt ICP}(1).
  (Run~2 of demo {\tt wav}.)
%=====================================================================
\item[-] {\tt IPS=12}~: 
  Continuation of traveling wave solutions to a system of parabolic PDEs.
  Starting data can be a Hopf bifurcation point from a previous run 
  with {\tt IPS}=11, or a traveling wave from a previous run with {\tt IPS}=12.
  (Run~3  and Run~4 of demo {\tt wav}.)
%=====================================================================
\item[-] {\tt IPS=14}~:  
  Time evolution for a system of parabolic PDEs subject to periodic 
  boundary conditions. 
  Starting data may be solutions from a previous run with {\tt IPS}=12 or 14. 
  Starting data can also be specified in {\tt STPNT}, in which case
  the wave length must be specified in {\tt PAR(11)}, and the diffusion
  constants in {\tt PAR(15,16,$\cdots$)}.
  {\cal AUTO} uses {\tt PAR(14)} for the time variable.
  {\tt DS}, {\tt DSMIN}, and {\tt DSMAX} govern the pseudo-arclength continuation 
  in the space-time variables.
  Note that the time discretization is only first order accurate, 
  so that results should be carefully interpreted. 
  Indeed, this option is mainly intended for the detection of stationary 
  waves.
  (Run~5 of demo {\tt wav}.)
%=====================================================================
\item[-] {\tt IPS=15}~:   
  Optimization of periodic solutions. The integrand of the
  objective functional must be specified in the user supplied
  routine {\tt FOPT}. Only {\tt PAR(1-9)} should be used for
  problem parameters. {\tt PAR(10)} is the value of the objective
  functional, {\tt PAR(11)} the period, {\tt PAR(12)} the norm of the
  adjoint variables, {\tt PAR(14)} and {\tt PAR(15)} are internal optimality
  variables. {\tt PAR(21-29)} and {\tt PAR(31)} are used to monitor the 
  optimality functionals associated with the problem parameters 
  and the period. 
  Computations can be started at a solution computed with {\tt IPS}=2
  or {\tt IPS}=15.
  For a detailed example see demo {\tt ops}.
%=====================================================================
\item[-] {\tt IPS=16}~:
  This option is similar to {\tt IPS}=14, except that the user supplies the
  boundary conditions. Thus this option can be used for 
  time-integration of parabolic systems subject to 
  user-defined boundary conditions. For examples see the first runs
  of demos {\tt pd1}, {\tt pd2}, and {\tt bru}. Note that
  the space-derivatives of the initial conditions must
  also be supplied in the user supplied routine {\tt STPNT}. 
  The initial conditions must satisfy the boundary conditions.
  This option is mainly intended for the detecting stationary solutions.
%=====================================================================
 \item[-] {\tt IPS=17}~: 
  This option can be used to continue stationary solutions
  of parabolic systems obtained from an evolution run with {\tt IPS}=16.
  For examples see the second runs of demos {\tt pd1} and {\tt pd2}.
\end{itemize}
%=====================================================================


\section{ Output Control.} \label{sec:Output_control}
\subsection{\tt NPR}  \label{sec:NPR}
 This constant can be used to regularly write {\tt fort.8} plotting and restart 
 data.  
 IF {\tt NPR}$>$0 then such output is written every {\tt NPR} steps.
 IF {\tt NPR}$=$0 or if {\tt NPR}$\ge${\tt NMX} then no such output is written.
 Note that special solutions, such as branch points, folds, end points, etc., 
 are always written in {\tt fort.8}.
 Furthermore, one can specify parameter values where plotting and restart 
 data is to be written; see Section~\ref{sec:NUZR}.
 For these reasons, and to limit the output volume, it is recommended that
 {\tt NPR} output be kept to a minimum.

\subsection{\tt IID} \label{sec:IID} 
 This constant controls the amount of diagnostic output printed in {\tt fort.9}~:
 the greater {\tt IID} the more detailed the diagnostic output.
\begin{itemize}
\item[-] {\tt IID=0}~:  
  Minimal diagnostic output. This setting is not recommended.
\item[-] {\tt IID=2}~: 
  Regular diagnostic output. This is the recommended value of {\tt IID}.
\item[-] {\tt IID=3}~: 
  This setting gives additional diagnostic output for algebraic equations,
  namely the Jacobian and the residual vector at the starting point.
  This information, which is printed at the beginning of {\tt fort.9},
  is useful for verifying whether the starting solution in {\tt STPNT} is indeed 
  a solution.
\item[-] {\tt IID=4}~: 
  This setting gives additional diagnostic output for differential equations,
  namely the reduced system and the associated residual vector. 
  This information is printed for every step and for every Newton iteration,
  and should normally be suppressed.
  In particular it can be used to verify whether the starting solution
  is indeed a solution. For this purpose, the stepsize {\tt DS} should
  be small, and one should look at the residuals printed in the {\tt fort.9}
  output-file. (Note that the first residual vector printed in {\tt fort.9} may
  be identically zero, as it may correspond to the computation of the starting
  direction. Look at the second residual vector in such case.)
  This residual vector has dimension 
  {\tt NDIM}+{\tt NBC}+{\tt NINT}+1, which accounts for the {\tt NDIM}
  differential equations, the {\tt NBC} boundary conditions, the {\tt NINT} user-defined
  integral constraints, and the pseudo-arclength equation.
  For proper interpretations of these data one may want to refer to the solution
  algorithm for solving the collocation system, as described in
  \citename{DoKeKe:91b} \citeyear{DoKeKe:91b}.
\item[-] {\tt IID=5}~:
  This setting gives very extensive diagnostic output for differential equations,
  namely, debug output from the linear equation solver.
  This setting should not normally be used as it may result
  in a huge {\tt fort.9} file. 
\end{itemize}

\subsection{\tt IPLT}  \label{sec:IPLT}
 This constant allows redefinition of the principal solution measure, which is
 printed as the second (real) column in the screen output and in the {\tt fort.7}
 output-file~:
 
\begin{itemize}
\item[-]
  If {\tt IPLT} = 0 then the $L_2$-norm is printed. Most demos use this setting.
  For algebraic problems, the standard definition of $L_2$-norm is used.
  For differential equations, the $L_2$-norm is defined as 
  $$ \sqrt{ \int_0^1 \sum_{k=1}^{NDIM} U_k(x)^2 ~ dx}~.$$
  Note that the interval of integration is $[0,1]$, the standard interval
 used by AUTO. For periodic solutions the independent variable is transformed
 to range from 0 to 1, before the norm is computed. The AUTO-constants THL(*) 
 and THU(*) (see Section~\ref{sec:NTHL} and Section~\ref{sec:NTHU})
 affect the definition of the $L_2$-norm.
\item[-]
  If 0 $<$ {\tt IPLT} $\le$ {\tt NDIM} then the maximum of the {\tt IPLT}'th solution component 
  is printed.
\item[-]
  If $-${\tt NDIM} $\le$ {\tt IPLT} $<$0 then the minimum of the {\tt IPLT}'th solution component
  is printed.  (Demo {\tt fsh}.)
\item[-]
  If {\tt NDIM} $<$ {\tt IPLT} $\le$ 2*{\tt NDIM} then the integral 
  of the ({\tt IPLT}$-${\tt NDIM})'th 
  solution component is printed. (Demos {\tt exp}, {\tt lor}.)
\item[-]
  If 2*{\tt NDIM} $<$ {\tt IPLT} $\le$ 3*{\tt NDIM} 
  then the $L_2$-norm of the ({\tt IPLT}$-${\tt NDIM})'th 
  solution component is printed. (Demo {\tt frc}.)
\end{itemize}

Note that for algebraic problems the maximum and the minimum are identical.
Also, for ODEs the maximum and the minimum of a solution component are generally
much less accurate than the $L_2$-norm and component integrals.
Note also that the routine {\tt PVLS} provides a second, more general way
of defining solution measures; see Section~\ref{sec:Parameter_over_specification}.


\subsection{\tt NUZR} \label{sec:NUZR} 
 This constant allows the setting of parameter values at which labeled plotting 
 and restart information is to be written in the {\tt fort.8} output-file.
 Optionally, it also allows the computation to terminate at such a point.

\begin{itemize}
\item[-]
 Set {\tt NUZR}=0 if no such output is needed. Many demos use this setting.
\item[-]
 If {\tt NUZR}$>$0 then one must enter {\tt NUZR} pairs,
            ~{\it Parameter-Index} ~ {\it Parameter-Value}~,
 with each pair on a separate line, to designate the parameters and the parameter
 values at which output is to be written.
 For examples see demos {\tt exp}, {\tt int}, and {\tt fsh}.
\item[-]
 If such a parameter index is preceded by a minus sign then the computation will
 terminate at such a solution point.
 (Demos {\tt pen} and {\tt bru}.)
\end{itemize}

Note that {\tt fort.8} output can also be written at selected values of 
overspecified parameters. For an example see demo {\tt pvl}.
For details on overspecified parameters see 
Section~\ref{sec:Parameter_over_specification}.
%=====================================================================
 

%==============================================================================
%==============================================================================
\chapter{ Notes on Using {\cal AUTO}.}  \label{ch:Notes_on_Using_AUTO}
%==============================================================================
%==============================================================================
\section{ Restrictions on the Use of {\tt PAR}.} \label{sec:Restrictions_on_PAR}
The array {\tt PAR} in the user-supplied routines is available
for equation parameters that the user wants to vary at some point
in the computations.
In any particular computation the free parameter(s) must be designated
in {\tt ICP}; see Section~\ref{sec:Free_parameters}.
The following restrictions apply~:

\begin{itemize}
\item[-]
  The maximum number of parameters, {\tt NPARX} in {\tt auto/07/include/auto.h},
  has pre-defined value {\tt NPARX}=36.  {\tt NPARX} should not normally be increased
  and it should never be decreased.
  Any increase of {\tt NPARX} must be followed by recompilation of {\cal AUTO}.
\item[-]
  Generally one should only use {\tt PAR(1)-PAR(9)} for equation parameters,
  as {\cal AUTO} may need the other components internally.  
\end{itemize}

\section{ Efficiency.} \label{sec:Efficiency}
In {\cal AUTO}, efficiency has at times been sacrificed for generality of programming.
This applies in particular to computations in which {\cal AUTO} generates
an extended system, for example, computations with {\tt ISW}=2.
However, the user has significant control over computational efficiency,
in particular through judicious choice of the {\cal AUTO}-constants  
{\tt DS}, {\tt DSMIN}, and {\tt DSMAX}, and, for ODEs, {\tt NTST} and {\tt NCOL}.
Initial experimentation normally suggests appropriate values.

Slowly varying solutions to ODEs can often 
be computed with remarkably small values of {\tt NTST} and {\tt NCOL}, 
for example, {\tt NTST}=5,  {\tt NCOL}=2.
Generally, however, it is recommended to set {\tt NCOL}=4,
and then to use the ``smallest'' value of {\tt NTST} that maintains convergence.

The choice of the pseudo-arclength stepsize parameters
{\tt DS}, {\tt DSMIN}, and {\tt DSMAX}
is highly problem dependent.
Generally, {\tt DSMIN} should not be taken too small,
in order to prevent excessive step refinement in case of non-convergence.
It should also not be too large, in order to avoid instant non-convergence.
{\tt DSMAX} should be sufficiently large, in order to reduce computation time
and amount of output data.
On the other hand, it should be sufficiently small, in order to prevent
stepping over bifurcations without detecting them.
For a given equation, appropriate values of these constants 
can normally be found after some initial experimentation.

The constants {\tt ITNW}, {\tt NWTN}, {\tt THL}, {\tt EPSU}, {\tt EPSL}, {\tt EPSS} 
also affect efficiency.
Understanding their significance is therefore useful; 
see Section~\ref{sec:Tolerances} and Section~\ref{sec:step_size}.
Finally, it is recommended that initial computations be done with 
{\tt ILP=0}; no fold detection;
and {\tt ISP}=1; no bifurcation detection for ODEs.
 
\section{ Correctness of Results.} \label{sec:Correctness}
{\cal AUTO}-computed solutions to ODEs are almost always structurally correct,
because the mesh adaption strategy, if {\tt IAD}$>$0, safeguards to some extent
against spurious solutions.
If these do occur, possibly near infinite-period orbits,
the unusual appearance of the solution family typically serves as a warning.
Repeating the computation with increased {\tt NTST} is then recommended.

\section{ Bifurcation Points and Folds.} \label{sec:Bifurcations}
It is recommended that the detection of folds 
and bifurcation points be initially disabled.
For example, if an equation has a ``vertical'' solution family
then {\cal AUTO} may try to locate one fold after another.

Generally, degenerate bifurcations cannot be detected.
Furthermore, bifurcations that are close to each other may not
be noticed when the pseudo-arclength step size is not sufficiently small.
Hopf bifurcation points may go unnoticed if no clear crossing of
the imaginary axis takes place. This may happen when there are other
real or complex eigenvalues near the imaginary axis and when 
the pseudo-arclength step is large compared to the rate
of change of the critical eigenvalue pair.
A typical  case is a Hopf bifurcation close to a fold.
Similarly, Hopf bifurcations may go undetected if switching from
real to complex conjugate, followed by crossing of the imaginary
axis, occurs rapidly with respect to the pseudo-arclength step size.
Secondary periodic bifurcations may not be detected for similar reasons.
In case of doubt, carefully inspect the contents of the diagnostics file
{\tt fort.9}.
 
\section{ Floquet Multipliers.} \label{sec:Floquet_multipliers}

{\cal AUTO} extracts an approximation to the linearized Poincar\'e map from 
the Jacobian of the linearized collocation system that arises in Newton's method.
This procedure is very efficient; the map is computed at negligible extra cost.
The linear equations solver of {\cal AUTO} is described in 
\citename{DoKeKe:91b} \citeyear{DoKeKe:91b}.
The actual Floquet multiplier solver was written by
\citename{Fa:94} \citeyear{Fa:94}.
For a detailed description of the algorithm see 
\citename{FaJe:91} \citeyear{FaJe:91}.

For periodic solutions, the exact linearized Poincar\'e map always has 
a multiplier $z=1$.
A good accuracy check is to inspect this 
multiplier in the diagnostics output-file {\tt fort.9}.
If this multiplier becomes inaccurate then the automatic detection 
of potential secondary periodic bifurcations (if {\tt ISP}=2) is discontinued 
and a warning is printed in {\tt fort.9}.
It is strongly recommended that the contents of this file be habitually inspected,
in particular to verify whether solutions labeled as BP or TR 
(cf.~Table~\ref{tbl:Solution_Types}) have indeed  been correctly classified.
 
\section{ Memory Requirements.} \label{sec:Memory_requirements}
Pre-defined maximum values of certain {\cal AUTO}-constants
are in {\tt auto/07/include/auto.h}; 
see also Section~\ref{sec:Restrictions}. 
These maxima affect the run-time memory requirements
and should not be set to unnecessarily large values.
If an application only solves algebraic systems and if {\tt NDIM} is ``large''
then memory requirements can be much reduced by setting each of
{\tt NTSTX}, {\tt NCOLX}, {\tt NBCX}, {\tt NINTX}, 
equal to $1$ in {\tt auto/07/include/auto.h},
followed by recompilation of the {\cal AUTO} libraries.



%==============================================================================
%==============================================================================
\chapter{ {\cal AUTO} Demos : Tutorial.} \label{ch:Demos:_Tutorial}
%==============================================================================
%==============================================================================
\newpage
\section{ Introduction.} \label{sec:Tutorial_Introduction}
The directory {\tt auto/07/demos} has a large number of subdirectories,
for example {\tt ab}, {\tt pp2}, {\tt exp}, etc.,
each containing all necessary files for certain illustrative calculations.
Each subdirectory, say {\tt xxx}, corresponds to a particular equation
and contains one equations-file {\tt xxx.f}
and one or more constants-files {\tt c.xxx.i}, 
one for each successive run of the demo.
To see how the equations have been programmed, inspect the equations-file. 
To understand in detail how {\cal AUTO} is instructed to carry out a 
particular task, inspect the appropriate constants-file.
In this chapter we describe the tutorial demo {\tt ab} in detail.
A brief description of other demos is given in later chapters.


\section{ ab : A Tutorial Demo.} \label{sec:Demos_ab}
%==============================================================================
%DEMO=ab=======================================================================
%==============================================================================
This demo illustrates the computation of 
stationary solutions,
Hopf bifurcations 
and 
periodic solutions,
and the computation loci of folds and Hopf bifurcation points.
The equations, that model an A $\to$ B  reaction, are those from
\citename{URP:74} \citeyear{URP:74}, namely
\begin{equation} \begin{array}{cl}
  u_1 ' &=  -u_1 + p_1 (1-u_1) e^{u_2}, \\
  u_2 ' &=  -u_2 +  p_1 p_2 ( 1-u_1) e^{u_2} - p_3 u_2.\\
\end{array} \end{equation}

\section{ Copying the Demo Files.}  \label{sec:Tutorial_copying}
The commands listed in Table~\ref{tbl:demo_ab_1}
will copy the demo files to your work directory.

\begin{table}[htbp]
\begin{center}
\begin{tabular}{| l | l |}
\hline
  {\cal Unix}-COMMAND  & ACTION \\
\hline
%==============================================================================
  {\it cd}  & go to your main directory (or other directory)\\ 
  {\it mkdir ab}  & create an empty work directory\\ 
  {\it cd ab}  & change to the work directory\\
\hline
  {\cal AUTO}-COMMAND  & ACTION \\
\hline
  {\it @dm ab}  & copy the demo files to the work directory\\
\hline
%==============================================================================
\end{tabular}
\caption{Copying the demo {\tt ab} files.}
\label{tbl:demo_ab_1}
\end{center}
\end{table}

At this point you may want to see what files have been copied
to the work directory. 
In particular, you may want to edit the equations-file {\tt ab.f}
to see how the equations have been entered (in routine {\tt FUNC})
and how the starting solution has been set (in routine {\tt STPNT}).
Note that, initially, $p_1=0$ $p_2=14$, and $p_3=2$, for which
$u_1=u_2=0$ is a stationary solution.
 
\section{ Executing all Runs Automatically.} \label{sec:Tutorial_all_runs}
To execute all prepared runs of demo {\tt ab},
simply type the command given in Table~\ref{tbl:demo_ab_2}.

\begin{table}[htbp]
\begin{center}
\begin{tabular}{| l | l |}
\hline
  {\cal Unix}-COMMAND  & ACTION \\
\hline
%==============================================================================
  {\it make}  & execute all runs of demo {\tt ab} \\ 
\hline
%==============================================================================
\end{tabular}
\caption{Executing all runs of demo {\tt ab}.}
\label{tbl:demo_ab_2}
\end{center}
\end{table}

The resulting screen output is given below
in somewhat abbreviated form.
Some differences in output are to be expected on different machines.
This does not mean that the results have different accuracy, but simply
that arithmetic differences have accumulated from step to step, possibly
leading to different step size decisions.

Note that there are five separate runs.
In the first run, a family of stationary solutions is traced out.
Along it, two folds (LP) and one Hopf bifurcation (HB) are located.
The free parameter is $p_1$. The other parameters remain fixed in this run.
Note also that only special, labeled solution points are printed on the screen.
More detailed results are saved 
in the data-files {\tt b.ab}, {\tt s.ab}, and {\tt d.ab}.

The second run traces out the family of periodic solutions that emanates
from the Hopf bifurcation. The free parameters are $p_1$ and the period.
The detailed results are appended to the existing data-files 
{\tt b.ab}, {\tt s.ab},
and {\tt d.ab}.

In the third run, one of the folds detected in the first run is followed in
the two parameters $p_1$ and $p_3$, while $p_2$ remains fixed.
The fourth run continues this family in opposite direction.
Similarly, in the fifth run, the Hopf bifurcation located in the first run 
is followed in the two parameters $p_1$ and $p_3$.
(In this example this is done in one direction only.)
The detailed results of these continuations are accumulated
in the data-files {\tt b.2p}, {\tt s.2p}, and {\tt d.2p}.

One could now use {\cal PLAUT} to graphically inspect the contents of the
data-files, but we shall do this later.
However, it may be useful to edit these files to view their contents.

Next, reset the work directory, by typing the command given
in Table~\ref{tbl:demo_ab_3}.

\begin{table}[htbp]
\begin{center}
\begin{tabular}{| l | l |}
\hline
  {\cal Unix}-COMMAND  & ACTION \\
\hline
%==============================================================================
  {\it make clean}  & remove data-files and temporary files of demo {\tt ab} \\ 
\hline
%==============================================================================
\end{tabular}
\caption{Cleaning the demo {\tt ab} work directory.}
\label{tbl:demo_ab_3}
\end{center}
\end{table}

\newpage
\begin{center}
\begin{verbatim}
ab : first run : stationary solutions
 
BR    PT  TY LAB   PAR(1)       L2-NORM        U(1)         U(2)     
 1     1  EP   1  0.00000E+00  0.00000E+00  0.00000E+00  0.00000E+00
 1    33  LP   2  1.05739E-01  1.48439E+00  3.11023E-01  1.45144E+00
 1    70  LP   3  8.89318E-02  3.28824E+00  6.88982E-01  3.21525E+00
 1    90  HB   4  1.30899E-01  4.27186E+00  8.95080E-01  4.17704E+00
 1    92  EP   5  1.51241E-01  4.36974E+00  9.15589E-01  4.27275E+00
 Saved as *.ab
 
ab : second run : periodic solutions
 
BR    PT  TY LAB   PAR(1)       L2-NORM      MAX U(1)     MAX U(2)     PERIOD    
 4    30       6  1.19881E-01  3.98712E+00  9.91911E-01  7.02034E+00  2.721E+00
 4    60       7  1.15303E-01  3.14630E+00  9.99577E-01  9.95764E+00  6.147E+00
 4    90       8  1.05650E-01  2.21917E+00  9.99166E-01  9.36609E+00  1.399E+01
 4   120       9  1.05507E-01  1.69684E+00  9.99086E-01  9.29629E+00  9.956E+01
 4   150  EP  10  1.05507E-01  1.60388E+00  9.99789E-01  9.28146E+00  1.867E+03
 Appended to *.ab
 
ab : third run : a 2-parameter locus of folds
 
BR    PT  TY LAB   PAR(1)       L2-NORM        U(1)         U(2)       PAR(3)     
 2    27  LP  11  1.35335E-01  2.06012E+00  4.99653E-01  1.99861E+00  2.499E+00
 2   100  EP  12  1.09381E-08  2.13650E+01  9.53147E-01  2.13437E+01 -3.748E-01
 Saved as *.2p
 
ab : fourth run : the locus of folds in reverse direction
 
BR    PT  TY LAB   PAR(1)       L2-NORM        U(1)         U(2)       PAR(3)     
 2    35  EP  11 -1.31939E-03  9.96432E-01 -3.58651E-03  9.96426E-01 -1.050E+00
 Appended to *.2p
 
ab : fifth run : a 2-parameter locus of Hopf points
 
BR    PT  TY LAB   PAR(1)       L2-NORM        U(1)         U(2)       PAR(3)     
 4   100  EP  11  8.80940E-05  1.17440E+01  9.14609E-01  1.17083E+01  9.362E-02
 Appended to *.2p
\end{verbatim}
\end{center}

\newpage

\section{ Executing Selected Runs Automatically.} \label{sec:Tutorial_selected_runs}
As illustrated by the commands in Table~\ref{tbl:demo_ab_4}, 
one can also execute selected runs of demo {\tt ab}.
In general, this cannot be done in arbitrary order, as any given
run may need restart data from a previous run.
Run~3 only requires the results of Run~1, so that the displayed 
command sequence is indeed appropriate.
The screen output of these runs will be identical to that of
the corresponding earlier runs, except for a change in solution labels in Run~3.

\begin{table}[htbp]
\begin{center}
\begin{tabular}{| l | l |}
\hline
  {\cal Unix}-COMMAND  & ACTION \\
\hline
%==============================================================================
  {\it make first}  & execute the first run of demo {\tt ab} \\ 
  {\it make third}  & execute the third run of demo {\tt ab} \\ 
\hline
%==============================================================================
\end{tabular}
\caption{Selected runs of demo {\tt ab}.}
\label{tbl:demo_ab_4}
\end{center}
\end{table}

Of course, in real use, one must prepare a constants-file for each run.
In the illustrative runs above, the constants-files 
were carefully prepared in advance.
For example, the file {\tt c.ab.1} contains the {\cal AUTO}-constants for Run~1,
{\tt c.ab.3} contains the {\cal AUTO}-constants for Run~3, etc.

\section{ Using {\cal AUTO}-Commands.} \label{sec:Tutorial_AUTO_commands}
Next, with the commands in Table~\ref{tbl:demo_ab_5},
we execute the first two runs of demo {\tt ab} again, but now using the
commands that one would normally use in an actual application.
We still use the demo constants-files that were prepared in advance.

\begin{table}[htbp]
\begin{center}
\begin{tabular}{| l | l |}
\hline
  COMMAND  & ACTION \\
\hline
%==============================================================================
  {\it make clean}  & reset the work directory \\ 
\hline
%==============================================================================
  {\it cp c.ab.1 c.ab} & get the first constants-file \\ 
  {\it @r ab} & compute a stationary solution family with folds and Hopf bifurcation \\ 
  {\it @sv ab} & save output-files as {\tt b.ab, s.ab, d.ab} \\ 
\hline
%==============================================================================
  {\it cp c.ab.2 c.ab} &  get the second constants-file \\ 
  {\it @r ab} & compute a family of periodic solutions from the Hopf point \\ 
  {\it @ap ab} & append the output-files to {\tt b.ab, s.ab, d.ab} \\ 
\hline
%==============================================================================
\end{tabular}
\caption{Commands for Run~1 and Run~2 of demo {\tt ab}.}
\label{tbl:demo_ab_5}
\end{center}
\end{table}
 
It is instructive to look at the constants-files
{\tt c.ab.1} and {\tt c.ab.2} used in the two runs above.
The significance of each {\cal AUTO}-constant set in these files
can be found in Chapter~\ref{ch:AUTO_constants}.
Note in particular the {\cal AUTO}-constants that were changed 
between the two runs; see Table~\ref{tbl:demo_ab_6}.
\begin{table}[htbp]
\begin{center}
\begin{tabular}{| l | r | r | l |}
\hline
  Constant &  Run~1  &  Run~2 & Reason for Change \\
\hline
%==============================================================================
  {\tt IPS}  & 1  & 2  &  To compute periodic solutions in Run~2 \\  
\hline
  {\tt IRS}  & 0  & 4  &  To specify the Hopf bifurcation restart label \\  
\hline
  {\tt NICP}  & 1  & 2  &  The second run has two free parameters\\  
\hline
  {\tt ICP}  & 1  &1,~11  &  To use and print {\tt PAR(1)} and {\tt PAR(11)} in Run~2\\  
\hline
  {\tt NMX}  & 100 &150  &  To allow more continuation steps in Run~2 \\  
\hline
  {\tt NPR}  & 100 & 30  &  To print output every 30 steps in Run~2 \\  
\hline
%==============================================================================
\end{tabular}
\caption{Differences in {\cal AUTO}-constants between {\tt c.ab.1} and {\tt c.ab.2}.}
\label{tbl:demo_ab_6}
\end{center}
\end{table}

Actually, for periodic solutions, {\cal AUTO} automatically adds {\tt PAR(11)}
(the period) as second parameter.
However, for the period to be printed, one must specify the index 11
in the {\tt ICP} list, as shown in Table~\ref{tbl:demo_ab_6}.

\section{ Plotting the Results with {\cal PLAUT}.} \label{sec:Tutorial_PLAUT}
The bifurcation diagram computed in the runs above
is stored in the file {\tt b.ab},
while each labeled solution is fully stored in {\tt s.ab}.
To use {\cal PLAUT} to graphically inspect these data-files,
type the {\cal AUTO}-command given in Table~\ref{tbl:demo_ab_7}.
The {\cal PLAUT} window (a Tektronix window) will appear, in which one can enter
the {\cal PLAUT}-commands given in Table~\ref{tbl:demo_ab_8}.
The saved plots are shown in Figure~\ref{fig:ab_1}
and  in Figure~\ref{fig:ab_2}.

\begin{table}[htbp]
\begin{center}
\begin{tabular}{| l | l |}
\hline
  {\cal AUTO}-COMMAND  & ACTION \\
\hline
%==============================================================================
  {\it @p ab} & run {\cal PLAUT} to graph the contents of {\tt b.ab} and {\tt s.ab}; \\  
%==============================================================================
\hline
\end{tabular}
\caption{Command for plotting the files {\tt b.ab} and {\tt s.ab}.}
\label{tbl:demo_ab_7}
\end{center}
\end{table}

\begin{table}[htbp]
\begin{center}
\begin{tabular}{| l | l |}
\hline
  {\cal PLAUT}-COMMAND  & ACTION \\
\hline
  {\it d1}  & choose one of the default settings\\ 
  {\it bd0}  & plot the default bifurcation diagram; $L_2$-norm versus $p_1$ \\ 
\hline
  {\it bd}  & make a blow-up of current bifurcation diagram \\ 
  {\it .08~ .14 ~.5~ 4.5} & enter diagram limits  \\
\hline
  {\it sav}  & save the current plot \\
  {\it fig.1}  & upon prompt, enter a new file name, e.g., {\tt fig.1} \\
\hline
  {\it 2d}  & enter 2D mode, for plotting labeled solutions\\ 
  {\it 6 7 10}  & select labeled orbits 6, 7, and 10 in {\tt s.ab}\\ 
  {\it d}  & default orbit display; $u_1$ versus scaled time\\
\hline
  {\it 1 3}  & select columns 1 and 3 in {\tt s.ab} \\
  {\it d}  & display the orbits; $u_2$ versus scaled time\\
\hline
  {\it 2 3}  & select columns 2 and 3 in {\tt s.ab} \\
  {\it d}  & phase plane display; $u_2$ versus $u_1$\\
\hline
  {\it sav}  & save the current plot \\
  {\it fig.2}  & upon prompt, enter a new file name \\
\hline
  {\it ex}  & exit from 2D mode  \\
  {\it end}  & exit from {\cal PLAUT} \\
\hline
\end{tabular}
\caption{Commands to be typed in the {\cal PLAUT} window.}
\label{tbl:demo_ab_8}
\end{center}
\end{table}

%------------------------------------------------------
\begin{figure}[p]
\epsfysize 9.0cm
\centerline{\epsffile{include/ab1.ps}}
\caption{The bifurcation diagram of demo {\tt ab}.}
\label{fig:ab_1}
\end{figure}
%------------------------------------------------------

%------------------------------------------------------
\begin{figure}[p]
\epsfysize 9.0cm
\centerline{\epsffile{include/ab2.ps}}
\caption{The phase plot of solutions 6, 7, and 10 in demo {\tt ab}.}
\label{fig:ab_2}
\end{figure}
%------------------------------------------------------


\section{ Following Folds and Hopf Bifurcations.} \label{sec:Tutorial_2_par}
The commands in Table~\ref{tbl:demo_ab_9} will execute the remaining
runs of demo {\tt ab}.
Here, as in later demos, some of the {\cal AUTO}-constants that have been changed
between runs are indicated in the Table.
%==============================================================================
%==============================================================================
\begin{table}[htbp]
\begin{center}
\begin{tabular}{| l | l |}
\hline
   COMMAND  & ACTION \\
\hline
  {\it cp c.ab.3 c.ab} & changes (from {\tt c.ab.1}) : IRS, NICP, ICP, ISW, DSMAX \\ 
  {\it @r ab} &  compute a locus of folds \\ 
  {\it @sv 2p} & save output-files as {\tt b.2p, s.2p, d.2p} \\ 
\hline
%==============================================================================
  {\it cp c.ab.4 c.ab} & changes (from {\tt c.ab.3}) : DS (sign) \\ 
  {\it @r ab} &  compute the  locus of folds in reverse direction \\ 
  {\it @ap 2p} &  append the output-files to {\tt b.2p, s.2p, d.2p} \\ 
\hline
%==============================================================================
  {\it cp c.ab.5 c.ab} & changes (from {\tt c.ab.4}) : IRS \\ 
  {\it @r ab} &  compute a locus of Hopf points \\ 
  {\it @ap 2p} & append the output-files to {\tt b.2p, s.2p, d.2p} \\ 
\hline
%==============================================================================
\end{tabular}
\caption{Commands for Runs~3, 4, and 5 of demo {\tt ab}.}
\label{tbl:demo_ab_9}
\end{center}
\end{table}

\section{ Relabeling Solutions in the Data-Files.} \label{sec:Tutorial_relabeling}
Next we want to plot the two-parameter diagram computed in the last three runs.
However, the solution labels in these runs are not distinct.
This is due to the fact that in each of these three runs
the restart solution was read from {\tt s.ab}, while the
computed solutions were stored in {\tt s.2p}.
Consequently, these runs were unaware of each other's results, 
which led to non-unique labels.
For relabeling purpose, and more generally for file maintenance,
there is a utility program that can be invoked as indicated in 
Table~\ref{tbl:demo_ab_10}.
Its use is illustrated in Table~\ref{tbl:demo_ab_11}.
\begin{table}[htbp]
\begin{center}
\begin{tabular}{| l | l |}
\hline
  {\cal AUTO}-COMMAND  & ACTION \\
\hline
  {\it @lb 2p} & run the relabeling  program on {\tt b.2p} and {\tt s.2p} \\ 
\hline
%==============================================================================
\end{tabular}
\caption{Command to run the relabeling program on {\tt b.2p} and {\tt s.2p}.}
\label{tbl:demo_ab_10}
\end{center}
\end{table}


\begin{table}[htbp]
\begin{center}
\begin{tabular}{| c | l |}
\hline
  RELABELING COMMAND  & ACTION \\
\hline
  l & list the labeled solutions in {\tt s.2p} \\
  r & relabel the solutions  \\  
  l & list the new solution labeling  \\
  w & rewrite {\tt b.2p} and {\tt s.2p}  \\
\hline
%==============================================================================
\end{tabular}
\caption{Relabeling commands for the files {\tt b.2p} and {\tt s.2p}.}
\label{tbl:demo_ab_11}
\end{center}
\end{table}
\section{ Plotting the 2-Parameter Diagram.} \label{sec:Tutorial_plotting_2p}
To run {\cal PLAUT} on the files  {\tt b.2p} and {\tt s.2p},
enter the command listed in Table~\ref{tbl:demo_ab_12}.
The {\cal PLAUT}-commands for plotting the two-parameter diagram are then as given
in Table~\ref{tbl:demo_ab_13}.
The saved plot is shown in Figure~\ref{fig:ab_3}.

\begin{table}[htbp]
\begin{center}
\begin{tabular}{| l | l |}
\hline
  {\cal AUTO}-COMMAND  & ACTION \\
\hline
  {\it @p 2p} & run {\cal PLAUT} to graph the contents of {\tt b.2p} and {\tt s.2p}; \\ 
\hline
\end{tabular}
\caption{Command to run {\cal PLAUT} for files {\tt b.2p} and {\tt s.2p}.}
\label{tbl:demo_ab_12}
\end{center}
\end{table}

\begin{table}[htbp]
\begin{center}
\begin{tabular}{| l | l |}
\hline
  {\cal PLAUT}-COMMAND  & ACTION \\
\hline
  {\it d0}  & set default option\\ 
  {\it ax}  & select axes \\ 
  {\it 1 5}  & select real columns 1 ($p_1$) and 5 ($p_3$) in {\tt b.2p} \\ 
  {\it bd0}  & plot the 2-parameter diagram; $p_3$ versus $p_1$ \\
  {\it cl}  & clear the screen  \\
\hline
  {\it d2}  & set other default option\\ 
  {\it bd0}  & plot the 2-parameter diagram; $p_3$ versus $p_1$ \\
\hline
  {\it bd}  & make a blow-up of the current diagram \\ 
  {\it 0 ~.15~ 0~ 2.5} & enter diagram limits  \\
  {\it sav}  & save plot \\
  {\it fig.3}  & upon prompt, enter a new file name, e.g., {\tt fig.3} \\
\hline
  {\it end}  & exit from {\cal PLAUT} \\
\hline
%==============================================================================
\end{tabular}
\caption{ {\cal PLAUT}-commands for files {\tt b.2p} and {\tt s.2p}.}
\label{tbl:demo_ab_13}
\end{center}
\end{table}

\begin{figure}[t]
\epsfysize 9.0cm
\centerline{\epsffile{include/ab3.ps}}
\caption{Loci of folds and Hopf bifurcations for demo {\tt ab}.}
\label{fig:ab_3}
\end{figure}



\section{ Converting Saved {\cal PLAUT} Files to PostScript.} \label{sec:Tutorial_plot_conversion}
Plots are saved in compact Tektronix {\cal PLOT10} format.
In Table~\ref{tbl:demo_ab_14} it is shown how such files can be converted to 
PostScript format. Note that the latter files are much bigger.
\begin{table}[htbp]
\begin{center}
\begin{tabular}{| l | l |}
\hline
  {\cal AUTO}-COMMAND  & ACTION \\
\hline
%==============================================================================
  {\it @ps fig.1} & convert file {\tt fig.1} into {\cal PostScript} file {\tt fig.1.ps} \\ 
  {\it lpr fig.1.ps} & system dependent : print {\tt fig.1.ps} on your printer \\ 
\hline
%==============================================================================
  {\it @ps fig.2} & convert file {\tt fig.2} into {\cal PostScript} file {\tt fig.2.ps} \\ 
  {\it lpr fig.2.ps} & system dependent : print {\tt fig.2.ps} on your printer \\ 
\hline
%==============================================================================
  {\it @ps fig.3} & convert file {\tt fig.3} into {\cal PostScript} file {\tt fig.3.ps} \\ 
  {\it lpr fig.3.ps} & system dependent : print {\tt fig.3.ps} on your printer \\ 
\hline
%==============================================================================
\end{tabular}
\caption{Printing commands for the saved Figures in demo {\tt ab}.}
\label{tbl:demo_ab_14}
\end{center}
\end{table}

\newpage
\section{ Using the GUI.} \label{sec:Tutorial_GUI}
Demos can also be run using the GUI. See Table~\ref{tbl:CM_GUI}
for the correspondence between Command Mode and GUI actions.
To activate the GUI, type the command in Table~\ref{tbl:demo_ab_15}.
The GUI actions to execute the first two runs of demo {\tt ab}
are given in Table~\ref{tbl:demo_ab_16}.
In GUI Mode one can copy demo files to the user work directory
using the {\it Equations/Demo}-button.
To load a selected constants-file, use the {\it Previous}-button
in the {\it LoadConsts} area of the GUI window.
Press the {\it Filter}-button in the pop-up window
to update the displayed list of files, and then select
the appropriate constants-file.


\begin{table}[htbp]
\begin{center}
\begin{tabular}{| l | l |}
\hline
  {\cal AUTO}-COMMAND  & ACTION \\
\hline
{\it @auto} & Activate the Graphical User Interface \\
\hline
\end{tabular}
\caption{Command to activate the GUI.}
\label{tbl:demo_ab_15}
\end{center}
\end{table}

\begin{table}[htbp]
\begin{center}
\begin{tabular}{| l | l |}
\hline
  GUI-button  & ACTION \\
\hline
{\it Equations/Demo} & Select demo {\tt ab}, then press {\it OK} \\
\hline
{\it Previous}       & Push {\it Filter}, select file {\tt c.ab.1}, then press {\it OK} \\
\hline
{\it Run}            & This will execute Run~1 of demo {\tt ab} \\ 
\hline
{\it Save/Save}      & Save the output files as {\tt b.ab}, {\tt s.ab}, {\tt d.ab} \\ 
\hline
{\it Previous}       & Select file {\tt c.ab.2}, then press {\it OK} \\
\hline
{\it Run}            & This will execute Run~2  of demo {\tt ab}\\ 
\hline
{\it Append/Append}  & Append the output-files to {\tt b.ab}, {\tt s.ab}, {\tt d.ab} \\ 
\hline
%==============================================================================
\end{tabular}
\caption{GUI actions for Run~1 and Run~2 of demo {\tt ab}.}
\label{tbl:demo_ab_16}
\end{center}
\end{table}



To execute all runs of a selected demo with the GUI,
click {\it Demos/Select}, select a demo,
and click the {\it Run}-button in the pop-up window.
This will actually run the demo in the corresponding subdirectory
of {\tt auto/07/demos}, which is only possible if you have write access
to this directory. 
Make sure to click the {\it Demos/Reset}-button afterwards.
Do not otherwise run {\cal AUTO} in the directory {\tt auto/07} 
or in any of its subdirectories.


\section{ Abbreviated {\cal AUTO}-Commands.} \label{sec:Abbreviated_AUTO_commands}
The {\cal AUTO}-commands given in, for example, Table~\ref{tbl:demo_ab_5} 
can be simplified by using the {\it @R} command.
For Table~\ref{tbl:demo_ab_5} the equivalent command sequence is given in 
Table~\ref{tbl:demo_ab_17}.

\begin{table}[htbp]
\begin{center}
\begin{tabular}{| l | l |}
\hline
   COMMAND  & ACTION \\
\hline
%==============================================================================
  {\it make clean}  & reset the work directory \\ 
\hline
%==============================================================================
  {\it @R ab 1} &  (reads {\cal AUTO}-constants from {\tt c.ab.1}) \\ 
  {\it @sv ab} & save output-files as {\tt b.ab, s.ab, d.ab} \\ 
\hline
%==============================================================================
  {\it @R ab 2} &  (reads {\cal AUTO}-constants from {\tt c.ab.2})\\ 
  {\it @ap ab} & append the output-files to {\tt b.ab, s.ab, d.ab} \\ 
\hline
%==============================================================================
\end{tabular}
\caption{Abbreviated {\cal AUTO}-commands.}
\label{tbl:demo_ab_17}
\end{center}
\end{table}

%==============================================================================
%==============================================================================
\chapter{ {\cal AUTO} Demos : Fixed points.} \label{ch:Demos_Fixed_points}
%==============================================================================
%==============================================================================

%==============================================================================
%DEMO=enz======================================================================
%==============================================================================
\section{ enz : Stationary Solutions of an Enzyme Model.} \label{sec:Demos_enz}
The equations, that model a two-compartment enzyme system 
(\citename{JPK:80} \citeyear{JPK:80}),
are given by
\begin{equation} \label{2'} \begin{array}{cl}
 s_1 '&=
 (s_0 - s_1) + (s_2 - s_1) - \rho R (s_1), \\
 s_2 '&=
 (s_0 +\mu - s_2) + (s_1 - s_2) - \rho R (s_2), \\\end{array} \end{equation}
where
$$ R (s)={s \over 1+s+ \kappa s^{2} }.$$
The free parameter is $s_0$. Other parameters are fixed.
This equation is also considered in 
\citename{DoKeKe:91a} \citeyear{DoKeKe:91a}.

\begin{table}[htbp]
\begin{center}
\begin{tabular}{| l | l |}
\hline
  COMMAND  & ACTION \\
\hline
%==============================================================================
  {\it mkdir enz} & create an empty work directory \\ 
  {\it cd enz} & change directory \\
  {\it @dm enz} & copy the demo files to the work directory \\
\hline
%==============================================================================
  {\it cp c.enz.1 c.enz} & get the constants-file \\ 
  {\it @r enz} & compute stationary solution families \\ 
  {\it @sv enz} & save output-files as {\tt b.enz, s.enz, d.enz} \\ 
\hline
\end{tabular}
\caption{Commands for running demo {\tt enz}.}
\label{tbl:demo_enz}
\end{center}
\end{table}

\newpage
%==============================================================================
%DEMO=dd2======================================================================
%==============================================================================
\section{ dd2 : Fixed Points of a Discrete Dynamical System.} \label{sec:Demos_dd2}
This demo illustrates the computation of a solution family and
its bifurcating families for a discrete dynamical system.
Also illustrated is the continuation of 
Naimark-Sacker (or Hopf) bifurcations
The equations, a discrete predator-prey system, are
\begin{equation} \begin{array}{cl}
 u_1^{k+1} &=p_1
 u_1^{k}(1-u_1^{k})-p_2u_1^{k} u_2^{k},\\
 u_2^{k+1}&=(1-p_3)u_2^{k}+p_2u_1^{k}u_2^{k}.\\
\end{array} \end{equation}
In the first run $p_1$ is free.
In the second run, both $p_1$ and $p_2$ are free.
The remaining equation parameter, $p_3$, is fixed in both runs.

\begin{table}[htbp]
\begin{center}
\begin{tabular}{| l | l |}
\hline
  {\cal AUTO}-COMMAND  & ACTION \\
\hline
%==============================================================================
  {\it mkdir dd2} & create an empty work directory \\ 
  {\it cd dd2} & change directory \\
  {\it @dm dd2} & copy the demo files to the work directory \\
\hline
%==============================================================================
 
  {\it cp c.dd2.1 c.dd2} & get the first constants-file \\ 
  {\it @r dd2} & 1st run; fixed point solution families \\ 
  {\it @sv dd2} & save output-files as {\tt b.dd2, s.dd2, d.dd2} \\ 
\hline
%==============================================================================
  {\it cp c.dd2.2 c.dd2} & constants changed : {\tt IRS, ISW} \\ 
  {\it @r dd2} & 2nd run; a locus of Naimark-Sacker bifurcations \\ 
  {\it @sv ns} & save output-files as {\tt b.ns, s.ns, d.ns} \\ 
\hline
\end{tabular}
\caption{Commands for running demo {\tt dd2}.}
\label{tbl:demo_dd2}
\end{center}
\end{table}


%==============================================================================
%==============================================================================
\chapter{ {\cal AUTO} Demos : Periodic solutions.} \label{ch:Demos_Periodic}
%==============================================================================
%==============================================================================

%==============================================================================
%DEMO=lrz======================================================================
%==============================================================================
\newpage
\section{ lrz : The Lorenz Equations.} \label{sec:Demos_lrz}
This demo computes two symmetric homoclinic orbits in the Lorenz equations
\begin{equation} \begin{array}{cl}
  u_1' &=  p_3 (u_2 - u_1), \\
  u_2' &=  p_1 u_1 - u_2 - u_1 u_3,  \\
  u_3' &=  u_1 u_2 - p_2 u_3. \\ \end{array} \end{equation}
Here $p_1$ is the free parameter, and $p_2=8/3$, $p_3=10$.
The two homoclinic orbits correspond to the final, large period orbits 
on the two periodic solution families.

\begin{table}[htbp]
\begin{center}
\begin{tabular}{| l | l |}
\hline
  COMMAND  & ACTION \\
\hline
%==============================================================================

  {\it mkdir lrz} & create an empty work directory \\ 
  {\it cd lrz} & change directory \\
  {\it @dm lrz} & copy the demo files to the work directory \\
\hline
%==============================================================================
  {\it cp c.lrz.1 c.lrz} & get the first constants-file \\ 
  {\it @r lrz} & compute stationary solutions \\ 
  {\it @sv lrz} & save output-files as {\tt b.lrz, s.lrz, d.lrz} \\ 
\hline
%==============================================================================
  {\it cp c.lrz.2 c.lrz} & constants changed : {\tt IPS, IRS, NICP, ICP, NMX, NPR, DS} \\ 
  {\it @r lrz} & compute periodic solutions; the final orbit is near-homoclinic \\ 
  {\it @ap lrz} & append the output-files to {\tt b.lrz, s.lrz, d.lrz} \\ 
\hline
%==============================================================================
  {\it cp c.lrz.3 c.lrz} & constants changed : {\tt IRS} \\ 
  {\it @r lrz} & compute the symmetric periodic solution family \\ 
  {\it @ap lrz} & append the output-files to {\tt b.lrz, s.lrz, d.lrz} \\ 
\hline
\end{tabular}
\caption{Commands for running demo {\tt lrz}.}
\label{tbl:demo_lrz}
\end{center}
\end{table}

\newpage
%==============================================================================
%DEMO=abc======================================================================
%==============================================================================
\section{ abc : The A $\to$ B $\to$ C Reaction.} \label{sec:Demos_abc}
This demo illustrates the computation of 
stationary solutions,
Hopf bifurcations 
and
periodic solutions
in the A $\to$ B $\to$ C reaction 
(\citename{DoHe:83} \citeyear{DoHe:83}).
\begin{equation} \begin{array}{cl}
  u_1 ' &=  -u_1 + p_1 (1-u_1) e^{u_3}, \\
  u_2 ' &=  -u_2 +  p_1 e^{u_3} ( 1-u_1 - p_5 u_2 ),\\
  u_3 ' &=  -u_3 - p_3 u_3 + p_1 p_4 e^{u_3}  
  ( 1-u_1 + p_2 p_5 u_2 ),\\ \end{array} \end{equation}
with $p_2=1$, $p_3=1.55$, $p_4=8$, and $p_5=0.04$. 
The free parameter is $p_1$.

The equations, as programmed in the equations-file {\tt abc.f},
appear in Table~\ref{tbl:demo_abcE1}.
The starting point, an equilibrium of the equations,
is also defined in  the equations-file {\tt abc.f},
as shown in  Table~\ref{tbl:demo_abcE2}.
(The equations-file {\tt abc.f} also contains the skeletons
of some other routines, which must be supplied, but which 
are not used in this application.)

In the constants-file ({\tt c.abc.1}) for the first run, as shown in 
Table~\ref{tbl:demo_abcC1}, we note the following:

\begin{itemize}
\item[-] {\tt IPS=1}~: a family of stationary solutions is computed.

\item[-] {\tt IRS=0}~: the starting point defined in {\tt STPNT} 
	 is to be used (see  Table~\ref{tbl:demo_abcE2}). 

\item[-] {\tt ICP(1)=1}~: the continuation parameter is PAR(1) 

\item[-] {\tt NUZR=1}~: there is one user output point, namely at
	 {\tt PAR(1)=0.4}. Moreover, since the index ("{\tt -1}") in
	 the last line of the constants-file {\tt c.abc.1} is negative, 
	 the calculation will terminate when the calculation reaches
	 the value {\tt PAR(1)=0.4}..
\end{itemize}

In the constants-file ({\tt c.abc.2}) for the second run, as shown in 
Table~\ref{tbl:demo_abcC2}, we note that:

\begin{itemize}
\item[-] {\tt IPS=2}~: a family of periodic solutions is computed.

\item[-] {\tt IRS=2}~: the starting point is the solution with label 2,
         (a Hopf bifurcation point), to be read from the solutions-file 
	 (here {\tt s.abc}).

\item[-] {\tt NICP=2}~: there are two continuation parameters
	 (namely {\tt PAR(1)}, and the period, {\tt PAR(11)}). 

\item[-] {\tt NUZR=1}~: there is one user output point, now at
	 {\tt PAR(1)=0.25}, where the calculation is to terminate,
	 since the index ("{\tt -1}") is negative.
\end{itemize}

%------------------------------------------------------
\begin{table}[htbp]
{\small
\begin{center}
\begin{boxedverbatim}
      SUBROUTINE FUNC(NDIM,U,ICP,PAR,IJAC,F,DFDU,DFDP)
C     ---------- ----
C
      IMPLICIT DOUBLE PRECISION (A-H,O-Z)
      DIMENSION U(NDIM),PAR(*),F(NDIM)
C
       X1=U(1)
       X2=U(2)
       X3=U(3)
C
       D=PAR(1)
       ALPHA=PAR(2)
       BETA=PAR(3)
       B=PAR(4)
       S=PAR(5)
C
       E=DEXP(X3)
       X1C=1-X1
C
       F(1)=-X1 + D*X1C*E
       F(2)=-X2 + D*E*(X1C - S*X2)
       F(3)=-X3 - BETA*X3 + D*B*E*(X1C + ALPHA*S*X2)
C
      RETURN
      END
\end{boxedverbatim}
\end{center}
}
\caption{The equations for demo {\tt abc}, 
as defined in the equations-file {\tt abc.f}.}
\label{tbl:demo_abcE1}
\end{table}
%------------------------------------------------------


%------------------------------------------------------
\begin{table}[htbp]
{\small
\begin{center}
\begin{boxedverbatim}
      SUBROUTINE STPNT(NDIM,U,PAR,T)
C     ---------- -----
C
      IMPLICIT DOUBLE PRECISION (A-H,O-Z)
      DIMENSION U(NDIM),PAR(*)
C
       PAR(1)=0.0
       PAR(2)=1.0
       PAR(3)=1.55
       PAR(4)=8.
       PAR(5)=0.04
C
       U(1)=0.
       U(2)=0.
       U(3)=0.
C
      RETURN
      END
\end{boxedverbatim}
\end{center}
}
\caption{The starting solution for demo {\tt abc}, 
as defined in the equations-file {\tt abc.f}.}
\label{tbl:demo_abcE2}
\end{table}
%------------------------------------------------------

%------------------------------------------------------
\begin{table}[htbp]
{\small
\begin{center}
\begin{boxedverbatim}
3 1 0 1                         NDIM,IPS,IRS,ILP
1   1                           NICP,(ICP(I),I=1,NICP)
15 4 3 1 1 0 0 0                NTST,NCOL,IAD,ISP,ISW,IPLT,NBC,NINT
130 0 0.4 0 25                  NMX,RL0,RL1,A0,A1
200 10 2 8 5 3 0                NPR,MXBF,IID,ITMX,ITNW,NWTN,JAC
1e-7 1e-7 1e-4                  EPSL,EPSU,EPSS
0.02 0.001 0.1 1                DS,DSMIN,DSMAX,IADS
1                               NTHL,((I,THL(I)),I=1,NTHL)
11 0
0                               NTHU,((I,THU(I)),I=1,NTHU)
1                               NUZR,((I,UZR(I)),I=1,NUZR)
-1 0.4
\end{boxedverbatim}
\end{center}
}
\caption{The constants-file {\tt c.abc.1} for Run 1 (stationary solutions)
of demo {\tt abc}.}
\label{tbl:demo_abcC1}
\end{table}
%------------------------------------------------------


%------------------------------------------------------
\begin{table}[htbp]
{\small
\begin{center}
\begin{boxedverbatim}
3 2 2 1                         NDIM,IPS,IRS,ILP
2   1 11                        NICP,(ICP(I),I=1,NICP)
25 4 3 1 1 0 0 0                NTST,NCOL,IAD,ISP,ISW,IPLT,NBC,NINT
200 0 0.4 0 25                  NMX,RL0,RL1,A0,A1
200 10 2 8 5 3 0                NPR,MXBF,IID,ITMX,ITNW,NWTN,JAC
1e-7 1e-7 1e-4                  EPSL,EPSU,EPSS
0.02 0.001 0.1 1                DS,DSMIN,DSMAX,IADS
1                               NTHL,((I,THL(I)),I=1,NTHL)
11 0
0                               NTHU,((I,THU(I)),I=1,NTHU)
1                               NUZR,((I,UZR(I)),I=1,NUZR)
-1 0.25
\end{boxedverbatim}
\end{center}
}
\caption{The constants-file {\tt c.abc.2} for Run 2 (periodic orbits) 
of demo {\tt abc}.}
\label{tbl:demo_abcC2}
\end{table}
%------------------------------------------------------


\begin{table}[htbp]
\begin{center}
\begin{tabular}{| l | l |}
\hline
  COMMAND  & ACTION \\ 
\hline
%==============================================================================
  {\it mkdir abc} & create an empty work directory \\ 
  {\it cd abc} & change directory \\
  {\it @dm abc} & copy the demo files to the work directory \\
\hline
%==============================================================================
  {\it @R abc 1} & compute the stationary solution family 
						with four Hopf bifurcations \\ 
  {\it @sv abc} & save output-files as {\tt b.abc, s.abc, d.abc} \\ 
\hline
%==============================================================================
  {\it @R abc 2} & compute a family of periodic solutions from the first Hopf point \\ 
  {\it @ap abc} & append the output-files to {\tt b.abc, s.abc, d.abc} \\ 
\hline
%==============================================================================
  {\it @R abc 3} & compute a family of periodic solutions from the second Hopf point \\ 
  {\it @ap abc} & append the output-files to {\tt b.abc, s.abc, d.abc} \\ 
\hline
%==============================================================================
  {\it @R abc 4} & compute a family of periodic solutions from the third Hopf point \\ 
  {\it @ap abc} & append the output-files to {\tt b.abc, s.abc, d.abc} \\ 
\hline
%==============================================================================
  {\it @R abc 5} & compute a family of periodic solutions from the fourth Hopf point \\ 
  {\it @ap abc} & append the output-files to {\tt b.abc, s.abc, d.abc} \\ 
\hline
%==============================================================================
\end{tabular}
\caption{Linux Commands for running demo {\tt abc}.}
\label{tbl:demo_abcL}
\end{center}
\end{table}

%------------------------------------------------------
\begin{table}[htbp]
{\small 
\begin{center} 
\begin{boxedverbatim}
ld(e='abc',c='abc.1')
run()
sv('abc')
ld(c='abc.2',s='abc')
run()
ap('abc')
ld(c='abc.3',s='abc')
run()
ap('abc')
ld(c='abc.4',s='abc')
run()
ap('abc')
ld(c='abc.5',s='abc')
run()
ap('abc')
\end{boxedverbatim}
\end{center}
}
\caption{Python Commands for running demo {\tt abc}.}
\label{tbl:demo_abcP1}
\end{table}
%------------------------------------------------------

%------------------------------------------------------
\begin{table}[htbp]
{\small
\begin{center}
\begin{boxedverbatim}
ld(e='abc',c='abc.1')
run()
sv('abc')
ld(c='abc.2',s='abc')
data = sl('abc')
for solution in data:
    if solution["Type name"] == "HB":
        ch("IRS", solution["Label"])
        run()
        ap('abc')

\end{boxedverbatim}
\end{center}
}
\caption{Python Program for running demo {\tt abc}.}
\label{tbl:demo_abcP2}
\end{table}
%------------------------------------------------------


\newpage
%==============================================================================
%DEMO=pp2======================================================================
%==============================================================================
\section{ pp2 : A 2D Predator-Prey Model.} \label{sec:Demos_pp2}
This demo illustrates the computation of families of stationary
solutions, including bifurcating stationary families, as well as
the detection of a Hopf bifurcation.
In the second run the family of periodic solutions that emanates
from the Hopf bifurcation is computed. This family terminates in
a heteroclinic orbit.

The equations, which model a predator-prey system with harvesting, are
\begin{equation} \begin{array}{cl}
  u_1 ' &= p_2 u_1 (1 - u_1 ) - u_1 u_2 - p_1 (1-e^{-p_3 u_1}) ,\\
  u_2 ' &= -u_2  + p_4 u_1 u_2  .\end{array} \end{equation}
Here $p_1$ is the principal continuation parameter,
while $p_2=p_4=3$ and $p_3=5$, are fixed
The use of {\cal PLAUT} is also illustrated. The saved plots are shown
in Figure~\ref{fig:pp2_1} and  Figure~\ref{fig:pp2_2}.
\begin{table}[htbp]
\begin{center}
\begin{tabular}{| l | l |}
\hline
  COMMAND  & ACTION \\
\hline
%==============================================================================
  {\it mkdir pp2} & create an empty work directory \\ 
  {\it cd pp2} & change directory \\ 
  {\it @dm pp2} & copy the demo files to the work directory \\ 
\hline
%============================================================================== 
  {\it @R pp2 1} & 1st run, using constants-file {\tt c.pp2.1}; stationary solutions \\ 
  {\it @sv pp2} & save output-files as {\tt b.pp2, s.pp2, d.pp2} \\ 
\hline
%============================================================================== 
  {\it @R pp2 2} & 2nd run, using constants-file {\tt c.pp2.2}; periodic solutions \\ 
  {\it @ap pp2} & append output-files to {\tt b.pp2, s.pp2, d.pp2} \\ 
\hline
%==============================================================================
\end{tabular}
\caption{Commands for running demo {\tt pp2}.}
\label{tbl:demo_pp2_1}
\end{center}
\end{table}


\begin{table}[htbp]
\begin{center}
\begin{tabular}{| l | l |}
\hline
  {\cal AUTO}-COMMAND  & ACTION \\
\hline
  {\it @p pp2} & run {\cal PLAUT} to graph the contents of {\tt b.pp2} and {\tt s.pp2}; \\ 
\hline
  {\cal PLAUT}-COMMAND  & ACTION \\
\hline
  {\it d2}  & set convenient defaults\\ 
  {\it ax}  & select axes \\ 
  {\it 1 3}  & select real columns 1 and 3 in {\tt b.pp2} \\ 
  {\it bd0}  & plot the bifurcation diagram; $max~u_1$ versus $p_1$ \\
\hline
  {\it d1}  & choose other default settings \\ 
  {\it bd}  & get blow-up of current bifurcation diagram \\ 
  {\it 0~ 1 ~-0.25~ 1} & enter diagram limits  \\
  {\it sav}  & save plot (see Figure~\ref{fig:pp2_1})\\
  {\it fig.1}  & upon prompt, enter a new file name, e.g., {\tt fig.1} \\
  {\it cl}  & clear the screen  \\
\hline
  {\it 2d}  & enter 2D mode, for plotting labeled solutions\\ 
  {\it 11 15 19 23}  & select these labeled orbits in {\tt s.pp2}\\ 
  {\it d}  & default orbit display; $u_1$ versus time\\
\hline
  {\it 1 3}  & select columns 1 and 3 in {\tt s.pp2} \\
  {\it d}  & display the orbits; $u_2$ versus time\\
\hline
  {\it 2 3}  & select columns 2 and 3 in {\tt s.pp2} \\
  {\it d}  & phase plane display; $u_2$ versus $u_1$\\
  {\it sav}  & save plot  (see Figure~\ref{fig:pp2_2})\\
  {\it fig.2}  & upon prompt, enter a new file name \\
  {\it ex}  & exit from 2D mode  \\
\hline
  {\it end}  & exit from {\cal PLAUT} \\
\hline
%==============================================================================
\end{tabular}
\caption{Plotting commands for demo {\tt pp2}.}
\label{tbl:demo_pp2_2}
\end{center}
\end{table}

%------------------------------------------------------
\begin{figure}[p]
\epsfysize 9.0cm
\centerline{\epsffile{include/pp21.ps}}
\caption{The bifurcation diagram of demo {\tt pp2}.}
\label{fig:pp2_1}
\end{figure}
%------------------------------------------------------

%------------------------------------------------------
\begin{figure}[p]
\epsfysize 9.0cm
\centerline{\epsffile{include/pp22.ps}}
\caption{The phase plot of solutions 11, 15, 19, and 23 in demo {\tt pp2}.}
\label{fig:pp2_2}
\end{figure}
%------------------------------------------------------


\newpage
%==============================================================================
%DEMO=lor======================================================================
%==============================================================================
\section{ lor : Starting an Orbit from Numerical Data.} \label{sec:Demos_lor}
This demo illustrates how to start the computation of a family of
periodic solutions from numerical data obtained, for example, from an
initial value solver.
As an illustrative application we consider the Lorenz equations
\begin{equation} \begin{array}{cl}
  u_1' &=  p_3 (u_2 - u_1), \\
  u_2' &=  p_1 u_1 - u_2 - u_1 u_3,  \\
  u_3' &=  u_1 u_2 - p_2 u_3. \\\end{array} \end{equation}
Numerical simulations with a simple initial value solver show the
existence of a stable periodic orbit when $p_1=280$, $p_2=8/3$, $p_3=10$.
Numerical data representing one complete periodic oscillation are
contained in the file {\tt lor.dat}. 
Each row in {\tt lor.dat} contains four real numbers, namely,
the time variable $t$, $u_1$, $u_2$ and $u_3$.
The correponding parameter values are defined in the user-supplied routine
{\tt STPNT}.
The {\cal AUTO}-command {\it @fc lor} then converts the data in {\tt lor.dat}
to a labeled
{\cal AUTO} solution (with label~1) in a new file {\tt s.dat}.
The mesh will be suitably adapted to the solution, using the number of
mesh intervals {\tt NTST} and the number of collocation point per mesh
interval {\tt NCOL} specified in the constants-file {\tt c.lor}.
(Note that the file {\tt s.dat} should be used for restart only.
Do not append new output-files to {\tt s.dat}, as the command {\it @fc lor}
only creates {\tt s.dat}, with no corresponding {\tt b.dat}.)

\begin{table}[htbp]
\begin{center}
\begin{tabular}{| l | l |}
\hline
  COMMAND  & ACTION \\
\hline
%==============================================================================
  {\it mkdir lor} & create an empty work directory \\ 
  {\it cd lor} & change directory \\
  {\it @dm lor} & copy the demo files to the work directory \\
\hline
%==============================================================================
  {\it cp c.lor.1 c.lor} & get the first constants-file \\ 
  {\it @fc lor} & convert {\tt lor.dat} to {\cal AUTO} format in {\tt s.dat} \\ 
\hline
%==============================================================================
  {\it @r lor dat} & compute a solution family, restart from {\tt s.dat} \\ 
  {\it @sv lor} & save output-files as {\tt b.lor, s.lor, d.lor} \\ 
\hline
%==============================================================================
  {\it cp c.lor.2 c.lor} & constants changed : {\tt IRS, ISW, NTST} \\ 
  {\it @r lor} & switch branches at a period-doubling detected in the first run \\ 
  {\it @ap lor} & append the output-files to {\tt b.lor, s.lor, d.lor} \\ 
\hline
%==============================================================================
\end{tabular}
\caption{Commands for running demo {\tt lor}.}
\label{tbl:demo_lor}
\end{center}
\end{table}

\newpage
%==============================================================================
%DEMO=frc======================================================================
%==============================================================================
\section{ frc : A Periodically Forced System.} \label{sec:Demos_frc}
This demo illustrates the computation of periodic solutions
to a periodically forced system.
In {\cal AUTO} this can be done by adding a nonlinear oscillator with
the desired periodic forcing as one of the solution components.
An example of such an oscillator is
\begin{equation} \begin{array}{cl}
 x'&=x + \beta y - x (x^{2} + y^{2}),  \\
 y'&=-\beta x + y - y (x^{2} + y^{2}), \\\end{array} \end{equation}
which has the asymptotically stable solution $x=sin (\beta t)$,
$y=cos (\beta t)$.
We couple this oscillator to the Fitzhugh-Nagumo equations~:
\begin{equation} \begin{array}{cl}
 v'&=\bigl( F(v) - w \bigr) / \eps,  \\
 w'&=v - dw - \bigl( b + r \sin(\beta t) \bigr) ,
\end{array} \end{equation}
by replacing $\sin(\beta t)$ by $x$.
Above, $F(v) = v (v-a) (1-v)$ and $a,b,\eps$ and $d$ are fixed.
The first run is a homotopy from $r=0$, where a solution is known analytically,
to $r=0.2$.
Part of the solution family with $r=0.2$ and varying $\beta$ 
is computed in the second run.
For detailed results see 
\citename{AlDoOt:90} \citeyear{AlDoOt:90}.


\begin{table}[htbp]
\begin{center}
\begin{tabular}{| l | l |}
\hline
  COMMAND  & ACTION \\
\hline
%==============================================================================
  {\it mkdir frc} & create an empty work directory \\ 
  {\it cd frc} & change directory \\
  {\it @dm frc} & copy the demo files to the work directory \\
\hline
%============================================================================== 
  {\it cp c.frc.1 c.frc} & get the first constants-file \\ 
  {\it @r frc} & homotopy to $r=0.2$ \\ 
  {\it @sv 0} & save output-files as {\tt b.0, s.0, d.0} \\ 
\hline
%==============================================================================
  {\it cp c.frc.2 c.frc} & constants changed : {\tt IRS, ICP(1), NTST, NMX, DS, DSMAX} \\ 
  {\it @r frc 0} & compute solution family; restart from {\tt s.0} \\ 
  {\it @sv frc} & save output-files as {\tt b.frc, s.frc, d.frc} \\ 
\hline
%==============================================================================
\end{tabular}
\caption{Commands for running demo {\tt frc}.}
\label{tbl:demo_frc}
\end{center}
\end{table}

\newpage
%==============================================================================
%DEMO=ppp======================================================================
%==============================================================================
\section{ ppp :  Continuation of Hopf Bifurcations.} \label{sec:Demos_ppp}
This demo illustrates the continuation of Hopf bifurcations in a 3-dimensional 
predator prey model (\citename{Do:84} \citeyear{Do:84}).
This curve contain branch points, where one locus of Hopf points
bifurcates from another locus of Hopf points.
The equations are
\begin{equation} \begin{array}{cl}
  u_1 ' &= u_1(1-u_1) - p_4 u_1 u_2  ,  \\
  u_2 ' &= -p_2 u_2 + p_4 u_1 u_2 - p_5 u_2 u_3
  -p_1(1-e^{-p_6 u_2}) \\
  u_3 ' &= -p_3 u_3  + p_5 u_2 u_3  .  \\  
\end{array} \end{equation}
Here $p_2=1/4$,  $p_3=1/2$,  $p_4=3$,  $p_5=3$,  $p_6=5$,
and $p_1$ is the free parameter.
In the continuation of Hopf points the parameter $p_4$
is also free.

\begin{table}[htbp]
\begin{center}
\begin{tabular}{| l | l |}
\hline
  COMMAND  & ACTION \\
\hline
%==============================================================================
  {\it mkdir ppp} & create an empty work directory \\ 
  {\it cd ppp} & change directory \\
  {\it @dm ppp} & copy the demo files to the work directory \\
\hline
%==============================================================================
  {\it cp c.ppp.1 c.ppp} & get the first constants-file \\ 
  {\it @r ppp} & compute stationary solutions; detect Hopf bifurcations \\ 
  {\it @sv ppp} & save output-files as {\tt b.ppp, s.ppp, d.ppp} \\ 
\hline
%==============================================================================
  {\it cp c.ppp.2 c.ppp} & constants changed : {\tt IPS, IRS, ICP}, etc.  \\ 
  {\it @r ppp} & compute a family of periodic solutions \\ 
  {\it @ap ppp} & append the output-files to {\tt b.ppp, s.ppp, d.ppp} \\ 
\hline
%==============================================================================

  {\it cp c.ppp.3 c.ppp} & constants changed : \\ 
  {\it @r ppp} & compute Hopf bifurcation curves \\ 
  {\it @sv hb} & save the output-files as {\tt b.hb, s.hb, d.hb} \\ 
\hline
%==============================================================================
\end{tabular}
\caption{Commands for running demo {\tt ppp}.}
\label{tbl:demo_ppp_1}
\end{center}
\end{table}


\newpage
%==============================================================================
%DEMO=plp======================================================================
%==============================================================================
\section{ plp : Fold Continuation for Periodic Solutions.} \label{sec:Demos_plp}
This demo, which corresponds to computations in 
\citename{DoKeKe:91a} \citeyear{DoKeKe:91a}, shows how one can
continue a fold on a family of periodic solution in two parameters.
The calculation of a locus of Hopf bifurcations is also included.
The equations, that model a one-compartment activator-inhibitor system 
(\citename{JPK:80} \citeyear{JPK:80}),
are given by
\begin{equation} \begin{array}{cl}
 s' &= (s_{0} - s) - \rho R (s,a), \\
 a' &=\alpha (a_{0} - a) - \rho R (s,a), \\
\end{array} \end{equation}
where
$$ R(s,a)={s a \over 1+s+ \kappa s^{2} },
\qquad \kappa  > 0. $$
The free parameter is $\rho$.
In the fold continuation $s_0$ is also free.

\begin{table}[htbp]
\begin{center}
\begin{tabular}{| l | l |}
\hline
  COMMAND  & ACTION \\
\hline
%==============================================================================
  {\it mkdir plp} & create an empty work directory \\ 
  {\it cd plp} & change directory \\
  {\it @dm plp} & copy the demo files to the work directory \\
\hline
%==============================================================================
  {\it cp c.plp.1 c.plp} & get the first constants-file \\ 
  {\it @r plp} & 1st run; compute a stationary solution family and locate HBs \\ 
  {\it @sv plp} & save output-files as {\tt b.plp, s.plp, d.plp} \\ 
\hline
%==============================================================================
  {\it cp c.plp.2 c.plp} & constants changed : {\tt IPS, IRS, NMX} \\ 
  {\it @r plp} & compute a family of periodic solutions and locate a fold \\ 
  {\it @ap plp} & append output-files to {\tt b.plp, s.plp, d.plp} \\ 
\hline
%==============================================================================
  {\it cp c.plp.3 c.plp} & constants changed : {\tt IPS, ICP, ISW, NMX, RL1} \\ 
  {\it @r plp} & Compute a locus of Hopf bifurcation points \\ 
  {\it @sv 2p} & save output-files as {\tt b.2p, s.2p, d.2p} \\ 
\hline
%==============================================================================
  {\it cp c.plp.4 c.plp} & constants changed : {\tt IPS, IRS, ICP, NMX} \\ 
  {\it @r plp} & generate starting data for the fold continuation \\ 
  {\it @sv tmp} & save output-files as {\tt b.tmp, s.tmp, d.tmp} \\ 
\hline
%==============================================================================
  {\it cp c.plp.5 c.plp} & constants changed : {\tt IRS, NUZR} \\ 
  {\it @r plp tmp} &  fold continuation; restart data from {\tt s.tmp} \\ 
  {\it @ap 2p} & append output-files to {\tt b.2p, s.2p, d.2p} \\ 
\hline
%==============================================================================
  {\it cp c.plp.6 c.plp} & constants changed : {\tt IRS, ISW, NMX, NUZR }\\ 
  {\it @r plp 2p} & compute an isola of periodic solutions; restart data from {\tt s.2p} \\ 
  {\it @sv iso} & save output-files as {\tt b.iso, s.iso, d.iso} \\ 
\hline
%==============================================================================
\end{tabular}
\caption{Commands for running demo {\tt plp}.}
\label{tbl:demo_plp}
\end{center}
\end{table}

\newpage
%==============================================================================
%DEMO=pp3======================================================================
%==============================================================================
\section{ pp3 : Periodic Families and Loci of Hopf Points.} \label{sec:Demos_pp3}
This demo illustrates the computation of stationary solution families
that contain Hopf bifurcations, and the computation of the emanating
families of periodic solutions.  In this example the periodic 
solution families intersect at a secondary bifurcation point 
(a branch point). It it also shown how to compute a locus of Hopf bifurcation
points in two parameters. (In this example the locus contains branch points,
which lead to another locus!)

The equations, which model a 3D predator-prey system with harvesting
(\citename{Do:84} \citeyear{Do:84}), are 
\begin{equation} \begin{array}{cl}
  u_1 ' &= u_1(1-u_1) - p_4 u_1 u_2  ,  \\
  u_2 ' &= -p_2 u_2 + p_4 u_1 u_2 - p_5 u_2 u_3
  -p_1(1-e^{-p_6 u_2}) \\
  u_3 ' &= -p_3 u_3  + p_5 u_2 u_3  .  \\\end{array} \end{equation}
The free parameter is $p_1$, while the other parameters are fixed,
namely $p_2=0.25$, $p_3=0.5$, $p_4=4$, $p_5=3$, and $p_6=5$.
However, both $p_1$ and $p_4$ are free in the computation of loci of Hopf points.

\begin{table}[htbp]
\begin{center}
\begin{tabular}{| l | l |}
\hline
  COMMAND  & ACTION \\
\hline
%==============================================================================
  {\it mkdir pp3} & create an empty work directory \\ 
  {\it cd pp3} & change directory \\
  {\it @dm pp3} & copy the demo files to the work directory \\
\hline
%==============================================================================
  {\it @R pp3 1} & 1st run; stationary solutions with 4 Hopf bifurcations \\ 
  {\it @sv pp3} & save output-files as {\tt b.pp3, s.pp3, d.pp3} \\ 
\hline
%==============================================================================
  {\it @R pp3 2} & compute a family of periodic solutions \\ 
  {\it @ap pp3} & append output-files to {\tt b.pp3, s.pp3, d.pp3} \\ 
\hline
%==============================================================================
  {\it @R pp3 3} & compute another family of periodic solutions \\ 
  {\it @ap pp3} & append output-files to {\tt b.pp3, s.pp3, d.pp3} \\ 
\hline
%==============================================================================
  {\it @R pp3 4} & compute loci of Hopf points \\ 
  {\it @rl} & relabel the labeled solutions from this run \\ 
  {\it @sv hb} & save the output-files as {\tt b.hb, s.hb, d.pp3} \\ 
\hline
%==============================================================================
\end{tabular}
\caption{Commands for running demo {\tt pp3}.}
\label{tbl:demo_pp3}
\end{center}
\end{table}

\begin{table}[htbp]
\begin{center}
\begin{tabular}{| l | l |}
\hline
  {\cal AUTO}-COMMAND  & ACTION \\
\hline
  {\it @p pp3} & run {\cal PLAUT} to graph the contents of {\tt b.pp3} and {\tt s.pp3}; \\ 
\hline
  {\cal PLAUT}-COMMAND  & ACTION \\
\hline
  {\it d2}  & set convenient defaults\\ 
  {\it ax}  & select axes \\ 
  {\it 1 3}  & select real columns 1 and 3 in {\tt b.pp3} \\ 
  {\it bd0}  & plot the bifurcation diagram; $max~u_1$ versus $p_1$ \\
\hline
  {\it bd}  & get blow-up of current bifurcation diagram \\ 
  {\it 0~ 0.6 ~0~ 1.2} & enter diagram limits  \\
\hline
  {\it d1}  & choose other default settings (with labels) \\ 
  {\it bd}  & another blow-up of the bifurcation diagram \\ 
  {\it 0~ 0.6 ~0~ 0.75} & enter diagram limits  \\
\hline
  {\it d2}  & set defaults\\ 
  {\it 2d}  & enter 2D mode, for plotting labeled solutions\\ 
  {\it 13 15 17 19 21 23 25 27}  & select these orbits from {\tt s.pp3}\\ 
  {\it d}  & default orbit display; $u_1$ versus time\\
\hline
  {\it 2 3}  & select columns 2 and 3 in {\tt s.pp3} \\
  {\it d}  & display the orbits; $u_2$ versus $u_1$\\
\hline
  {\it 2d}  & enter 2D mode, for plotting labeled solutions\\ 
  {\it 28 30 32 34 36 38 40 42 44}  & select these orbits\\ 
  {\it d}  & default orbit display; $u_1$ versus time\\
\hline
  {\it 2 3}  & select columns 2 and 3 in {\tt s.pp3} \\
  {\it d}  & phase plane display; $u_2$ versus $u_1$\\
\hline
  {\it 2 4}  & select columns 2 and 4 in {\tt s.pp3} \\
  {\it d}  & phase plane display; $u_3$ versus $u_1$\\
  {\it ex}  & exit from 2D mode  \\
\hline
  {\it end}  & exit from {\cal PLAUT} \\
\hline
%==============================================================================
\end{tabular}
\caption{Plotting commands for demo {\tt pp3}.}
\label{tbl:demo_pp3_2}
\end{center}
\end{table}

\begin{table}[htbp]
\begin{center}
\begin{tabular}{| l | l |}
\hline
  {\cal AUTO}-COMMAND  & ACTION \\
\hline
  {\it @p hb} & run {\cal PLAUT} to graph the contents of {\tt b.hb} and {\tt s.hb}; \\ 
\hline
  {\cal PLAUT}-COMMAND  & ACTION \\
\hline
  {\it d0}  & set defaults\\ 
  {\it ax}  & select axes \\ 
  {\it 1 6}  & select real columns 1 and 6 in {\tt b.hb} \\ 
  {\it bd0}  & plot the bifurcation diagram; $p_4$ versus $p_1$ \\
\hline
  {\it end}  & exit from {\cal PLAUT} \\
\hline
%==============================================================================
\end{tabular}
\caption{Plotting the Hopf loci for demo {\tt pp3}.}
\label{tbl:demo_pp3_3}
\end{center}
\end{table}


\newpage
%==============================================================================
%DEMO=tor======================================================================
%==============================================================================
\section{ tor : Detection of Torus Bifurcations.} \label{sec:Demos_tor}
This demo uses a model in 
\citename{FrRLuGaPo:93} \citeyear{FrRLuGaPo:93}
 to illustrate the detection of a torus bifurcation. 
It also illustrates branch switching at a secondary periodic bifurcation
with double Floquet multiplier at $z=1$.
The computational results also include folds, homoclinic orbits,
and period-doubling bifurcations.
Their continuation is not illustrated here;
see instead the demos {\tt plp}, {\tt pp2}, and {\tt pp3}, respectively.  
The equations are
\begin{equation} \begin{array}{cl}
  x'(t) & = \bigr[ -(\beta+\nu)x + \beta y - a_3 x^3 + b_3 (y-x)^3 \bigr] / r,\\
  y'(t) &= \beta x - (\beta + \gamma) y - z - b_3 (y-x)^3, \\
  z'(t) &= y,\end{array} \end{equation}
where $\gamma=-0.6$, $r=0.6$, $a_3=0.328578$, and $b_3=0.933578$.
Initially $\nu=-0.9$ and $\beta=0.5$.

\begin{table}[htbp]
\begin{center}
\begin{tabular}{| l | l |}
\hline
  COMMAND  & ACTION \\
\hline
%==============================================================================
  {\it mkdir tor} & create an empty work directory \\ 
  {\it cd tor} & change directory \\
  {\it @dm tor} & copy the demo files to the work directory \\
\hline
%==============================================================================
  {\it cp c.tor.1 c.tor} & get the first constants-file \\ 
  {\it @r tor} & 1st run; compute a stationary solution family with Hopf bifurcation \\ 
  {\it @sv 1} & save output-files as {\tt b.1, s.1, d.1} \\ 
\hline
%==============================================================================
  {\it cp c.tor.2 c.tor} & constants changed : {\tt IPS, IRS }\\ 
  {\it @r tor 1} & compute a family of periodic solutions; restart from {\tt s.1} \\ 
  {\it @ap 1} & append output-files to {\tt b.1, s.1, d.1} \\ 
\hline
%==============================================================================
  {\it cp c.tor.3 c.tor} & constants changed : {\tt IRS, ISW, NMX} \\ 
  {\it @r tor 1} & compute a bifurcating family of periodic solutions; restart from {\tt s.1} \\ 
  {\it @ap 1} & append output-files to {\tt b.1, s.1, d.1} \\ 
\hline
%==============================================================================
\end{tabular}
\caption{Commands for running demo {\tt tor}.}
\label{tbl:demo_tor}
\end{center}
\end{table}

\newpage
%==============================================================================
%DEMO=pen======================================================================
%==============================================================================
\section{ pen : Rotations of Coupled Pendula.} \label{sec:Demos_pen}
This demo illustrates the computation of rotations, i.e., solutions that
are periodic, modulo a phase gain of an even multiple of $\pi$.
{\cal AUTO} checks the starting data for components with such a phase gain
and, if present, it will automatically adjust the computations accordingly.
The model equations, a system of two coupled  pendula, 
(\citename{DoArOt:91} \citeyear{DoArOt:91}),
are given by
\begin{equation} \begin{array}{cl}
 & \phi_1'' + \eps \phi_1' + \sin \phi_1 
  = I + \gamma(\phi_2-\phi_1), \\
 & \phi_2'' + \eps \phi_2' + \sin \phi_2 
  = I + \gamma(\phi_1-\phi_2) ,\\
\end{array} \end{equation}
or, in equivalent first order form,
\begin{equation} \begin{array}{cl}
 & \phi_1'  =  \psi_1, \\
 & \phi_2'  =  \psi_2, \\
 & \psi_1'  = - \eps \psi_1 - \sin \phi_1 + I + \gamma(\phi_2-\phi_1), \\
 & \psi_2'  = - \eps \psi_2 - \sin \phi_2 + I + \gamma(\phi_1-\phi_2).\\
\end{array} \end{equation}
Throughout $\gamma=0.175$. Initially, $\eps=0.1$ and $I=0.4$.

Numerical data representing one complete rotation are
contained in the file {\tt pen.dat}. 
Each row in {\tt pen.dat} contains five real numbers, namely,
the time variable $t$, $\phi_1$, $\phi_2$, $\psi_1$ and $\psi_2$.
The correponding parameter values are defined in the user-supplied routine
{\tt STPNT}.

Actually, in this example, a scaled time variable $t$ is given in {\tt pen.dat}. 
For this reason the period ({\tt PAR(11)}) is also set in {\tt STPNT}.
Normally {\cal AUTO} would automatically set the period according to
the data in {\tt pen.dat}.

The {\cal AUTO}-command {\it @fc pen} converts the data in {\tt pen.dat}
to a labeled
{\cal AUTO} solution (with label~1) in a new file {\tt s.dat}.
The mesh will be suitably adapted to the solution, using the number of
mesh intervals {\tt NTST} and the number of collocation point per mesh
interval {\tt NCOL} specified in the constants-file {\tt c.pen}.
(Note that the file {\tt s.dat} should be used for restart only.
Do not append new output-files to {\tt s.dat}, as the command {\it @fc pen}
only creates {\tt s.dat}, with no corresponding {\tt b.dat}.)

The first run, with $I$ as free problem parameter,
starts from the converted solution with label~1 in {\tt pen.dat}.
A period-doubling bifurcation is located, and the period-doubled family
is computed in the second run.
Two branch points are located, and the bifurcating
families are traced out in the third and fourth run, respectively.
The fifth run generates starting data for the subsequent computation of
a locus of period-doubling bifurcations.
The actual computation is done in the sixth run, with $\eps$ and $I$
as free problem parameters.

\begin{table}[htbp]
\begin{center}
\begin{tabular}{| l | l |}
\hline
  COMMAND  & ACTION \\
\hline
%==============================================================================
  {\it mkdir pen} & create an empty work directory \\ 
  {\it cd pen} & change directory \\
  {\it @dm pen} & copy the demo files to the work directory \\
\hline
%==============================================================================
  {\it cp c.pen.1 c.pen} & get the first constants-file \\ 
  {\it @fc pen} & convert {\tt pen.dat} to {\cal AUTO} format in {\tt s.dat} \\ 
\hline
%==============================================================================
  {\it @r pen dat} & locate a period doubling bifurcation; restart from {\tt s.dat} \\ 
  {\it @sv pen} & save output-files as {\tt b.pen, s.pen, d.pen} \\ 
\hline
%==============================================================================
  {\it cp c.pen.2 c.pen} & constants changed : {\tt IPS, NTST, ISW, NMX} \\ 
  {\it @r pen} & a family of  period-doubled (and out-of-phase) rotations \\ 
  {\it @ap pen} & append output-files tp {\tt b.pen, s.pen, d.pen} \\ 
\hline
%============================================================================== 
  {\it cp c.pen.3 c.pen} & constants changed : {\tt IRS, ISP} \\ 
  {\it @r pen} &  a secondary bifurcating family (without bifurcation detection) \\ 
  {\it @ap pen} & append output-files to {\tt b.pen, s.pen, d.pen} \\ 
\hline
%==============================================================================
  {\it cp c.pen.4 c.pen} & constants changed : {\tt IRS} \\ 
  {\it @r pen} &  another secondary bifurcating family (without bifurcation detection) \\ 
  {\it @ap pen} & append output-files to {\tt b.pen, s.pen, d.pen} \\ 
\hline
%==============================================================================
  {\it cp c.pen.5 c.pen} & constants changed : {\tt IRS, ICP, ICP, ISW, NMX} \\ 
  {\it @r pen} &  generate starting data for period doubling continuation \\ 
  {\it @sv t} & save output-files as {\tt b.t, s.t, d.t} \\ 
\hline
%==============================================================================
  {\it cp c.pen.6 c.pen} & constants changed : {\tt IRS} \\ 
  {\it @r pen t} &  compute a locus of period doubling bifurcations; restart from {\tt s.t} \\ 
  {\it @sv pd} & save output-files as {\tt b.pd, s.pd, d.pd} \\ 
\hline
%==============================================================================
  {\it @pn pen} & run an animation program to view the solutions in {\tt s.pen} \\ 
  & (on SGI machines only; see also the file {\tt auto/07/pendula/README}).
  \\ 
\hline
%==============================================================================
\end{tabular}
\caption{Commands for running demo {\tt pen}.}
\label{tbl:demo_pen}
\end{center}
\end{table}

\newpage
%==============================================================================
%DEMO=chu======================================================================
%==============================================================================
\section{ chu :  A Non-Smooth System (Chua's Circuit).} \label{sec:Demos_chu}
Chua's circuit 
is one of the simplest electronic devices to exhibit complex behavior. 
For related calculations see
\citename{KhRoCh:93} \citeyear{KhRoCh:93}.
The equations modeling the circuit are
\begin{equation} \begin{array}{cl}
 u_1' &=  \alpha \bigl[~ u_2 - h(u_1) ~\bigr]~,\\ 
 u_2' &=  u_1 - u_2 + u_3~, \\  
 u_3' &=  - \beta~ u_2~,  
\end{array} \end{equation}
where
$$ h(x) = a_1 x + {1 \over 2}~ (a_0 - a_1) ~
  \bigl\{ \abs{x+1} -  \abs{x-1} \bigr\}~,$$
and where we take
$\beta = 14.3$, $a_0 = - 1/7$, $a_1 = 2/7$.

Note that $h(x)$ is not a smooth function, and hence the solution 
to the equations  may have non-smooth derivatives.
However, for the orthogonal collocation method to attain its optimal accuracy,
it is necessary that the solution be sufficiently smooth.
Moreover, the adaptive mesh selection strategy will fail
if the solution or one of its lower order derivatives has discontinuities.
For these reasons  we use the smooth approximation
$$ \abs{x} ~\approx~ {2 x \over \pi } ~ {\rm arctan}(Kx),$$
which get better as $K$ increases.
In the numerical calculations below we use $K = 10$.
The free parameter is $\alpha$.


\begin{table}[htbp]
\begin{center}
\begin{tabular}{| l | l |}
\hline
  COMMAND  & ACTION \\
\hline
%==============================================================================
  {\it mkdir chu} & create an empty work directory \\ 
  {\it cd chu} & change directory \\
  {\it @dm chu} & copy the demo files to the work directory \\
\hline
%==============================================================================
  {\it cp c.chu.1 c.chu} & get the first constants-file \\ 
  {\it @r chu} & 1st run; stationary solutions \\ 
  {\it @sv chu} & save output-files as {\tt b.chu, s.chu, d.chu} \\ 
\hline
%==============================================================================
  {\it cp c.chu.2 c.chu} & constants changed : {\tt IPS, IRS, ICP, ICP} \\ 
  {\it @r chu} & 2nd run; periodic solutions, with detection of period-doubling \\ 
  {\it @ap chu} & append the output-files to {\tt b.chu, s.chu, d.chu} \\ 
\hline
%==============================================================================
\end{tabular}
\caption{Commands for running demo {\tt chu}.}
\label{tbl:demo_chu}
\end{center}
\end{table}

\newpage
%==============================================================================
%DEMO=phs======================================================================
%==============================================================================
\section{ phs : Effect of the Phase Condition.} \label{sec:Demos_phs}
This demo illustrates the effect of the phase condition 
on the computation of periodic solutions.
We consider the differential equation
\begin{equation} \begin{array}{cl}
 u_1'&= \lambda u_1 - u_2,  \\
 u_2'&= u_1 (1-u_1) .  \\
\end{array} \end{equation}
This equation has a Hopf bifurcation from the trivial solution at $\lambda=0$. 
The bifurcating family of periodic solutions
is vertical and along it the period increases monotonically.
The family terminates in a homoclinic orbit containing the
saddle point $(u_1,u_2)=(1,0)$.
Graphical inspection of the computed periodic orbits,
for example $u_1$ versus the scaled time variable $t$,
shows how the phase condition has the effect of keeping the ``peak'' 
in the solution in the same location.

\begin{table}[htbp]
\begin{center}
\begin{tabular}{| l | l |}
\hline
  COMMAND  & ACTION \\
\hline
%==============================================================================
  {\it mkdir phs} & create an empty work directory \\ 
  {\it cd phs} & change directory \\
  {\it @dm phs} & copy the demo files to the work directory \\
\hline
%==============================================================================
  {\it cp c.phs.1 c.phs} & get the first constants-file \\ 
  {\it @r phs} & detect Hopf bifurcation \\ 
  {\it @sv phs} & save output-files as {\tt b.phs, s.phs, d.phs} \\ 
\hline
%==============================================================================
  {\it cp c.phs.2 c.phs} & constants changed : {\tt IRS, IPS, NPR} \\ 
  {\it @r phs} & compute periodic solutions \\ 
  {\it @ap phs} & append output-files to {\tt b.phs, s.phs, d.phs} \\ 
\hline
%==============================================================================
\end{tabular}
\caption{Commands for running demo {\tt phs}.}
\label{tbl:demo_phs}
\end{center}
\end{table}

\newpage
%==============================================================================
%DEMO=ivp======================================================================
%==============================================================================
\section{ ivp :  Time Integration with Euler's Method.} \label{sec:Demos_ivp}
This demo uses Euler's method to locate a stationary solution of the
following predator-prey system with harvesting~:

\begin{equation} \begin{array}{cl}
  u_1 ' &= p_2 u_1 (1 - u_1 ) - u_1 u_2 - p_1 (1-e^{-p_3 u_1}) ,\\
  u_2 ' &= -u_2  + p_4 u_1 u_2  ,\\\end{array} \end{equation}
where all problem parameters have a fixed value.
The equations are the same as those in demo {\tt pp2}.
The continuation parameter is the independent time variable, namely {\tt PAR(14)}.

Note that Euler time integration is only first order accurate, so that
the time step must be sufficiently small to ensure correct results.
Indeed, this option has been added only as a convenience, and should 
generally be used only to locate stationary states.
Note that the {\cal AUTO}-constants {\tt DS}, {\tt DSMIN}, and {\tt DSMAX}
control the step size
in the space consisting of time, here {\tt PAR(14)}, and the state vector,
here $(u_1,u_2)$.

\begin{table}[htbp]
\begin{center}
\begin{tabular}{| l | l |}
\hline
  COMMAND  & ACTION \\
\hline
%==============================================================================
  {\it mkdir ivp} & create an empty work directory \\ 
  {\it cd ivp} & change directory \\
  {\it @dm ivp} & copy the demo files to the work directory \\
\hline
%==============================================================================
  {\it cp c.ivp.1 c.ivp} & get the constants-file \\ 
  {\it @r ivp} & time integration \\ 
  {\it @sv ivp} & save output-files as {\tt b.ivp, s.ivp, d.ivp} \\ 
\hline
%==============================================================================
\end{tabular}
\caption{Commands for running demo {\tt ivp}.}
\label{tbl:demo_ivp}
\end{center}
\end{table}


%==============================================================================
%==============================================================================
\chapter{ {\cal AUTO} Demos : BVP.} \label{ch:Demos_BVP}
%==============================================================================
%==============================================================================

%==============================================================================
%DEMO=exp======================================================================
%==============================================================================
\section{ exp : Bratu's Equation.} \label{sec:Demos_exp}
This demo illustrates the computation of a solution family to
the boundary value problem


\begin{equation} \begin{array}{cl}
  u_1 ' &= u_2  ,  \\
  u_2 ' &= -p_1  e^{u_1} , \\
\end{array} \end{equation}
with boundary conditions $ u_1(0)=0 ,  \quad  u_1(1)=0.$
This equation is also considered in 
\citename{DoKeKe:91a} \citeyear{DoKeKe:91a}.
\begin{table}[htbp]
\begin{center}
\begin{tabular}{| l | l |}
\hline
  COMMAND  & ACTION \\
\hline
%==============================================================================
  {\it mkdir exp} & create an empty work directory \\ 
  {\it cd exp} & change directory \\
  {\it @dm exp} & copy the demo files to the work directory \\
\hline
%==============================================================================
  {\it @R exp 1} & 1st run; compute solution family containing fold \\ 
  {\it @sv exp} & save output-files as {\tt b.exp, s.exp, d.exp} \\ 
\hline
%==============================================================================
  {\it @r exp 2} & 2nd run; restart at a labeled solution, using increased accuracy \\ 
  {\it @ap exp} & append output-files to {\tt b.exp, s.exp, d.exp} \\ 
\hline
%==============================================================================
\end{tabular}
\caption{Commands for running demo {\tt exp}.}
\label{tbl:demo_exp}
\end{center}
\end{table}

\newpage
%==============================================================================
%DEMO=int======================================================================
%==============================================================================
\section{ int : Boundary and Integral Constraints.} \label{sec:Demos_int}
This demo illustrates the computation of a solution family to
the equation

\begin{equation} \begin{array}{cl}
 u_1 ' &= u_2 , \\
  u_2 ' &= -p_1  e^{u_1} , \\\end{array} \end{equation}
with a non-separated boundary condition and an integral constraint:

$$ u_1(0)-u_1(1)-p_2=0 ,\qquad \int_0^{1}u(t)dt-p_3=0 . $$
The solution family contains a fold, which, in the second run, is  
continued in two equation parameters.

\begin{table}[htbp]
\begin{center}
\begin{tabular}{| l | l |}
\hline
  COMMAND  & ACTION \\
\hline
%==============================================================================
  {\it mkdir int} & create an empty work directory \\ 
  {\it cd int} & change directory \\
  {\it @dm int} & copy the demo files to the work directory \\
\hline
%==============================================================================
  {\it cp c.int.1 c.int} & get the first constants-file \\ 
  {\it @r int} & 1st run; detection of a fold \\ 
  {\it @sv int} & save output-files as {\tt b.int, s.int, d.int} \\ 
\hline
%==============================================================================
  {\it cp c.int.2 c.int} & constants changed : {\tt IRS, ISW }\\ 
  {\it @r int} & 2nd run; generate starting data for a curve of folds \\ 
  {\it @sv t} & save the output-files as {\tt b.t, s.t, d.t} \\ 
\hline
%==============================================================================

  {\it cp c.int.3 c.int} & constants changed : {\tt IRS} \\ 
  {\it @r int t} & 2nd run; compute a curve of folds; restart from {\tt s.t} \\ 
  {\it @sv lp} & save the output-files as {\tt b.lp, s.lp, d.lp} \\ 
\hline
%==============================================================================
\end{tabular}
\caption{Commands for running demo {\tt int}.}
\label{tbl:demo_int}
\end{center}
\end{table}

\newpage
%==============================================================================
%DEMO=bvp======================================================================
%==============================================================================
\section{ bvp : A Nonlinear ODE Eigenvalue Problem.} \label{sec:Demos_bvp}
This demo illustrates the location of eigenvalues of a nonlinear ODE
boundary value problem as bifurcations from the trivial solution family.
The family of solutions that bifurcates at the first eigenvalue
is computed in both directions.
The equations are
\begin{equation} \begin{array}{cl}
 u_1 ' &= u_2  ,  \\
  u_2 ' &=-(p_1 \pi)^{2}u_1 + u_1^{2} ,\end{array} \end{equation}
with boundary conditions $ u_1(0)=0 ,  \quad  u_1(1)=0.$~~~

\begin{table}[htbp]
\begin{center}
\begin{tabular}{| l | l |}
\hline
  COMMAND  & ACTION \\
\hline
%==============================================================================
  {\it mkdir bvp} & create an empty work directory \\ 
  {\it cd bvp} & change directory \\
  {\it @dm bvp} & copy the demo files to the work directory \\
\hline
%==============================================================================
  {\it cp c.bvp.1 c.bvp} & get the first constants-file \\ 
  {\it @r bvp} &  compute the trivial solution family and locate eigenvalues \\ 
  {\it @sv bvp} & save output-files as {\tt b.bvp, s.bvp, d.bvp} \\ 
\hline
%==============================================================================
  {\it cp c.bvp.2 c.bvp} & constants changed : {\tt IRS, ISW, NPR, DSMAX} \\ 
  {\it @r bvp} &  compute the first bifurcating family \\ 
  {\it @ap bvp} & append output-files to {\tt b.bvp, s.bvp, d.bvp} \\ 
\hline
%==============================================================================
  {\it cp c.bvp.3 c.bvp} & constants changed : {\tt DS} \\ 
  {\it @r bvp} &  compute the first bifurcating family in opposite direction \\ 
  {\it @ap bvp} & append output-files to {\tt b.bvp, s.bvp, d.bvp} \\ 
\hline
%==============================================================================
\end{tabular}
\caption{Commands for running demo {\tt bvp}.}
\label{tbl:demo_bvp}
\end{center}
\end{table}

\newpage
%==============================================================================
%DEMO=lin======================================================================
%==============================================================================
\section{ lin : A Linear ODE Eigenvalue Problem.} \label{sec:Demos_lin}
This demo illustrates the location of eigenvalues of a linear ODE
boundary value problem as bifurcations from the trivial solution family.
By means of branch switching an eigenfunction is computed,
as is illustrated for the first eigenvalue. 
This eigenvalue is then continued in two parameters
by fixing the $L_2$-norm of the first solution component.
The eigenvalue problem is given by the equations

\begin{equation} \begin{array}{cl}
  u_1 ' &= u_2  ,  \\
  u_2 ' &= (p_1 \pi)^{2} u_1 , \end{array} \end{equation}
with boundary conditions $ u_1(0)-p_2=0 $ and $  u_1(1)=0.$
We add the integral constraint
 $$ \int_0^{1} u_1(t)^{2} dt - p_3 = 0. $$
Then $p_3$ is simply the $L_2$-norm of the first solution component.
In the first two runs $p_2$ is fixed, while $p_1$ and $p_3$ are free.
In the third run  $p_3$ is fixed, while $p_1$ and $p_2$ are free.

\begin{table}[htbp]
\begin{center}
\begin{tabular}{| l | l |}
\hline
  COMMAND  & ACTION \\
\hline
%==============================================================================
  {\it mkdir lin} & create an empty work directory \\ 
  {\it cd lin} & change directory \\
  {\it @dm lin} & copy the demo files to the work directory \\
\hline
%==============================================================================
  {\it cp c.lin.1 c.lin} & get the first constants-file \\ 
  {\it @r lin} & 1st run; compute the trivial solution family and locate eigenvalues \\ 
  {\it @sv lin} & save output-files as {\tt b.lin, s.lin, d.lin} \\ 
\hline
%==============================================================================
  {\it cp c.lin.2 c.lin} & constants changed : {\tt IRS, ISW, DSMAX} \\ 
  {\it @r lin} & 2nd run; compute a few steps along the bifurcating family \\ 
  {\it @ap lin} & append output-files to {\tt b.lin, s.lin, d.lin} \\ 
\hline
%==============================================================================
  {\it cp c.lin.3 c.lin} & constants changed : {\tt IRS, ISW, ICP(2)} \\ 
  {\it @r lin} & 3rd run; compute a two-parameter curve of eigenvalues \\ 
  {\it @sv 2p} & save the output-files as {\tt b.2p, s.2p, d.2p} \\ 
\hline
%==============================================================================
\end{tabular}
\caption{Commands for running demo {\tt lin}.}
\label{tbl:demo_lin}
\end{center}
\end{table}

\newpage
%==============================================================================
%DEMO=non======================================================================
%==============================================================================
\section{ non : A Non-Autonomous BVP.} \label{sec:Demos_non}
This demo illustrates the continuation of solutions to
the non-autonomous boundary value problem

\begin{equation} \begin{array}{cl}
  u_1 ' &= u_2  ,  \\
  u_2 ' &= -p_1  e^{x^3 u_1} , \\\end{array} \end{equation}
with boundary conditions $ u_1(0)=0 ,  \quad  u_1(1)=0.$
Here $x$ is the independent variable.
This system is first converted to the following equivalent
autonomous system~:
\begin{equation} \begin{array}{cl}
  u_1 ' &= u_2  ,  \\
  u_2 ' &= -p_1  e^{u_3^3 u_1} ,  \\  
  u_3 ' &= 1 ,  \\
\end{array} \end{equation}
 with boundary conditions $ u_1(0)=0 ,  \quad  u_1(1)=0, \quad u_3(0)=0.$
(For a periodically forced system see demo {\tt frc}).

\begin{table}[htbp]
\begin{center}
\begin{tabular}{| l | l |}
\hline
  COMMAND  & ACTION \\
\hline
%==============================================================================
  {\it mkdir non} & create an empty work directory \\ 
  {\it cd non} & change directory \\
  {\it @dm non} & copy the demo files to the work directory \\
\hline
%==============================================================================

  {\it cp c.non.1 c.non} & get the constants-file \\ 
  {\it @r non} & compute the solution family \\ 
  {\it @sv non} & save output-files as {\tt b.non, s.non, d.non} \\ 
\hline
%==============================================================================
\end{tabular}
\caption{Commands for running demo {\tt non}.}
\label{tbl:demo_non}
\end{center}
\end{table}

\newpage
%==============================================================================
%DEMO=kar======================================================================
%==============================================================================
\section{ kar : The Von Karman Swirling Flows.} \label{sec:Demos_kar}
The steady axi-symmetric flow of a viscous incompressible fluid
above an infinite rotating disk is modeled by the following 
ODE boundary value problem (Equation (11) in
\citename{LeKe:80} \citeyear{LeKe:80}~:
\begin{equation} \begin{array}{cl}
  u_1' &= T u_2,  \\
  u_2' &= T u_3,  \\
  u_3' &= T \bigl[ -2 \gamma u_4 + u_2^2 - 2 u_1 u_3 - u_4^2 \bigr], \\
  u_4' &= T u_5, \\
  u_5' &= T \bigl[ 2 \gamma u_2 + 2 u_2 u_4 - 2 u_1 u_5 \bigr], \\
\end{array} \end{equation}
with left boundary conditions
$$ u_1(0)=0, \qquad u_2(0)=0, \qquad u_4(0)=1-\gamma, $$
and (asymptotic) right boundary conditions
\begin{equation} \begin{array}{cl}
  & \bigl[ f_\infty + a(f_\infty,\gamma) \bigr] ~ u_2(1) + u_3(1)
  - \gamma ~ { u_4(1) \over a(f_\infty,\gamma) } = 0,  \\
  & a(f_\infty,\gamma)~ { b^2(f_\infty,\gamma) \over \gamma } ~u_2(1)
  + \bigl[ f_\infty + a(f_\infty,\gamma) \bigr] ~u_4(1) 
  + u_5(1) = 0, \\
 & u_1(1) = f_\infty,
 \end{array} \end{equation}
where
\begin{equation} \begin{array}{cl}
 & a(f_\infty,\gamma) = {1 \over \sqrt{2} }
  \bigl[ (f_\infty^4 + 4 \gamma^2)^{1/2} + f_\infty^2 \bigr]^{1/2}, \\
 & b(f_\infty,\gamma) = {1 \over \sqrt{2} }
  \bigl[ (f_\infty^4 + 4 \gamma^2)^{1/2} - f_\infty^2 \bigr]^{1/2}.  \\
\end{array} \end{equation}
Note that there are five differential equations and six boundary conditions.
Correspondingly, there are two free parameters in the computation of a 
solution family, namely $\gamma$ and $f_\infty$.
The ``period'' $T$ is fixed; $T=500$.
The starting solution is $u_i=0$, $i=1,\cdots,5$, 
at $\gamma=1$, $f_\infty=0$.

\begin{table}[htbp]
\begin{center}
\begin{tabular}{| l | l |}
\hline
  COMMAND  & ACTION \\
\hline
%==============================================================================
  {\it mkdir kar} & create an empty work directory \\ 
  {\it cd kar} & change directory \\
  {\it @dm kar} & copy the demo files to the work directory \\
\hline
%==============================================================================
  {\it cp c.kar.1 c.kar} & get the constants-file \\ 
  {\it @r kar} & computation of the solution family \\ 
  {\it @sv kar} & save output-files as {\tt b.kar, s.kar, d.kar} \\ 
\hline
%==============================================================================
\end{tabular}
\caption{Commands for running demo {\tt kar}.}
\label{tbl:demo_kar}
\end{center}
\end{table}

\newpage
%==============================================================================
%DEMO=spb======================================================================
%==============================================================================
\section{ spb : A Singularly-Perturbed BVP.} \label{sec:Demos_spb}
This demo illustrates the use of continuation to compute 
solutions to the singularly perturbed boundary value problem
\begin{equation} \begin{array}{cl}
  u_1 ' &= u_2  ,  \\
  u_2 ' &= {\lambda \over \eps} \bigl(
  u_1 u_2 (u_1^2 - 1) + u_1
  \bigr)  , \\ \end{array} \end{equation}
with boundary conditions $u_1(0)=3/2$,  $u_1(1)=\gamma.$
The parameter $\lambda$ has been introduced into the equations in order
to allow a homotopy from a simple equation with known exact solution
to the actual equation. This is done in the first run.
In the second run $\eps$ is decreased by continuation.
In the third run $\eps$ is fixed at $\eps=.001$ and the solution is continued 
in $\gamma$.
This run takes more than 1500 continuation steps.
For a detailed analysis of the solution behavior see 
\citename{JL:82} \citeyear{JL:82}.
\begin{table}[htbp]
\begin{center}
\begin{tabular}{| l | l |}
\hline
  COMMAND  & ACTION \\
\hline
%==============================================================================
  {\it mkdir spb} & create an empty work directory \\ 
  {\it cd spb} & change directory \\
  {\it @dm spb} & copy the demo files to the work directory \\
\hline
%==============================================================================
  {\it cp c.spb.1 c.spb} & get the first constants-file \\ 
  {\it @r spb} & 1st run; homotopy from $\lambda=0$ to $\lambda=1$ \\ 
  {\it @sv 1} & save output-files as {\tt b.1, s.1, d.1} \\ 
\hline
%==============================================================================
  {\it cp c.spb.2 c.spb} & constants changed : {\tt IRS, ICP(1), NTST, DS} \\ 
  {\it @r spb 1} & 2nd run; let $\eps$ tend to zero; restart from {\tt s.1} \\ 
  {\it @sv 2} & save the output-files as {\tt b.2, s.2, d.2} \\ 
\hline
%==============================================================================
  {\it cp c.spb.3 c.spb} & constants changed : {\tt IRS, ICP(1), RL0, ITNW, EPSL, EPSU, NUZR} \\ 
  {\it @r spb 2} & 3rd run; continuation in $\gamma$; $\eps=0.001$; restart from {\tt s.2} \\ 
  {\it @sv 3} & save the output-files as {\tt b.3, s.3, d.3} \\ 
\hline
%==============================================================================
\end{tabular}
\caption{Commands for running demo {\tt spb}.}
\label{tbl:demo_spb}
\end{center}
\end{table}

\newpage
%==============================================================================
%DEMO=ezp======================================================================
%==============================================================================
\section{ ezp : Complex Bifurcation in a BVP.} \label{sec:Demos_ezp}
This demo illustrates the computation of a solution family to
the the complex boundary value problem

\begin{equation} \begin{array}{cl}
  u_1 ' &= u_2  ,  \\
  u_2 ' &= -p_1  e^{u_1} , \\
\end{array} \end{equation}
with boundary conditions $ u_1(0)=0 , ~u_1(1)=0.$
Here $u_1$ and $u_2$ are allowed to be complex, 
while the parameter $p_1$ can only take real values.
In the real case, this is Bratu's equation, whose solution family 
contains a fold; see the demo {\tt exp}.
It is known 
(\citename{HeKe:90} \citeyear{HeKe:90}) that a simple quadratic fold gives rise to a pitch fork
bifurcation in the complex equation.
This bifurcation is located in the first computation below.
In the second and third run, both legs of the bifurcating solution family
are computed.
On it, both solution components $u_1$ and $u_2$ have nontrivial 
imaginary part.



\begin{table}[htbp]
\begin{center}
\begin{tabular}{| l | l |}
\hline
  COMMAND  & ACTION \\
\hline
%==============================================================================
  {\it mkdir ezp} & create an empty work directory \\ 
  {\it cd ezp} & change directory \\
  {\it @dm ezp} & copy the demo files to the work directory \\
\hline
%==============================================================================
  {\it cp c.ezp.1 c.ezp} & get the first constants-file \\ 
  {\it @r ezp} & 1st run; compute solution family containing fold \\ 
  {\it @sv ezp} & save output-files as {\tt b.ezp, s.ezp, d.ezp} \\ 
\hline
%==============================================================================
  {\it cp c.ezp.2 c.ezp} & constants changed : {\tt IRS, ISW} \\ 
  {\it @r ezp} & 2nd run; compute bifurcating complex solution family \\ 
  {\it @ap ezp} & append output-files to {\tt b.ezp, s.ezp, d.ezp} \\ 
\hline
%==============================================================================
  {\it cp c.ezp.3 c.ezp} & constant changed : {\tt DS} \\ 
  {\it @r ezp} & 3rd run; compute 2nd leg of bifurcating family \\ 
  {\it @ap ezp} & append output-files to {\tt b.ezp, s.ezp, d.ezp} \\
\hline
%==============================================================================
\end{tabular}
\caption{Commands for running demo {\tt ezp}.}
\label{tbl:demo_ezp}
\end{center}
\end{table}
%==============================================================================

%==============================================================================
%==============================================================================
\chapter{ {\cal AUTO} Demos : Parabolic PDEs.} \label{ch:Demos_PDE}
%==============================================================================
%==============================================================================

\newpage
%==============================================================================
%DEMO=pd1======================================================================
%==============================================================================
\section{ pd1 : Stationary States (1D Problem).} \label{sec:Demos_pd1}
This demo uses Euler's method to locate a stationary solution of
a nonlinear parabolic PDE, followed by continuation of this stationary
state in a free problem parameter. The equation is
 $$ {\partial u \over \partial t} 
  = D~{\partial^2 u \over \partial x^2} ~+~  p_1~ u ~( 1-u) , $$
on the space interval $[0,L]$, where $L=$~{\tt PAR(11)}~$=10$ is fixed throughout,
as is the diffusion constant $D=$~{\tt PAR(15)}~$=0.1$.
The boundary conditions are $u(0) = u(L) = 0$ for all time.

In the first run the continuation parameter is the independent time variable,
namely {\tt PAR(14)}, while $p_1=1$ is fixed.
The {\cal AUTO}-constants {\tt DS}, {\tt DSMIN}, and {\tt DSMAX} then control the step size
in space-time, here consisting of {\tt PAR(14)} and  $u(x)$.
Initial data are $u(x)=\sin(\pi x/L)$ at time zero.
Note that in the routine {\tt STPNT} the initial data must be scaled to 
the unit interval, and that the scaled derivative must also be provided; 
see the equations-file {\tt pv1.f}.
In the second run the continuation parameter is $p_1$.

Euler time integration is only first order accurate, so that
the time step must be sufficiently small to ensure correct results.
Indeed, this option has been added only as a convenience, and should 
generally be used only to locate stationary states.

\begin{table}[htbp]
\begin{center}
\begin{tabular}{| l | l |}
\hline
  COMMAND  & ACTION \\
\hline
%==============================================================================
  {\it mkdir pd1} & create an empty work directory \\ 
  {\it cd pd1} & change directory \\
  {\it @dm pd1} & copy the demo files to the work directory \\
\hline
%==============================================================================
  {\it cp c.pd1.1 c.pd1} & get the first constants-file \\ 
  {\it @r pd1} & time integration towards stationary state \\ 
  {\it @sv 1} & save output-files as {\tt b.1, s.1, d.1} \\ 
\hline
%==============================================================================
  {\it cp c.pd1.2 c.pd1} & constants changed : {\tt IPS, IRS, ICP}, etc.  \\ 
  {\it @r pd1 1} & continuation of stationary states; read restart data from {\tt s.1} \\ 
  {\it @sv 2} & save output-files as {\tt b.2, s.2, d.2} \\ 
\hline
%==============================================================================
\end{tabular}
\caption{Commands for running demo {\tt pd1}.}
\label{tbl:demo_pd1}
\end{center}
\end{table}

\newpage
%==============================================================================
%DEMO=pd2======================================================================
%==============================================================================
\section{ pd2 : Stationary States (2D Problem).} \label{sec:Demos_pd2}
This demo uses Euler's method to locate a stationary solution of
a nonlinear parabolic PDE, followed by continuation of this stationary
state in a free problem parameter. The equations are
\begin{equation} \begin{array}{cl}
  {\partial u_1 / \partial t} &= D_1~{\partial^2 u_1 / \partial x^2}
  ~+~  p_1~ u ~( 1-u) ~-~ u_1 u_2 , \\
  {\partial u_2 / \partial t} 
  &= D_2~{\partial^2 u_2 / \partial x^2} ~-~ u_2 ~+~ u_1 u_2 , \\
\end{array} \end{equation}
on the space interval $[0,L]$, where $L=$~{\tt PAR(11)}~$=1$ is fixed throughout,
as are the diffusion constants $D_1=$~{\tt PAR(15)}~$=1$ and $D_2=$~{\tt PAR(16)}~$=1$.
The boundary conditions are $u_1(0) = u_1(L) = 0$ and $u_2(0) = u_2(L) = 1$,
for all time.

In the first run the continuation parameter is the independent time variable,
namely {\tt PAR(14)}, while $p_1=12$ is fixed.
The {\cal AUTO}-constants {\tt DS}, {\tt DSMIN}, and {\tt DSMAX} then control the step size
in space-time, here consisting of {\tt PAR(14)} and $(u_1(x),u_2(x))$.
Initial data at time zero are $u_1(x)=\sin(\pi x/L)$ and $u_2(x)=1$.
Note that in the routine {\tt STPNT} the initial data must be scaled to 
the unit interval, and that the scaled derivatives must also be provided; 
see the equations-file {\tt pv2.f}.
In the second run the continuation parameter is $p_1$.
A branch point is located during this run.

Euler time integration is only first order accurate, so that
the time step must be sufficiently small to ensure correct results.
Indeed, this option has been added only as a convenience, and should 
generally be used only to locate stationary states.


\begin{table}[htbp]
\begin{center}
\begin{tabular}{| l | l |}
\hline
  COMMAND  & ACTION \\
\hline
%==============================================================================
  {\it mkdir pd2} & create an empty work directory \\ 
  {\it cd pd2} & change directory \\
  {\it @dm pd2} & copy the demo files to the work directory \\
\hline
%==============================================================================
  {\it cp c.pd2.1 c.pd2} & get the first constants-file \\ 
  {\it @r pd2} & time integration towards stationary state \\ 
  {\it @sv 1} & save output-files as {\tt b.1, s.1, d.1} \\ 
\hline
%==============================================================================
  {\it cp c.pd2.2 c.pd2} & constants changed : {\tt IPS, IRS, ICP}, etc.  \\ 
  {\it @r pd2 1} & continuation of stationary states; read restart data from {\tt s.1} \\ 
  {\it @sv 2} & save output-files as {\tt b.2, s.2, d.2} \\ 
\hline
%==============================================================================
\end{tabular}
\caption{Commands for running demo {\tt pd2}.}
\label{tbl:demo_pd2}
\end{center}
\end{table}

\newpage
%==============================================================================
%DEMO=wav======================================================================
%==============================================================================
\section{ wav : Periodic Waves.} \label{sec:Demos_wav}
This demo illustrates the computation of various periodic wave solutions
to a system of coupled parabolic partial differential equations
on the spatial interval $[0,1]$.
The equations, that model an enzyme catalyzed reaction 
(\citename{DoKe:86a} \citeyear{DoKe:86a}) are~:
\begin{equation} \begin{array}{cl}
 {\partial u_1 / \partial t}
  &=
  ~{\partial^{2} u_1 / \partial x^{2}}
  -p_1 \bigl[p_4 R(u_1,u_2) - (p_2 - u_1) \bigr] ,\\
 {\partial u_2 / \partial t}
  &=
  \beta {\partial^{2} u_2 / \partial x^{2}}
  -p_1 \bigl[p_4 R(u_1,u_2) - p_7 (p_3 - u_2) \bigr].\\
\end{array} \end{equation}
All equation parameters, except $p_3$, are fixed throughout.

\begin{table}[htbp]
\begin{center}
\begin{tabular}{| l | l |}
\hline
  COMMAND  & ACTION \\
\hline
%==============================================================================
  {\it mkdir wav} & create an empty work directory \\ 
  {\it cd wav} & change directory \\
  {\it @dm wav} & copy the demo files to the work directory \\
\hline
%==============================================================================
  {\it cp c.wav.1 c.wav} & get the first constants-file \\ 
  {\it @r wav} & 1st run; stationary solutions of the system without diffusion \\ 
  {\it @sv ode} & save output-files as {\tt b.ode, s.ode, d.ode} \\ 
\hline
%==============================================================================
  {\it cp c.wav.2 c.wav} & constants changed : {\tt IPS} \\ 
  {\it @r wav} & 2nd run; detect bifurcations to wave train solutions \\ 
  {\it @sv wav} & save output-files as {\tt b.wav, s.wav, d.wav} \\ 
\hline
%==============================================================================
  {\it cp c.wav.3 c.wav} & constants changed : {\tt IRS, IPS, NUZR, ILP} \\ 
  {\it @r wav} & 3rd run; wave train solutions of fixed wave speed \\ 
  {\it @ap wav} & append output-files to {\tt b.wav, s.wav, d.wav} \\ 
\hline
%==============================================================================
  {\it cp c.wav.4 c.wav} & constants changed : {\tt IRS, IPS, NMX, ICP, NUZR} \\ 
  {\it @r wav} & 4th run; wave train solutions of fixed wave length \\ 
  {\it @sv rng} & save output-files as {\tt b.rng, s.rng, d.rng} \\ 
\hline
%==============================================================================
  {\it cp c.wav.5 c.wav} & constants changed : {\tt IPS, NMX, NPR, ICP} \\ 
  {\it @r wav} & 5th run; time evolution computation \\ 
  {\it @sv tim} & save output-files as {\tt b.tim, s.tim, d.tim} \\ 
\hline
%==============================================================================
\end{tabular}
\caption{Commands for running demo {\tt wav}.}
\label{tbl:demo_wav}
\end{center}
\end{table}

\newpage
%==============================================================================
%DEMO=brc======================================================================
%==============================================================================
\section{ brc : Chebyshev Collocation in Space.} \label{sec:Demos_brc}
This demo illustrates the computation of stationary solutions and periodic
solutions to systems of parabolic PDEs in one space variable,
using Chebyshev collocation in space.
More precisely, the approximate solution is assumed of the form
$u(x,t) = \sum_{k=0}^{n+1} u_k(t) \ell_k(x)$.
Here $u_k(t)$ corresponds to $u(x_k,t)$ at the Chebyshev points
$\bigl\{ x_k \bigr\}_{k=1}^{n}$ with respect to the interval $[0,1]$.
The polynomials $\bigl\{ \ell_k(x) \bigr\}_{k=0}^{n+1}$ are the Lagrange
interpolating coefficients with respect to points 
$\bigl\{ x_k \bigr\}_{k=0}^{n+1}$, where $x_0=0$ and $x_{n+1}=1$.
The number of Chebyshev points in $[0,1]$,
as well as the number of equations in the PDE system,
can be set by the user in the file {\tt brc.inc}.

As an illustrative application we consider the Brusselator
(\citename{HoKnKu:87} \citeyear{HoKnKu:87})
\begin{equation} \begin{array}{cl}
  u_t &= {D_x / L^2} u_{xx} + u^2v - (B+1)u + A,  \\
  v_t &= {D_y / L^2} v_{xx} - u^2v + Bu,  \\
\end{array} \end{equation}
with boundary conditions $u(0,t)=u(1,t)=A$
and $v(0,t)=v(1,t)=B/A$.

Note that, given the non-adaptive spatial discretization,
the computational procedure here is not appropriate for
PDEs with solutions that rapidly vary in space, and care must
be taken to recognize spurious solutions and bifurcations.

\begin{table}[htbp]
\begin{center}
\begin{tabular}{| l | l |}
\hline
  COMMAND  & ACTION \\
\hline
%==============================================================================
  {\it mkdir brc} & create an empty work directory \\ 
  {\it cd brc} & change directory \\
  {\it @dm brc} & copy the demo files to the work directory \\
\hline
%==============================================================================
  {\it cp c.brc.1 c.brc} & get the first constants-file \\ 
  {\it @r brc} & compute the stationary solution family with Hopf bifurcations \\ 
  {\it @sv brc} & save output-files as {\tt b.brc, s.brc, d.brc} \\ 
\hline
%==============================================================================
  {\it cp c.brc.2 c.brc} & constants changed : {\tt IRS, IPS} \\ 
  {\it @r brc} & compute a family of periodic solutions from the first Hopf point \\ 
  {\it @ap brc} & append the output-files to {\tt b.brc, s.brc, d.brc} \\ 
\hline
%==============================================================================
  {\it cp c.brc.3 c.brc} & constants changed : {\tt IRS, ISW}
  \\ 
  {\it @r brc} & compute a solution family from a secondary periodic bifurcation \\ 
  {\it @ap brc} & append the output-files to {\tt b.brc, s.brc, d.brc} \\ 
\hline
%==============================================================================
\end{tabular}
\caption{Commands for running demo {\tt brc}.}
\label{tbl:demo_brc}
\end{center}
\end{table}


\newpage
%==============================================================================
%DEMO=brf======================================================================
%==============================================================================
\section{ brf : Finite Differences in Space.} \label{sec:Demos_brf}
This demo illustrates the computation of stationary solutions and periodic
solutions to systems of parabolic PDEs in one space variable.
A fourth order accurate finite difference approximation is used to
approximate the second order space derivatives. 
This reduces the PDE to an autonomous ODE of fixed dimension
which {\cal AUTO} is capable of treating.
The spatial mesh is uniform; the number of mesh intervals,
as well as the number of equations in the PDE system,
can be set by the user in the file {\tt brf.inc}.

As an illustrative application we consider the Brusselator
(\citename{HoKnKu:87} \citeyear{HoKnKu:87})
\begin{equation} \begin{array}{cl}
  u_t &= {D_x / L^2} u_{xx} + u^2v - (B+1)u + A,  \\
  v_t &= {D_y / L^2} v_{xx} - u^2v + Bu,  \\
\end{array} \end{equation}
with boundary conditions $u(0,t)=u(1,t)=A$
and $v(0,t)=v(1,t)=B/A$.

Note that, given the non-adaptive spatial discretization,
the computational procedure here is not appropriate for
PDEs with solutions that rapidly vary in space, and care must
be taken to recognize spurious solutions and bifurcations.


\begin{table}[htbp]
\begin{center}
\begin{tabular}{| l | l |}
\hline
  COMMAND  & ACTION \\
\hline
%==============================================================================
  {\it mkdir brf} & create an empty work directory \\ 
  {\it cd brf} & change directory \\
  {\it @dm brf} & copy the demo files to the work directory \\
\hline
%==============================================================================
  {\it cp c.brf.1 c.brf} & get the first constants-file \\ 
  {\it @r brf} & compute the stationary solution family with Hopf bifurcations \\ 
  {\it @sv brf} & save output-files as {\tt b.brf, s.brf, d.brf} \\ 
\hline
%==============================================================================
  {\it cp c.brf.2 c.brf} & constants changed : {\tt IRS, IPS} \\ 
  {\it @r brf} & compute a family of periodic solutions from the first Hopf point \\ 
  {\it @ap brf} & append the output-files to {\tt b.brf, s.brf, d.brf} \\ 
\hline
%==============================================================================
  {\it cp c.brf.3 c.brf} & constants changed : {\tt IRS, ISW} \\ 
  {\it @r brf} & compute a solution family from a secondary periodic bifurcation \\ 
  {\it @ap brf} & append the output-files to {\tt b.brf, s.brf, d.brf} \\ 
\hline
%==============================================================================
\end{tabular}
\caption{Commands for running demo {\tt brf}.}
\label{tbl:demo_brf}
\end{center}
\end{table}

\newpage
%==============================================================================
%DEMO=bru======================================================================
%==============================================================================
\section{ bru : Euler Time Integration (the Brusselator).} \label{sec:Demos_bru}
This demo illustrates the use of Euler's method for time integration
of a nonlinear parabolic PDE.
The example is the Brusselator
(\citename{HoKnKu:87} \citeyear{HoKnKu:87}), given by
\begin{equation} \begin{array}{cl}
  u_t &= {D_x / L^2} u_{xx} + u^2v - (B+1)u + A,  \\
  v_t &= {D_y / L^2} v_{xx} - u^2v + Bu,  \\
\end{array} \end{equation}
with boundary conditions $u(0,t)=u(1,t)=A$
and $v(0,t)=v(1,t)=B/A$. All parameters are given fixed values
for which a stable periodic solution is known to exist.

The continuation parameter is the independent time variable,
namely {\tt PAR(14)}.
The {\cal AUTO}-constants {\tt DS}, {\tt DSMIN}, and {\tt DSMAX}
then control the step size
in space-time, here consisting of {\tt PAR(14)} and $(u(x),v(x))$.
Initial data at time zero are 
$u(x)=A - 0.5 \sin(\pi x)$ and 
$v(x)=B/A + 0.7 \sin(\pi x)$.
Note that in the routine {\tt STPNT} the space derivatives of $u$ and $v$
must also be provided; 
see the equations-file {\tt bru.f}.

Euler time integration is only first order accurate, so that
the time step must be sufficiently small to ensure correct results.
This option has been added only as a convenience, and should 
generally be used only to locate stationary states.
Indeed, in the case of the asymptotic periodic state of this demo,
the number of required steps is very large and use of a better time 
integrator is advisable.


\begin{table}[htbp]
\begin{center}
\begin{tabular}{| l | l |}
\hline
  COMMAND  & ACTION \\
\hline
%==============================================================================
  {\it mkdir bru} & create an empty work directory \\ 
  {\it cd bru} & change directory \\
  {\it @dm bru} & copy the demo files to the work directory \\
\hline
%==============================================================================
  {\it cp c.bru.1 c.bru} & get the constants-file \\ 
  {\it @r bru} & time integration \\ 
  {\it @sv bru} & save output-files as {\tt b.bru, s.bru, d.bru} \\ 
\hline
%==============================================================================
\end{tabular}
\caption{Commands for running demo {\tt bru}.}
\label{tbl:demo_bru}
\end{center}
\end{table}

%==============================================================================
%==============================================================================
\chapter{ {\cal AUTO} Demos : Optimization.} \label{ch:Demos_Opt}
%==============================================================================
%==============================================================================

\newpage
%==============================================================================
%DEMO=opt======================================================================
%==============================================================================
\section{ opt : A Model Algebraic Optimization Problem.} \label{sec:Demos_opt}
This demo illustrates the method of successive continuation 
for constrained optimization problems 
 by applying it to the following
simple problem~:~
Find the
maximum sum of coordinates on the unit sphere in $R^{5}$.
Coordinate 1 is treated as the state variable.
Coordinates 2-5 are treated as control parameters.
For details on the successive continuation procedure
see  \citename{DoKeKe:91a} \citeyear{DoKeKe:91a},
\citename{DoKeKe:91b} \citeyear{DoKeKe:91b}.

\begin{table}[htbp]
\begin{center}
\begin{tabular}{| l | l |}
\hline
  COMMAND  & ACTION \\
\hline
%==============================================================================
  {\it mkdir opt} & create an empty work directory \\ 
  {\it cd opt} & change directory \\
  {\it @dm opt} & copy the demo files to the work directory \\
\hline
%==============================================================================
  {\it cp c.opt.1 c.opt} & get the first constants-file \\ 
  {\it @r opt} & one free equation parameter \\ 
  {\it @sv 1} & save output-files as {\tt b.1, s.1, d.1} \\ 
\hline
%==============================================================================
  {\it cp c.opt.2 c.opt} & constants changed : {\tt IRS} \\ 
  {\it @r opt 1} & two free equation parameters; read restart data from {\tt s.1} \\ 
  {\it @sv 2} & save output-files as {\tt b.2, s.2, d.2} \\ 
\hline
%============================================================================== 
  {\it cp c.opt.3 c.opt} & constants changed : {\tt IRS} \\ 
  {\it @r opt 2} & three free equation parameters; read restart data from {\tt s.2} \\ 
  {\it @sv 3} & save output-files as {\tt b.3, s.3, d.3} \\ 
\hline
%==============================================================================
  {\it cp c.opt.4 c.opt} & constants changed : {\tt IRS} \\ 
  {\it @r opt 3} & four free equation parameters; read restart data from {\tt s.3} \\ 
  {\it @sv 4} & save output-files as {\tt b.4, s.4, d.4} \\ 
\hline
%==============================================================================
\end{tabular}
\caption{Commands for running demo {\tt opt}.}
\label{tbl:demo_opt}
\end{center}
\end{table}

\newpage
%==============================================================================
%DEMO=ops======================================================================
%==============================================================================
\section{ ops : Optimization of Periodic Solutions.} \label{sec:Demos_ops}
This demo illustrates the method of successive continuation
for the optimization of periodic solutions.
For a detailed description of the basic method see
\citename{DoKeKe:91b} \citeyear{DoKeKe:91b}.
The illustrative system of autonomous ODEs, taken from 
\citename{Alej:91} \citeyear{Alej:91}, is
\begin{equation} \begin{array}{cl}
  x'(t) & = [-\lambda_4(x^3/3-x) + (z-x)/\lambda_2 - y]/\lambda_1, \\
  y'(t) &= x-\lambda_3, \\
  z'(t) &= -(z-x)/\lambda_2,
\end{array} \end{equation}
with objective functional
$$ \omega = 
  \int_0^{1} g(x,y,z;\lambda_1,\lambda_2,\lambda_3,\lambda_4) ~ dt, $$
where $g(x,y,z;\lambda_1,\lambda_2,\lambda_3,\lambda_4) \equiv \lambda_3$.
Thus, in this application, a one-parameter extremum of $g$ corresponds
to a fold with respect to the problem parameter $\lambda_3$, 
and multi-parameter extrema correspond to generalized folds.
Note that, in general, the objective functional is an integral along 
the periodic orbit, so that a variety of optimization problems
can be addressed.

For the case of periodic solutions, the extended optimality system
can be generated automatically, i.e., one need only define the vector field 
and the objective functional, as in done in the file {\tt ops.f}.
For reference purpose it is convenient here to write down
the full extended system in its general form~:

\begin{equation} \begin{array}{cl}
  &u'(t)  = T f \bigl( u(t),\lambda \bigr) ,
  \qquad T\in \R {\rm ~(period)},~ u(\cdot),f(\cdot,\cdot) \in \Rn, 
  ~ \lambda \in \R^{n_{\lambda}},  \\
  & \cr
  &w'(t)  = -Tf_u\bigl( u(t),\lambda \bigr)^{*} w(t) 
  + \kappa u_0'(t) 
  + \gamma g_u\bigl( u(t),\lambda \bigr)^{*}, 
  \qquad w(\cdot) \in \Rn,~ \kappa, \gamma \in \R, \\
  & \cr
  &u(1) - u(0) = 0, \qquad w(1) - w(0) = 0,  \\
  & \cr  
  &\int_{0}^{1} u(t)^{*} u_0'(t)~ dt = 0,  \\
  & \cr
  &\int_{0}^{1}  \omega - g\bigl(u(t),\lambda\bigr)  ~dt = 0,  \\
  & \cr
  &\int_0^{1}  w(t)^{*}w(t)
  + \kappa^2 + \gamma^{2} - \alpha ~ dt = 0, 
  \qquad \alpha \in \R,  \\ 
  & \cr
  &\int_0^{1}  f\bigl( u(t),\lambda \bigr)^{*}w(t) 
  - \gamma g_{T}\bigl( u(t),\lambda \bigr)
  - \tau_0  ~ dt= 0, \qquad \tau_0 \in \R,  \\
  & \cr
  &\int_0^{1}  T f_{\lambda_i}\bigl( u(t),\lambda \bigr)^{*}w(t)
  - \gamma g_{\lambda_i}\bigl( u(t),\lambda \bigr)
  - \tau_i  ~dt= 0, 
  \qquad \tau_i \in \R, \quad i=1, \cdots, n_{\lambda}.\\
\end{array} \end{equation}
Above  $u_0$ is a reference solution, namely, the previous solution along 
a solution family.  

\newpage
In the computations below, the two preliminary runs, with {\tt IPS}=1 and {\tt IPS}=2,
respectively, locate periodic solutions. 
The subsequent runs are with {\tt IPS}=15 and hence use the automatically
 generated extended system.

\begin{itemize}
\item[-] 
  {\it Run 1.}~ Locate a Hopf bifurcation. 
  The free system parameter is $\lambda_3$. 
\item[-]{\it Run 2.}~ 
  Compute a family of periodic solutions from the Hopf bifurcation.
\item[-]{\it Run 3.}~ 
  This run retraces part of the periodic solution family, 
  using the full optimality system,
  but with all adjoint variables, $w(\cdot), \kappa, \gamma$, 
  and hence $\alpha$, equal to zero.
  The optimality parameters $\tau_0$ and $\tau_3$ are zero throughout.
  An extremum of the objective functional with respect to $\lambda_3$
  is located.
  Such a point corresponds to a branch point of the extended system. 
  Given the choice of objective functional in this demo, 
  this extremum is also a fold with respect to $\lambda_3$.
\item[-]{\it Run 4.}~
  Branch switching at the above-found branch point yields nonzero
  values of the adjoint variables.
  Any point on the bifurcating family away from the branch point
  can serve as starting solution for the next run.
  In fact, the branch-switching can be viewed as generating
  a nonzero eigenvector in an eigenvalue-eigenvector relation.
  Apart from the adjoint variables, all other variables remain
  unchanged along the bifurcating family.
\item[-]{\it Run 5.}~ 
  The above-found starting solution is continued in two system parameters, 
  here $\lambda_3$ and $\lambda_2$; i.e., a two-parameter family 
  of extrema with respect to $\lambda_3$ is computed.
  Along this family the value of the optimality parameter $\tau_2$ 
  is monitored, i.e., the value of the functional that vanishes 
  at an extremum with respect to the system parameter $\lambda_2$.
  Such a zero of $\tau_2$ is, in fact, located, and hence an extremum 
  of the objective functional with respect to both $\lambda_2$ and 
  $\lambda_3$ has been found.
  Note that, in general, $\tau_i$ is the value of the
  functional that vanishes at an extremum with respect to the system
  parameter $\lambda_i$.
\item[-]{\it Run 6.}~ 
  In the final run, the above-found two-parameter extremum is continued
  in three system parameters, here $\lambda_1$, $\lambda_2$, 
  and $\lambda_3$, toward $\lambda_1=0$.
  Again, given the particular choice of objective functional,
  this final continuation has an alternate significance here~:
  it also represents a three-parameter family of transcritical
  secondary periodic bifurcations points.
\end{itemize}

Although not illustrated here, one can restart an ordinary
continuation of periodic solutions, using {\tt IPS}=2 or {\tt IPS}=3,
from a labeled solution point on a family computed with {\tt IPS}=15.

\newpage
The free scalar variables specified in the {\cal AUTO} constants-files
for Run~3 and Run~4 are shown in  Table~\ref{tbl:demo_ops_1}.

\begin{table}[htbp]
\begin{center}
\begin{tabular}{| l | r | r | r | r | r | r | r |}
\hline
  Index& 3 & 11 & 12 & 22  & -22 & -23 & -31 \\
\hline
  Variable& $\lambda_3$ & $T$ &  $\alpha$ & $\tau_2$  
  & $[\lambda_2]$ & $[\lambda_3]$ & $[T]$ \\
\hline
\end{tabular}
\caption{{\it Runs 3 and 4}~ (files {\tt c.ops.3} and {\tt c.ops.4}).}
\label{tbl:demo_ops_1}
\end{center}
\end{table}

The parameter $\alpha$, which is the norm of the adjoint variables,
becomes nonzero after branch switching in Run~4.
The negative indices (-22, -23, and -31) set the active optimality 
functionals, namely for $\lambda_2$, $\lambda_3$, and $T$, respectively,
with corresponding variables $\tau_2$, $\tau_3$, and $\tau_0$,
respectively.
These should be set in the first run with {\tt IPS}=15 and remain unchanged
in all subsequent runs.


\begin{table}[htbp]
\begin{center}
\begin{tabular}{| l | r | r | r | r | r | r | r |}
\hline
  Index& 3 & ~2 & 11 & 22  & -22 & -23 & -31 \\
\hline
  Variable& $\lambda_3$ & $\lambda_2$ & $T$ & $\tau_2$  
  & $[\lambda_2]$ & $[\lambda_3]$ & $[T]$ \\
\hline
\end{tabular}
\caption{{\it Run 5}~ (file {\tt c.ops.5}).}
\label{tbl:demo_ops_2}
\end{center}
\end{table}

In Run~5 the parameter $\alpha$, which has been replaced by $\lambda_2$,
remains fixed and nonzero.
The variable $\tau_2$ monitors the value of the optimality functional 
associated with $\lambda_2$.
The zero of $\tau_2$ located in this run signals an extremum  
with respect to $\lambda_2$.


\begin{table}[htbp]
\begin{center}
\begin{tabular}{| l | r | r | r | r | r | r | r |}
\hline
  Index& 3 & ~2 & ~1 & 11  & -22 & -23 & -31 \\
\hline
  Variable& $\lambda_3$ & $\lambda_2$ & $\lambda_1$ & $T$  
  & $[\lambda_2]$ & $[\lambda_3]$ & $[T]$ \\
\hline
\end{tabular}
\caption{{\it Run 6}~ (file {\tt c.ops.6}).}
\label{tbl:demo_ops_3}
\end{center}
\end{table}


In Run~6 $\tau_2$, which has been replaced by $\lambda_1$, remains zero.


Note that $\tau_0$ and $\tau_3$ are not used as variables in any
of the runs; in fact, their values remain zero throughout.
Also note that the optimality functionals corresponding to 
$\tau_0$ and $\tau_3$ (or, equivalently, to $T$ and $\lambda_3$) 
{\it are} active in all runs.
This set-up allows the detection of the extremum of the objective functional,
with $T$ and $\lambda_3$ as scalar equation parameters,
as a bifurcation in the third run.

The parameter $\lambda_4$, and its corresponding optimality variable $\tau_4$,
are not used in this demo.
Also, $\lambda_1$ is used in the last run only, and its corresponding 
optimality variable $\tau_1$ is never used.



\begin{table}[htbp]
\begin{center}
\begin{tabular}{| l | l |}
\hline
  COMMAND  & ACTION \\
\hline
%==============================================================================
  {\it mkdir ops} & create an empty work directory \\ 
  {\it cd ops} & change directory \\
  {\it @dm ops} & copy the demo files to the work directory \\
\hline
%==============================================================================
  {\it cp c.ops.1 c.ops} & get the first constants-file \\ 
  {\it @r ops} & locate a Hopf bifurcation \\ 
  {\it @sv 0} & save output-files as {\tt b.0, s.0, d.0} \\ 
\hline
%==============================================================================
  {\it cp c.ops.2 c.ops} & constants changed : {\tt IPS, IRS, NMX, NUZR} \\ 
  {\it @r ops  0} & compute a family of periodic solutions;  restart from {\tt s.0} \\ 
  {\it @ap 0} & append the output-files to {\tt b.0, s.0, d.0} \\ 
\hline
%==============================================================================
  {\it cp c.ops.3 c.ops} & constants changed : {\tt IPS, IRS, ICP}, $\cdots$ \\ 
  {\it @r ops 0} & locate a 1-parameter extremum as a bifurcation; restart from {\tt s.0} \\ 
  {\it @sv 1} & save the output-files as {\tt b.1, s.1, d.1} \\ 
\hline
%==============================================================================
  {\it cp c.ops.4 c.ops} & constants changed : {\tt IRS, ISP, ISW, NMX} \\ 
  {\it @r ops 1} & switch branches to generate optimality starting data; restart from {\tt s.1} \\ 
  {\it @ap 1} & append the output-files to {\tt b.1, s.1, d.1} \\ 
\hline
%==============================================================================
  {\it cp c.ops.5 c.ops} & constants changed : {\tt IRS, ISW, ICP, ISW}, $\cdots$ \\ 
  {\it @r ops 1} & compute 2-parameter family of 1-parameter extrema; restart from {\tt s.1} \\ 
  {\it @sv 2} & save the output-files as {\tt b.2, s.2, d.2} \\ 
\hline
%==============================================================================
  {\it cp c.ops.6 c.ops} & constants changed : {\tt IRS, ICP, EPSL, EPSU, NUZR} \\ 
  {\it @r ops 2} & compute 3-parameter family of 2-parameter extrema; restart from {\tt s.2} \\ 
  {\it @sv 3} & save the output-files as {\tt b.3, s.3, d.3} \\ 
\hline
%==============================================================================
\end{tabular}
\caption{Commands for running demo {\tt ops}.}
\label{tbl:demo_ops_4}
\end{center}
\end{table}

\newpage
%==============================================================================
%DEMO=obv======================================================================
%==============================================================================
\section{ obv : Optimization for a BVP.} \label{sec:Demos_obv}
This demo illustrates use of the method of successive continuation
for a  boundary value optimization problem.
A detailed description of the basic method, as well as a discussion
of the specific application considered here, is given in 
\citename{DoKeKe:91b} \citeyear{DoKeKe:91b}.
The required extended system is fully programmed here in the user-supplied
routines in {\tt obv.f}.
For the case of periodic solutions the optimality system can be generated
automatically; see the demo {\tt ops}.

Consider the system
\begin{equation} \begin{array}{cl}
  u_1'(t) & = u_2(t), \\
  u_2'(t) &=
  -\lambda_1 e^{p(u_1,\lambda_2,\lambda_3)},
\end{array} \end{equation}
where
$ p(u_1,\lambda_2,\lambda_3) \equiv
  u_1 + \lambda_2 u_1^{2} + \lambda_3 u_1^{4},$
with boundary conditions
\begin{equation} \begin{array}{cl}
  u_1(0) &= 0, \\
  u_1(1) &= 0. \\
\end{array} \end{equation}
The objective functional is
$$ \omega = \int_0^{1} (u_1(t)-1)^{2}~ dt
  +  {1\over{10}} \sum_{k=1}^{3} \lambda_{k}^{2}.  $$
The  successive continuation equations are given by
\begin{equation} \begin{array}{cl}
  u_1'(t) &= u_2(t), \\
  u_2'(t) &=
  -\lambda_1 e^{p(u_1,\lambda_2,\lambda_3)}, \\
  w_1'(t) &=
  \lambda_1 e^{p(u_1,\lambda_2,\lambda_3)} p_{u_1} w_2(t)
  + 2 \gamma(u_1(t)-1), \\
  w_2'(t) &= -w_1(t), \\
\end{array} \end{equation}
where
$$ p_{u_1} \equiv
  {{\partial p} \over {\partial u_1}} =
  1 + 2\lambda_2 u_1 + 4\lambda_3 u_1^{3},$$
with 
\begin{equation} \begin{array}{cl}
  u_1(0) = 0,\qquad  &w_1(0) - \beta_1 = 0,\qquad  w_2(0) = 0, \\
  u_1(1) = 0,\qquad  &w_1(1) + \beta_2 = 0,\qquad  w_2(1) = 0, \\\end{array} \end{equation}

$$ \int_0^{1} \bigl[ \omega - (u_1(t)-1)^{2}
  - {1\over{10}} \sum_{k=1}^{3} \lambda_{k}^{2} \bigr]~ dt = 0, $$

$$ \int_0^{1} \bigl[w_1^{2}(t) - \alpha_0 \bigr]~ dt = 0, $$
 
\begin{equation} \begin{array}{cl}
  &\int_0^{1} \bigl[
  -e^{p(u_1,\lambda_2,\lambda_3)} w_2(t)
  - {1\over 5}\gamma \lambda_1
  \bigr]~ dt = 0, \\
  &\int_0^{1} \bigl[
  -\lambda_1 e^{p(u_1,\lambda_2,\lambda_3)} u_1(t)^{2} w_2(t)
  - {1\over 5}\gamma \lambda_2
  - \tau_2  \bigr]~ dt = 0, \\
  &\int_0^{1} \bigl[
  -\lambda_1 e^{p(u_1,\lambda_2,\lambda_3)} u_1(t)^{4} w_2(t)
  - {1\over 5}\gamma \lambda_3
  - \tau_3 \bigr]~ dt = 0. \\\end{array} \end{equation}

In the first run the free equation parameter is $\lambda_1$.
All adjoint variables are zero.
Three extrema of the objective function are located.
These correspond to branch points and, in the second run,
branch switching is done at one of these.
Along the bifurcating family the adjoint variables become nonzero,
while state variables and $\lambda_1$ remain constant.
Any such non-trivial solution point can be used for continuation 
in two equation parameters, after fixing the $L_2$-norm of one of 
the adjoint variables. This is done in the third run.
Along the resulting family several two-parameter extrema are located 
by monotoring certain inner products.
One of these is further continued in three equation parameters in the final run,
where a three-parameter extremum is located.


\begin{table}[htbp]
\begin{center}
\begin{tabular}{| l | l |}
\hline
  COMMAND  & ACTION \\
\hline
%==============================================================================
  {\it mkdir obv} & create an empty work directory \\ 
  {\it cd obv} & change directory \\
  {\it @dm obv} & copy the demo files to the work directory \\
\hline
%==============================================================================
  {\it cp c.obv.1 c.obv} & get the first constants-file \\ 
  {\it @r obv} & locate 1-parameter extrema as branch points \\ 
  {\it @sv obv} & save output-files as {\tt b.obv, s.obv, d.obv} \\ 
\hline
%==============================================================================
  {\it cp c.obv.2 c.obv} & constants changed : {\tt IRS, ISW, NMX} \\ 
  {\it @r obv} & compute a few step on the first bifurcating family \\ 
  {\it @sv 1} & save the output-files as {\tt b.1, s.1, d.1} \\ 
\hline
%==============================================================================
  {\it cp c.obv.3 c.obv} & constants changed : {\tt IRS, ISW, NMX, ICP(3)} \\ 
  {\it @r obv  1} & locate 2-parameter extremum; restart from {\tt s.1} \\ 
  {\it @sv 2} & save the output-files as {\tt b.2, s.2, d.2} \\ 
\hline
%==============================================================================
  {\it cp c.obv.4 c.obv} & constants changed : {\tt IRS, ICP(4)} \\ 
  {\it @r obv  2} & locate 3-parameter extremum; restart from {\tt s.2} \\ 
  {\it @sv 3} & save the output-files as {\tt b.3, s.3, d.3} \\ 
\hline
%==============================================================================
\end{tabular}
\caption{Commands for running demo {\tt obv}.}
\label{tbl:demo_obv}
\end{center}
\end{table}

%==============================================================================
%==============================================================================
\chapter{ {\cal AUTO} Demos : Connecting orbits.} \label{ch:Demos_Heteroclinics}
%==============================================================================
%==============================================================================

\newpage
%==============================================================================
%DEMO=fsh======================================================================
%==============================================================================
\section{ fsh : A Saddle-Node Connection.} \label{sec:Demos_fsh}
This demo illustrates the computation of travelling wave front solutions
to the Fisher equation,
\begin{equation} \begin{array}{cl}
  & w_t = w_{xx} + f(w),
  \qquad -\infty < x < \infty,
  \quad  t > 0,  \\
  & f(w) \equiv w(1-w) .  \\
\end{array} \end{equation}
We look for solutions of the form $w(x,t)=u(x+ct)$, where
$c$ is the wave speed.
This gives the first order system
\begin{equation} \begin{array}{cl}
  &  u_1'(z)  = u_2(z),  \\
  &  u_2'(z)  = c u_2(z) - f\bigl(u_1(z)\bigr).  \\
\end{array} \end{equation}
Its fixed point $(0,0)$ has two positive eigenvalues when $c>2$.
The other fixed point, $(1,0)$, is a saddle point.
A family of orbits connecting the two fixed points
requires one free parameter; see 
\citename{FrDo:91} \citeyear{FrDo:91}.
Here we take this parameter to be the wave speed $c$.

In the first run a starting connecting orbit is computed 
by continuation in the period $T$.
This procedure can be used generally for time integration of an ODE with {\cal AUTO}.
Starting data in {\tt STPNT} correspond to a point on the approximate stable manifold
of $(1,0)$, with $T$ small.
In this demo the ``free'' end point of the orbit necessary approaches the
unstable fixed point $(0,0)$.
A computed orbit with sufficiently large $T$ is then chosen as restart orbit
in the second run, where, typically, one replaces $T$ by $c$ as continuation
parameter.
However, in the second run below, we also add a phase condition, 
and both $c$ and $T$ remain free.



\begin{table}[htbp]
\begin{center}
\begin{tabular}{| l | l |}
\hline
  COMMAND  & ACTION \\
\hline
%==============================================================================
  {\it mkdir fsh} & create an empty work directory \\ 
  {\it cd fsh} & change directory \\
  {\it @dm fsh} & copy the demo files to the work directory \\
\hline
%==============================================================================
  {\it cp c.fsh.1 c.fsh} & get the first constants-file \\ 
  {\it @r fsh} & continuation in the period $T$, with $c$ fixed; no phase condition \\ 
  {\it @sv 0} & save output-files as {\tt b.0, s.0, d.0} \\ 
\hline
%==============================================================================
  {\it cp c.fsh.2 c.fsh} & constants changed : {\tt IRS, ICP, NINT, DS} \\ 
  {\it @r fsh 0} & continuation in $c$ and $T$, with active phase condition \\ 
  {\it @sv fsh} & save output-files as {\tt b.fsh, s.fsh, d.fsh} \\ 
\hline
%==============================================================================
\end{tabular}
\caption{Commands for running demo {\tt fsh}.}
\label{tbl:demo_fsh}
\end{center}
\end{table}

\newpage
%==============================================================================
%DEMO=nag======================================================================
%==============================================================================
\section{ nag : A Saddle-Saddle Connection.} \label{sec:Demos_nag}
This demo illustrates the computation of traveling wave front solutions
to Nagumo's equation,
\begin{equation} \begin{array}{cl}
  & w_t = w_{xx} + f(w,a),
  \qquad -\infty < x < \infty,
  \quad  t > 0,  \\
  & f(w,a) \equiv w(1-w)(w-a), \qquad 0<a<1.  \\
\end{array} \end{equation}
We look for solutions of the form $w(x,t)=u(x+ct)$, where
$c$ is the wave speed.
This gives the first order system
\begin{equation} \begin{array}{cl}
  &  u_1'(z)  = u_2(z),  \\
  &  u_2'(z)  = c u_2(z) - f\bigl(u_1(z),a\bigr),  \\
\end{array} \end{equation}
where $z=x+ct$, and $' = d/dz$.
If $a=1/2$ and $c=0$ then there are two analytically known
heteroclinic connections, one of which is given by
$$ u_1(z) = {
  {e^{{1 \over 2} \sqrt{2} z}}
  \over
  {1 + e^{{1 \over 2} \sqrt{2} z}}  },
  \qquad  u_2(z) = u_1'(z),  \qquad  -\infty < z < \infty.
  $$
The second heteroclinic connection is obtained by reflecting the
phase plane representation of the first with respect to the
$u_1$-axis.
In fact, the two connections together constitute a heteroclinic cycle.
One of the exact solutions is used below as starting orbit.
To start from the second exact solution, change SIGN=-1 in the  
routine {\tt STPNT} in {\tt nag.f} and repeat the computations below;
see also
\citename{FrDo:91} \citeyear{FrDo:91}.

\begin{table}[htbp]
\begin{center}
\begin{tabular}{| l | l |}
\hline
  COMMAND  & ACTION \\
\hline
%==============================================================================
  {\it mkdir nag} & create an empty work directory \\ 
  {\it cd nag} & change directory \\
  {\it @dm nag} & copy the demo files to the work directory \\
\hline
%==============================================================================
  {\it cp c.nag.1 c.nag} & get the first constants-file \\ 
  {\it @r nag} & compute part of first family of heteroclinic orbits \\ 
  {\it @sv nag} & save output-files as {\tt b.nag, s.nag, d.nag} \\ 
\hline
%==============================================================================
  {\it cp c.nag.2 c.nag} & constants changed : {\tt DS} \\ 
  {\it @r nag} & compute first family in opposite direction \\ 
  {\it @ap nag} & append output-files to {\tt b.nag, s.nag, d.nag} \\ 
\hline
%==============================================================================
\end{tabular}
\caption{Commands for running demo {\tt nag}.}
\label{tbl:demo_nag}
\end{center}
\end{table}

\newpage
%==============================================================================
%DEMO=stw======================================================================
%==============================================================================
\section{ stw : Continuation of Sharp Traveling Waves.} \label{sec:Demos_stw}
This demo illustrates the computation of sharp traveling wave front solutions
to nonlinear diffusion problems of the form
$$ w_t = A(w) w_{xx} + B(w) w_x^{2} + C(w),  $$
with
$A(w) = a_1 w + a_2 w^{2}$,
$B(w) = b_0 + b_1 w + b_2 w^{2}$,
and
$C(w) = c_0 + c_1 w + c_2 w^{2}$.
Such equations can have {\it sharp traveling wave fronts} as solutions, i.e., solutions of the form
$w(x,t)=u(x+ct)$ for which there is a $z_0$ such that
$u(z)=0$ for $z \ge z_0$,
$u(z) \not= 0$ for $z < z_0$, and
$u(z) \rightarrow constant$ as $z \rightarrow -\infty$.
These solutions are actually generalized solutions, since they need
not be differentiable at $z_0$.

Specifically, in this demo a homotopy path will be computed 
from an analytically known exact sharp traveling wave solution of
$$ w_t = 2w w_{xx} + 2 w_x^{2} + w(1-w),  \leqno(1) $$
to a corresponding sharp traveling wave of
$$ w_t = (2w+w^{2}) w_{xx} + w w_x^{2} + w(1-w). \leqno(2) $$
This problem is also considered in
\citename{DoKeKe:91b} \citeyear{DoKeKe:91b}.
For these two special cases the functions $A,B,C$ are defined
by the coefficients in Table~\ref{tbl:demo_stw_1}.

\begin{table}[htbp]
\begin{center}
\begin{tabular}{| l | l | l | l | l | l | l | l | l |}
\hline
         & $a_1$ & $a_2$ & $b_0$ &$b_1$ &$b_2$ &$c_0$ &$c_1$ &$c_2$ \\
\hline
Case (1) &  2    &   0   &   2   &  0   &  0   &  0   &  1   &  -1 \\
\hline
Case (2) &  2    &   1   &   0   &  1   &  0   &  0   &  1   &  -1 \\
\hline
\end{tabular}
\caption{Problem coefficients in demo {\tt stw}.}
\label{tbl:demo_stw_1}
\end{center}
\end{table}

With $w(x,t)=u(x+ct)$, $z=x+ct$, one obtains the reduced system
\begin{equation} \begin{array}{cl}
  & u_1'(z) = u_2,  \\
  & u_2'(z) = \bigl[c u_2 - B(u_1) u_2^{2} - C(u_1) \bigr]/A(u_1). \\
\end{array} \end{equation}
To remove the singularity when $u_1=0$, we apply a
nonlinear transformation of the independent variable 
(see \citename{Ar:80} \citeyear{Ar:80}), viz.,
${d / d \tilde z} = A(u_1) {d / dz}$,
which changes the above equation into
\begin{equation} \begin{array}{cl}
  & u_1'(\tilde z) = A(u_1) u_2,  \\
  & u_2'(\tilde z) = c u_2 - B(u_1) u_2^{2} - C(u_1). \\
\end{array} \end{equation}
Sharp traveling waves then correspond to heteroclinic connections
in this transformed system.

\newpage 
Finally, we  map $ [0,T] \rightarrow [0,1] $
by the transformation $\xi = \tilde z / T$.
With this scaling of the independent variable, the reduced system
becomes
\begin{equation} \begin{array}{cl}
  & u_1'(\xi) = T A(u_1) u_2,  \\
  & u_2'(\xi) = T \bigl[ c u_2 - B(u_1) u_2^{2} - C(u_1)\bigr]. \\
\end{array} \end{equation}
For Case~1 this equation has a known exact solution, namely,
$$ u(\xi) = { 1 \over 1 + exp(T\xi) }, \qquad
  v(\xi) = { -{1 \over 2}  \over 1 + exp(-T\xi) }. $$
This solution has wave speed $c=1$.
In the limit as $T \rightarrow \infty$ its phase plane trajectory
connects the stationary points $(1,0)$ and $(0,-{1 \over 2})$.
 
The sharp traveling wave in Case~2
can now be obtained using the following homotopy.
Let
$(a_1,a_2,b_0,b_1,b_2) =
  (1-\lambda) (2,0,2,0,0) + \lambda (2,1,0,1,0)$.
Then as $\lambda$ varies continuously from 0 to 1, the parameters
$(a_1,a_2,b_0,b_1,b_2)$
vary continously from the values for Case~1
  to the values for Case~2.


\begin{table}[htbp]
\begin{center}
\begin{tabular}{| l | l |}
\hline
  COMMAND  & ACTION \\
\hline
%==============================================================================
  {\it mkdir stw} & create an empty work directory \\ 
  {\it cd stw} & change directory \\
  {\it @dm stw} & copy the demo files to the work directory \\
\hline
%==============================================================================
  {\it cp c.stw.1 c.stw} & get the constants-file \\ 
  {\it @r stw} & continuation of the sharp traveling wave \\ 
  {\it @sv stw} & save output-files as {\tt b.stw, s.stw, d.stw} \\ 
\hline
%==============================================================================
\end{tabular}
\caption{Commands for running demo {\tt stw}.}
\label{tbl:demo_stw_2}
\end{center}
\end{table}


%==============================================================================
%==============================================================================
\chapter{ {\cal AUTO} Demos : Miscellaneous.} \label{ch:Demos_Misc}
%==============================================================================
%==============================================================================

\newpage
%==============================================================================
%DEMO=pvl======================================================================
%==============================================================================
\section{ pvl : Use of the Routine {\tt PVLS}.} \label{sec:Demos_pvl}

Consider Bratu's equation
\begin{equation} \begin{array}{cl}
  u_1 ' &= u_2  ,  \\
  u_2 ' &= -p_1  e^{u_1} , \\ 
\end{array} \end{equation}
with boundary conditions $ u_1(0)=0$, $u_1(1)=0.$
As in demo {\tt exp}, a solution curve requires one free parameter;
here $p_1$.

Note that additional parameters are specified in the user-supplied routine 
{\tt PVLS} in file {\tt pvls.f}, namely,
$p_2$ (the $L_2$-norm of $u_1$),
$p_3$ (the minimum of $u_2$ on the space-interval $[0,1]$~),
$p_4$ (the boundary value $u_2(0)$~).
These additional parameters should be considered as ``solution measures''
for output purposes; they should not be treated as true
continuation parameters.

Note also that four free parameters are specified in the {\cal AUTO}-constants file 
{\tt c.pvl.1}, namely, $p_1$, $p_2$, $p_3$, and $p_4$.
The first one in this list, $p_1$, is the true continuation parameter. 
The parameters $p_2$, $p_3$, and $p_4$ are {\it overspecified}
so that their values will appear in the output.
However, 
{\it it is essential that the true continuation parameter appear first.}
For example, it would be an error to specify the parameters
in the following order~: $p_2$, $p_1$, $p_3$, $p_4$.

In general, true continuation parameters must appear first in the
parameter-specification in the {\cal AUTO} constants-file.
Overspecified parameters will be printed, and can be
defined in {\tt PVLS}, but they are not part of the intrinsic continuation
procedure.

As this demo also illustrates (see the {\tt UZR} values in {\tt c.pvl.1}),
labeled solutions can also be output at selected values 
of the overspecified parameters.

\begin{table}[htbp]
\begin{center}
\begin{tabular}{| l | l |}
\hline
  COMMAND  & ACTION \\
\hline
%==============================================================================
  {\it mkdir pvl} & create an empty work directory \\ 
  {\it cd pvl} & change directory \\
  {\it @dm pvl} & copy the demo files to the work directory \\
\hline
%==============================================================================
  {\it cp c.pvl.1 c.pvl} & get the constants-file \\ 
  {\it @r pvl} & compute a solution family \\ 
  {\it @sv pvl} & save output-files as {\tt b.pvl, s.pvl, d.pvl} \\ 
\hline
%==============================================================================
\end{tabular}
\caption{Commands for running demo {\tt pvl}.}
\label{tbl:demo_pvl}
\end{center}
\end{table}

\newpage
%==============================================================================
%DEMO=ext======================================================================
%==============================================================================
\section{ ext : Spurious Solutions to BVP.} \label{sec:Demos_ext}

This demo illustrates the computation of spurious solutions
to the boundary value problem
\begin{equation} \begin{array}{cl}
& u_1' - u_2 = 0 , \\
& u_2' + \lambda^2 \pi^2 \sin( u_1 + u_1^2 + u_1^3 ) = 0,
  \qquad t \in [0,1], \\ 
& u_1(0) = 0, \quad u_1(1) = 0. \\
\end{array} \end{equation}
Here the differential equation is discretized using a fixed uniform mesh.
This results in spurious solutions that disappear when an adaptive mesh is used.
See the {\cal AUTO}-constant {\tt IAD} in Section~\ref{sec:Discretization_constants}.
This example is also considered in
\citename{BeDo:81} \citeyear{BeDo:81}
and
\citename{DoKeKe:91b} \citeyear{DoKeKe:91b}.

\begin{table}[htbp]
\begin{center}
\begin{tabular}{| l | l |}
\hline
  COMMAND  & ACTION \\
\hline
%==============================================================================
  {\it mkdir ext} & create an empty work directory \\ 
  {\it cd ext} & change directory \\
  {\it @dm ext} & copy the demo files to the work directory \\
\hline
%==============================================================================
  {\it cp c.ext.1 c.ext} & get the first constants-file \\ 
  {\it @r ext} & detect bifurcations from the trivial solution family \\ 
  {\it @sv ext} & save output-files as {\tt b.ext, s.ext, d.ext} \\ 
\hline
%==============================================================================
  {\it cp c.ext.2 c.ext} & constants changed : {\tt IRS, ISW, NUZR} \\ 
  {\it @r ext} & compute a bifurcating family containing spurious bifurcations \\ 
  {\it @ap ext} & append output-files to {\tt b.ext, s.ext, d.ext} \\ 
\hline
%==============================================================================
\end{tabular}
\caption{Commands for running demo {\tt ext}.}
\label{tbl:demo_ext}
\end{center}
\end{table}

\newpage
%==============================================================================
%DEMO=tim======================================================================
%==============================================================================
\section{ tim : A Test Problem for Timing {\cal AUTO}.} \label{sec:Demos_tim}
This demo is a boundary value problem with variable dimension {\tt NDIM}. 
It can be used to time the performance of {\cal AUTO} 
for various choices of {\tt NDIM} (which must be even), {\tt NTST}, and {\tt NCOL}.
The equations are
\begin{equation} \begin{array}{cl}
  u_i ' &= u_i  ,  \\
  v_i ' &= -p_1 ~  e(u_i) , \\
\end{array} \end{equation}
$i=1,\cdots$,{\tt NDIM}/2,
with boundary conditions $ u_i(0)=0$, $u_i(1)=0.$
Here 
$$ e(u) = \sum_{k=0}^{n} ~ {u^k \over k!} ~ , $$
with $n=25$.
The computation requires 10 full $LU$-decompositions of the linearized system
that arises from Newton's method for solving the collocation equations.
The commands for running the timing problem for a particular choice 
of {\tt NDIM}, {\tt NTST}, and {\tt NCOL} are given below.
(Note that if {\tt NDIM} is changed then {\tt NBC} must be changed accordingly.)

\begin{table}[htbp]
\begin{center}
\begin{tabular}{| l | l |}
\hline
  COMMAND  & ACTION \\
\hline
%==============================================================================
  {\it mkdir tim} & create an empty work directory \\ 
  {\it cd tim} & change directory \\
  {\it @dm tim} & copy the demo files to the work directory \\
\hline
%==============================================================================
  {\it cp c.tim.1 c.tim} & get the first constants-file \\ 
  {\it @r tim} & Timing run \\ 
  {\it @sv tim} & save output-files as {\tt b.tim, s.tim, d.tim} \\ 
\hline
%==============================================================================
\end{tabular}
\caption{Commands for running demo {\tt tim}.}
\label{tbl:demo_tim}
\end{center}
\end{table}


%==============================================================================
%==============================================================================
\chapter{ {\cal HomCont}.} \label{ch:HomCont}
%==============================================================================
%==============================================================================
\section{ Introduction.} \label{sec:HomCont_Intro}
{\cal HomCont} is a collection of routines for the continuation 
of homoclinic solutions to ODEs in two or more parameters.
The accurate detection and multi-parameter continuation of certain
codimension-two singularities is allowed for, including all known
cases that involve a unique homoclinic orbit at the singular point.
Homoclinic connections to hyperbolic and non-hyperbolic equilibria are 
allowed as are certain heteroclinic orbits. 
Homoclinic orbits in reversible systems can also be computed.
The theory behind the methods used is
explained in \citeasnoun{ChKu:94}, \citeasnoun{BaCh:94},
\citename{Sa:95} \citeyear{Sa:95,Sa:95a}, \citeasnoun{ChKuSa:95} and
references therein.  The final cited paper contains a concise
description of the present version. 

The current implementation of {\cal HomCont} must be considered as experimental,
and updates are anticipated.
The {\cal HomCont} routines are in the file {\tt auto/07/src/autlib5.f}. 
Expert users wishing to modify the routines may look there.
Note also that at present, {\cal HomCont} can be run only in 
{\cal AUTO} Command Mode and not with the GUI. 


\section{{\cal HomCont} Files and Routines.} \label{sec:HomCont_files}

In order to run {\cal HomCont} one must prepare an equations file {\tt xxx.f}, 
where {\tt xxx} is the name of the example, 
and two constants-files {\tt c.xxx} and {\tt s.xxx}.
The first two of these files are in the standard {\cal AUTO} format, 
whereas the {\tt s.xxx} file
contains constants that are specific to homoclinic continuation.
The choice {\tt IPS}=9 in {\tt c.xxx} specifies the problem as
being homoclinic continuation, in which case {\tt s.xxx} is required.

The equation-file {\tt kpr.f} serves as a sample for new equation
files. It contains the Fortran routines 
{\tt FUNC}, {\tt STPNT}, {\tt PVLS}, {\tt BCND}, {\tt ICND} 
and {\tt FOPT}. The final three are
dummy routines which are never needed for homoclinic continuation.
Note a minor difference in {\tt STPNT} and {\tt PVLS} with other 
{\cal AUTO} equation-files, in that the common block 
{\tt /BLHOM/} is required.

The constants-file {\tt c.xxx} is identical in format to other
{\cal AUTO} constants-files. Note that the values of the constants
{\tt NBC} and {\tt NINT} are irrelevant, as these are set
automatically by the choice {\tt IPS}=9. Also, the choice {\tt JAC=1}
is strongly recommended, because the Jacobian is used extensively for
calculating the linearization at the equilibria and hence for
evaluating boundary conditions and certain test functions. However,
note that {\tt JAC=1} does not necessarily mean that {\cal auto} will
use the analytically specified Jacobian for continuation.

\section{ {\cal HomCont}-Constants.} \label{sec:HomCont_Constants}
An example for the additional file {\tt s.xxx} is listed below:
\begin{verbatim}
          1 2 1 1 1    NUNSTAB,NSTAB,IEQUIB,ITWIST,ISTART
          0            NREV,(/,I,IREV(I)),I=1,NREV)
          1            NFIXED,(/,I,IFIXED(I)),I=1,NFIXED)
            13
          1            NPSI,(/,I,IPSI(I)),I=1,NPSI)
            9 10 13
\end{verbatim}
The constants specified in {\tt s.xxx} have the following meaning. 

\subsection{\tt NUNSTAB}  \label{sec:NUNSTAB}

Number of unstable eigenvalues of the left-hand equilibrium (the equilibrium 
approached by the orbit as $t \to -\infty$).


\subsection{\tt NSTAB}  \label{sec:NSTAB}
Number of stable eigenvalues of the right-hand equilibrium (the equilibrium
approached by the orbit as $t \to +\infty$).

\subsection{\tt IEQUIB}  \label{sec:IEQUIB}
\begin{itemize}
\item[-] {\tt IEQUIB=0}~: 
Homoclinic orbits to hyperbolic equilibria;  
the equilibrium is specified explicitly in {\tt PVLS} and stored in
{\tt PAR(11+I)}, {\tt I=1,NDIM}.
\item[-] {\tt IEQUIB=1}~: 
Homoclinic orbits to hyperbolic equilibria;  
the equilibrium is solved for during continuation. Initial values for
the equilibrium are stored in {\tt PAR(11+I)}, {\tt I=1,NDIM} in {\tt STPNT}.
\item[-] {\tt IEQUIB=2}~: 
Homoclinic orbits to a saddle-node; initial values for
the equilibrium are stored in {\tt PAR(11+I)}, {\tt I=1,NDIM} in {\tt STPNT}.
\item[-] {\tt IEQUIB=-1}~: 
Heteroclinic orbits to hyperbolic equilibria;
the equilibria are specified explicitly in {\tt PVLS} and stored in
{\tt PAR(11+I)},  
{\tt I=1,NDIM} (left-hand equilibrium) and {\tt PAR(11+I)}, 
{\tt I=NDIM+1,2*NDIM} (right-hand equilibrium). 
\item[-] {\tt IEQUIB=-2}~: 
Heteroclinic orbits to hyperbolic equilibria;
the equilibria are solved for during continuation. Initial values are
specified in {\tt STPNT} and stored in {\tt PAR(11+I)}, {\tt I=1,NDIM} (left-hand equilibrium), 
{\tt PAR(11+I)}, {\tt I=NDIM+1,2*NDIM} (right-hand equilibrium).
\end{itemize}

\subsection{\tt ITWIST}  \label{sec:ITWIST}
\begin{itemize}
\item[-] {\tt ITWIST=0}~: 
the orientation of the homoclinic orbit is not computed.
\item[-] {\tt ITWIST=1}~: 
the orientation of the homoclinic orbit is computed. For this purpose, the
adjoint variational equation is solved for the unique bounded
solution. If {\tt IRS = 0}, an initial solution to the adjoint equation
must be specified as well. However, if {\tt IRS>0} and {\tt ITWIST} 
has just been increased from zero, then {\cal AUTO} will
automatically generate the initial solution to the adjoint. 
In this case, a dummy Newton-step should be performed, see 
Section~\ref{sec:Starting_strategies} for more details.
\end{itemize}

\subsection{\tt ISTART}  \label{sec:ISTART}
\begin{itemize}
\item[-] {\tt ISTART=1}~:  
This option is no obsolete in the current version. 
It may be used as a flag that a solution is to be
restarted from a previously computed point or
from numerical data converted into {\cal AUTO} format using {\tt @fc}.
In this case {\tt IRS>0}.
%If {\tt IRS=0} then starting data are read from the file 
%{\tt fort.4} (copied from {\tt xxx.dat} in the examples).
%These data must be {\tt t,U} in multi-column format at each 
%point with {\tt t} in the interval {\tt [0,1]}. If {\tt IRS}$\neq${\tt 0},
%this value of {\tt ISTART} should be used to read starting data from
%a previously-obtained output point from {\cal AUTO}.
\item[-] {\tt ISTART=2}~: 
If {\tt IRS=0}, an explicit solution must be specified in the
routine {\tt STPNT} in the usual format. 
\item[-] {\tt ISTART=3}~: 
The ``homotopy" approach is used for starting, see 
Section~\ref{sec:Starting_strategies} 
for more details. Note that this is not available with the choice 
{\tt IEQUIB=2}.
\end{itemize}

\subsection{\tt NREV, IREV}  \label{sec:IREV}
If {\tt NREV=1} then it is assumed that
the system is reversible under the transformation 
$t \to -t$ and $U(i) \to -U(i)$ for all $i$ with 
{\tt IREV(i)>0}. Then only half the homoclinic solution is
solved for with right-hand boundary conditions specifying
that the solution is symmetric under the reversibility
(see \citeasnoun{ChSp:93}). The number of free parameters
is then reduced by one. Otherwise {\tt IREV=0}.

\subsection{\tt NFIXED, IFIXED}  \label{sec:IFIXED}
Number and labels of test functions that are held fixed. 
E.g., with {\tt NFIXED=1} one can compute a locus in
one extra parameter of a singularity defined by 
test function {\tt PSI(IFIXED(1))=0}.

\subsection{\tt NPSI, IPSI}  \label{sec:IPSI}
Number and labels of activated test functions for detecting homoclinic
bifurcations, see Section~\ref{sec:HomCont_Test_functions} 
for a list. If a test function is activated then the
corresponding parameter ({\tt IPSI(I)+20}) 
must be added to the list of continuation parameters {\tt NICP,(ICP(I),I=1 NICP)}
and zero of this parameter added to the list of user-defined
output points {\tt NUZR,} {\tt (/,I,PAR(I)),I=1, NUZR} in {\tt c.xxx}.

\section{ Restrictions on {\cal HomCont} Constants.}
Note that certain combinations of these constants are not allowed
in the present implementation. In particular,
\begin{itemize}
\item[-] 
The computation of orientation {\tt ITWIST=1} is not
implemented for {\tt IEQUIB<0} (heteroclinic orbits), 
{\tt IEQUIB=2} (saddle-node homoclinics),
{\tt IREV=1} (reversible systems), {\tt ISTART=3} (homotopy
method for starting), or if the equilibrium contains complex
eigenvalues in its linearization.  
\item[-] The homotopy method {\tt ISTART=3} is not fully implemented
for heteroclinic connections {\tt IEQUIB<0}, saddle-node homoclinic
orbits {\tt IEQUIB=2} or reversible systems {\tt IREV=1}.
\item[-] Certain test functions are not valid for certain forms
of continuation 
(see Section~\ref{sec:HomCont_Test_functions} below); 
for example
{\tt PSI(13)} and {\tt PSI(14)} only make sense if 
{\tt ITWIST=1} and {\tt PSI(15)} and {\tt PSI(16)} only apply
to {\tt IEQUIB=2}.
\end{itemize}

\section{ Restrictions on the Use of {\tt PAR}.}
The parameters {\tt PAR(1)} -- {\tt PAR(9)} can be used freely by
the user. The other parameters are used as follows~:

\begin{itemize}

\item[-] {\tt PAR(11)}~: 
The value of {\tt PAR(11)} equals the length of the time interval over
which a homoclinic solution is computed. Also referred to as ``period''.
This must be specified in {\tt STPNT}.

\item[-] {\tt PAR(10)}~: 
If {\tt ITWIST=1} then {\tt PAR(10)} is used internally as a
dummy parameter so that the adjoint equation is well-posed.

\item[-] {\tt PAR(12)-PAR(20)}~:
These are used for specifying the 
equilibria and (if {\tt ISTART=3}) the artificial parameters of
the homotopy method (see Section~\ref{sec:Starting_strategies} below).

\item[-] {\tt PAR(21)-PAR(36)}~: 
These parameters are used for storing the test functions 
(see Section~\ref{sec:HomCont_Test_functions}).
\end{itemize}

The output is in an identical format to {\cal AUTO} except that
additional information at each computed point is written 
in {\tt fort.9}. This information comprises the eigenvalues of
the (left-hand) equilibrium, the values of each activated test
function and, if {\tt ITWIST=1}, 
whether the saddle homoclinic loop is orientable
or not.
Note that the statement about orientability is only meaningful if the
leading eigenvalues are not complex and the homoclinic solution is not
in a flip configuration, that is, none of the test functions 
$\psi_i$ for $i=11,12,13,14$ is zero (or close to zero), 
see Section~\ref{sec:HomCont_Test_functions}.
 Finally, the values of the {\tt NPSI} activated test functions are written. 

\section{ Test Functions.} \label{sec:HomCont_Test_functions}
Codimension-two homoclinic orbits are detected along families of codim
1 homoclinics by locating zeroes of certain test functions
$\psi_i$. The test functions that are ``switched on'' during any
continuation are given by the choice of the labels $i$, and are
specified by the parameters {\tt NPSI,(/,I,IPSI(I)),I=1,NPSI)} in {\tt
s.xxx}.  Here {\tt NPSI} gives the number of activated test functions
and {\tt IPSI(1),$\ldots$,IPSI(NPSI)} give the labels of
the test functions (numbers between 1 and 16). A zero of
each labeled test function defines a certain codimension-two 
homoclinic singularity, specified as follows.
The notation used for eigenvalues is the same as that in
\citeasnoun{ChKu:94} or \citeasnoun{ChKuSa:95}. 

\begin{itemize}
\item[-] {\tt i=1}~: 
Resonant eigenvalues (neutral saddle); $\mu_1=-\lambda_1$.
\item[-] {\tt i=2}~: 
Double real leading stable eigenvalues (saddle to saddle-focus
transition); $\mu_1=\mu_2$. 
\item[-] {\tt i=3}~: 
Double real leading unstable eigenvalues (saddle to saddle-focus
transition);\\ 
$\lambda_1=\lambda_2$. 
\item[-] {\tt i=4}~: 
Neutral saddle, saddle-focus or bi-focus (includes {\tt i=1});
$\mbox{Re}(\mu_1)  =  - \mbox{Re}(\lambda_1)$. 
\item[-] {\tt i=5}~: 
Neutrally-divergent saddle-focus (stable eigenvalues complex);\\
$\mbox{Re}(\lambda_1) = - \mbox{Re}(\mu_1) - \mbox{Re}(\mu_2)$.
\item[-] {\tt i=6}~: 
Neutrally-divergent saddle-focus (unstable eigenvalues complex);\\
$\mbox{Re}(\mu_1) = - \mbox{Re}(\lambda_1) - \mbox{Re}(\lambda_2)$. 
\item[-] {\tt i=7}~: 
Three leading eigenvalues (stable);
$\mbox{Re}(\lambda_1) = - \mbox{Re}(\mu_1) - \mbox{Re}(\mu_2)$. 
\item[-] {\tt i=8}~: 
Three leading eigenvalues (unstable);
$\mbox{Re}(\mu_1) = - \mbox{Re}(\lambda_1) - \mbox{Re}(\lambda_2)$.
\item[-] {\tt i=9}~: 
Local bifurcation (zero eigenvalue or Hopf): 
number of stable eigenvalues decreases; $\mbox{Re}(\mu_1)=0$.
\item[-] {\tt i=10}~: 
Local bifurcation (zero eigenvalue or Hopf): 
number of unstable eigenvalues decreases; $\mbox{Re}(\lambda_1)=0$.
\item[-] {\tt i=11}~: 
Orbit flip with respect to leading stable direction 
(e.g., 1D unstable manifold).
\item[-] {\tt i=12}~: 
Orbit flip with respect to leading unstable direction, 
(e.g., 1D stable manifold).
\item[-] {\tt i=13}~: 
Inclination flip with respect to stable manifold
(e.g., 1D unstable manifold).
\item[-] {\tt i=14}~: 
Inclination flip with respect to unstable manifold
(e.g., 1D stable manifold).
\item[-] {\tt i=15}~: 
Non-central homoclinic to saddle-node (in stable manifold).
\item[-] {\tt i=16}~: 
Non-central homoclinic to saddle-node (in unstable manifold).
\end{itemize}

Expert users may wish to add their own test functions by editing 
the function {\tt PSIHO} in {\tt autlib5.f}.

{\it It is important to remember that, in order to specify activated 
test functions, it is required to also 
add the corresponding label $+20$ to the list of continuation
parameters and a zero of this parameter to the list of user-defined
output points. Having done this, the corresponding parameters
are output to the screen and zeros are accurately located.} 

\section{ Starting Strategies.} \label{sec:Starting_strategies}
There are four possible starting procedures for continuation. 

\begin{itemize}

\item[{\bf(i)}]
Data can be read from a previously-obtained output point from {\cal AUTO}
 (e.g., from continuation of a periodic orbit up to large period;
note that the end-point of the data stored must be close to the
equilibrium). These data can be read from fort.8 (saved to {\tt
s.xxx}) by making {\tt IRS} correspond to the label of the data
point in question.

\item[{\bf(ii)}]
Data from numerical integration (e.g.,\ computation of a stable
periodic orbit, or an approximate homoclinic obtained by shooting)  
can be read in from a data file using the general {\cal AUTO} 
utility {\tt @fc} (see earlier in the manual). 
The  numerical data should be stored in
a file  {\tt xxx.dat}, in multi-column format according to the read statement
\begin{verbatim}
       READ(...,*) T(J),(U(I,J),I=1,NDIM)
\end{verbatim}
where {\tt T} runs in the interval {\tt [0,1]}.
After running {\tt @fc} the restart data is stored in
the format of a previously computed solution in {\tt s.dat}.
When starting from this solution {\tt IRS} should be set to 1 and 
the value of {\tt ISTART} is irrelevant.

\item[{\bf(iii)}]
By setting {\tt ISTART=2},  
an explicit homoclinic solution can be specified in the routine {\tt STPNT} 
in the usual {\cal AUTO} format, that is 
{\tt U=...(T)} where {\tt T} is scaled to lie in the
interval{\tt [0,1]}. 

\item[{\bf(iv)}]
The choice {\tt ISTART=3}, allows for
a homotopy method to be used to approach a homoclinic orbit
starting from a small approximation to a solution to the 
linear problem in the unstable manifold \cite{DoFrMo:93}. For
details of implementation, the reader is referred to 
Section~5.1.2.\ of \citeasnoun{ChKu:94}, under the simplification
that we do not solve for the adjoint $u(t)$ here. The basic idea
is to start with a small solution in the unstable manifold, and perform
continuation in {\tt PAR(11)=}$2T$ and dummy initial-condition 
parameters $\xi_i$ in order to satisfy the correct right-hand boundary
conditions, which are defined by zeros of other dummy parameters
$\omega_i$. More precisely, the left-hand end point is placed in the
tangent space to the unstable manifold of the saddle and is characterized by
{\tt NUNSTAB} coordinates $\xi_i$ satisfying the condition
$$
\xi_1^2 + \xi_2^2 + \ldots +\xi_{\tt NUNSTAB}^2  = \eps_0^2,
$$
where $\eps_0$ is a user-defined small number.
At the right-hand end point, {\tt NUNSTUB} values $\omega_i$ 
measure the deviation of this point from the tangent
space to the stable manifold of the saddle. 
\par
Suppose that {\tt IEQUIB=0,1} and set {\tt IP=12+IEQUIB*NDIM}. Then
\par
\medskip
\begin{center}
\begin{tabular}{ll}
{\tt PAR(IP)} & :$\ \ \eps_0$\\
{\tt PAR(IP+i)} &  :$\ \ \xi_{\tt i}$, {\tt i=1,2,...,NUNSTAB}\\
{\tt PAR(IP+NUNSTAB+i)} & :$\ \ \omega_{\tt i}$, {\tt i=1,2,...,NUNSTAB}
\end{tabular}
\end{center}
\par
\medskip
{\it Note that to avoid interference with the test functions 
(i.e. {\tt PAR(21)-PAR(36)}), one must have {\tt IP+2*NUNSTAB < 21}.} 
\par
If an $\omega_i$ is vanished, it can be frozen while another dummy or system parameter is allowed to
vary in order to make consequently all $\omega_i=0$. The resulting final solution
gives the initial homoclinic orbit provided the right-hand end point
is sufficiently close to the saddle. 
See Chapter~\ref{ch:HomCont_kpr} for an example, 
however, we recommend the homotopy method only for ``expert users''.
\end{itemize}

To compute the orientation of a homoclinic orbit (in order to detect
inclination-flip bifurcations) it is necessary to compute, in tandem,
a solution to the modified adjoint variational equation, by setting
{\tt ITWIST=1}. In order to obtain starting data for such a
computation when restarting from a point where just the homoclinic
is computed, upon increasing {\tt ITWIST} to 1, {\cal AUTO} generates
trivial data for the adjoint. Because the adjoint equations are
linear, only a single step of Newton's method is required to
enable these trivial data to converge to the correct unique bounded
solution. This can be achieved by making a single continuation step in a
trivial parameter (i.e. a parameter that does not appear
in the problem). 

Decreasing {\tt ITWIST} to 0 automatically deletes the data for the adjoint
from the continuation problem.


\section{ Notes on Running {\cal HomCont} Demos.} \label{sec:HomCont_Tutorial_examples}
{\cal HomCont} demos are given in the following chapters.
To copy all files of a demo {\tt xxx} (for example, {\tt san}),
move to a clean directory and type {\it @dm xxx}.
Simply typing {\it make} or {\it make all} will then automatically
execute all runs of the demo.
To automatically run a demo in ``step-by-step'' mode,
type  {\it make first}, {\it make second}, etc.,
to run each separate computation of the demo. 
At each step, the user is encouraged to plot the data
saved by using the command {\it @p} (e.g., {\it @p 1} plots the data
saved in {\tt b.1} and {\tt s.1}).

Of course, in a real application, the runs will not have been prepared
in advance, and {\cal AUTO}-commands must be used.
Such commands can be found in a table at the end of each chapter.
Note that the sequence of detailed {\cal AUTO}-commands given in these tables
can be abbreviated, as illustrated in Table~\ref{tbl:HomCont_demos_1} 
and Table~\ref{tbl:HomCont_demos_2} for two representative runs of 
{\cal HomCont} demo {\tt san}.


The user is encouraged to copy the format of one of these demos
when constructing new examples.

The output of the {\cal HomCont} demos reproduced in  the following chapters
is somewhat machine dependent, as already noted 
in Section~\ref{sec:Tutorial_all_runs}.
In exceptional circumstances, {\cal AUTO} may reach its maximum number of
steps {\tt NMX} before a certain output point, or the label of
an output point may change. In such case the user may have
to make appropriate changes in the {\cal AUTO} constants-files.


\begin{table}[htbp]
\begin{center}
\begin{tabular}{| l | l |}
\hline
  COMMAND  & ACTION \\
\hline
%==============================================================================
  {\it cp c.san.1 c.san} & get the AUTO constants-file    \\ 
  {\it cp s.san.1 s.san} & get the HomCont constants-file \\ 
  {\it @h san}           & run {\cal AUTO}/{\cal HomCont}   \\ 
  {\it @sv 6}            & save output-files as {\tt b.6, s.6, d.6}  \\ 
\hline
%==============================================================================
  {\it @H san 1}           &    \\ 
  {\it @sv 6}              &    \\ 
\hline
%==============================================================================
\end{tabular}
\caption{ These two sets of {\cal AUTO}-Commands are equivalent.}
\label{tbl:HomCont_demos_1}
\end{center}
\end{table}


\begin{table}[htbp]
\begin{center}
\begin{tabular}{| l | l |}
\hline
  COMMAND  & ACTION \\
\hline
%==============================================================================
  {\it cp c.san.9 c.san} & get the AUTO constants-file \\ 
  {\it cp s.san.9 s.san} & get the HomCont constants-file \\ 
  {\it @h san 6}         & run {\cal AUTO}/{\cal HomCont}; restart solution read from {\tt s.6} \\ 
  {\it @ap 6}            & append output-files to {\tt b.6, s.6, d.6} \\ 
\hline
%==============================================================================
  {\it @H san 9 6}         &  \\ 
  {\it @ap 6}              &  \\ 
\hline
%==============================================================================
\end{tabular}
\caption{ These two sets of {\cal AUTO}-Commands are equivalent.}
\label{tbl:HomCont_demos_2}
\end{center}
\end{table}



%==============================================================================
%==============================================================================
\chapter{ {\cal HomCont} Demo : san.} \label{ch:HomCont_san}
%==============================================================================
%==============================================================================

%==============================================================================
%DEMO=san======================================================================
%==============================================================================
\section{ Sandstede's Model.}
\newcommand{\ti}{\tilde}
Consider the system \cite{Sa:95b}
\begin{equation} \label{bs1} \begin{array}{rcl}
\dot{x} & = & a \, x + b \, y - a \, x^2 + (\ti \mu - \alpha \, z) \, x
\, (2-3x) \\
\dot{y} & = & b \, x + a \, y - \frac{3}{2} \, b \, x^2 - 
\frac{3}{2} \, a \, x \, y - (\ti \mu - \alpha \, z) \, 2 \, y \\
\dot{z} & = & c \, z + \mu \, x + \gamma\, x\, y + \alpha \, 
\beta \, (x^2 \, (1-x) - y^2)

\end{array} \end{equation}
as given in the file {\tt san.f}.
Choosing the constants appearing
in (\ref{bs1}) appropriately allows for computing inclination and
orbit flips as well as non-orientable resonant bifurcations, see
\cite{Sa:95b} for details and proofs. The starting point for all
calculations is $a=0$, $b=1$ where there exists an explicit solution
given by  
$$ 
(x(t),y(t),z(t)) = 
\left( 1 - \left(\frac{1-e^t}{1+e^t}\right)^2 , 4 \, e^t \,
\frac{1-e^t}{(1+e^t)^3} , 0 \right). 
$$
This solution is specified in the routine {\tt STPNT}.

\section{Inclination Flip.}
We start by copying the demo to the current work directory 
and running the first step
\begin{center}
\it @dm san \\
make first
\end{center}
This computation starts from the analytic solution above with 
$a=0$, $b=1$, $c=-2$, $\alpha=0$, $\beta=1$ and 
$\gamma = \mu=\ti \mu =0$. The homoclinic solution is followed in the
parameters $(a,\ti \mu)$ {\tt =(PAR(1), PAR(8))} up to $a=0.25$. 
The output is summarised on the screen as
\begin{verbatim}
   BR  PT  TY LAB    PAR(1)        L2-NORM           PAR(8)     
    1   1  EP   1  0.000000E+00  4.000000E-01 ...  0.000000E+00
    1   5  UZ   2  2.500000E-01  4.030545E-01 ... -3.620329E-11
    1  10  EP   3  7.384434E-01  4.339575E-01 ... -9.038826E-09
\end{verbatim}
and saved in more detail as {\tt b.1}, {\tt s.1} and {\tt d.1}.

Next we want to add a solution to the adjoint equation to the
solution obtained at $a=0.25$. This is
achieved by making the change {\tt ITWIST = 1} saved in {\tt s.san.2},
and {\tt IRS = 2},  {\tt NMX = 2} and {\tt ICP(1) = 9} saved in 
{\tt c.san.2}. We also disable any
user-defined functions {\tt NUZR=0}. The computation so-defined 
is a single step in a trivial parameter {\tt PAR(9)} (namely a parameter
that does not appear in the problem). The effect is to perform a Newton
step to enable {\cal AUTO} to converge to a solution of the adjoint equation.
\begin{center}
{\it make second}
\end{center} 
The output is stored in {\tt b.2}, {\tt s.2}  and {\tt d.2}.

We can now continue the homoclinic plus
adjoint in $(\alpha,\ti \mu)$ {\tt =(PAR(4), PAR(8))} by
changing the constants (stored in {\tt c.san.3}) to read
{\tt IRS = 4}, {\tt NMX = 50} and {\tt ICP(1) = 4}.
We also add {\tt PAR(10)} to the list of continuation parameters
{\tt NICP,(ICP(I),I=1 NICP)}. Here {\tt PAR(10)} is a dummy parameter used in
order to make the continuation of the adjoint well posed. Theoretically,
it should be zero if the computation of the adjoint is successful
\cite{Sa:95b}.
The test functions for detecting resonant bifurcations 
({\tt ISPI(1)=1}) and inclination flips ({\tt ISPI(1)=13}) are
also activated. Recall that this should be specified in
three ways. First we add {\tt PAR(21)} and {\tt PAR(33)}
to the list of continuation parameters in {\tt c.san.3}, second we set up user defined
output at zeros of these parameters in the same file, and finally we set {\tt NPSI=2}
{\tt (IPSI(1),IPSI(2))=1,13} in {\tt s.san.3}. We also add to {\tt c.san.3} another user zero
for detecting when {\tt PAR(4)=1.0}.
Running 
\begin{center}
{\it make third }\\
\end{center}
reads starting data from {\tt s.2} and outputs to the screen
\begin{verbatim}
 BR  PT  TY LAB    PAR(4)     ...    PAR(8)        PAR(10)    ...    PAR(33)    
  1  20       5  7.847219E-01 ... -3.001440E-11 -4.268884E-09 ... -1.441124E+01
  1  27  UZ   6  1.000000E+00 ... -3.844872E-11 -4.460769E-09 ... -5.701675E+00
  1  35  UZ   7  1.230857E+00 ... -5.833977E-11 -4.530541E-09 ...  9.434843E-06
  1  40       8  1.383969E+00 ... -8.133899E-11 -4.671817E-09 ...  1.348810E+00
  1  50  EP   9  1.695209E+00 ... -1.386324E-10 -5.098460E-09 ...  5.311065E-01
\end{verbatim}
Full output is stored in {\tt b.3}, {\tt s.3} and {\tt d.3}. 
\begin{figure}[b]
\epsfysize 9.0cm
\centerline{\epsffile{include/san1.ps}}
\caption{Second versus third component of the solution to the adjoint
equation at labels {\tt 5, 7} and {\tt 9}}
\label{Ftest1}
\end{figure}
Note that the artificial parameter $\epsilon=${\tt PAR(10)} is zero within
the allowed tolerance. At label {\tt 7}, a zero of test function $\psi_{13}$ has
been detected which corresponds to an inclination flip with respect to
the stable manifold. That the orientation of the homoclinic loop
changes as the family passes through this point can be read from
the information in {\tt d.3}.
However in {\tt d.3}, the line 
\begin{verbatim} 
ORIENTABLE (    0.2982090775D-03)
\end{verbatim}
at {\tt PT=35} would seems to contradict the 
detection of the inclination flip at this point. Nonetheless, the
important fact is the zero of the test function; and note that 
the value of the variable indicating the orientation is 
small compared to its value at the other regular points. 
Data for the adjoint equation at {\tt LAB= 5, 7} and {\tt 9} at
and on either side of the inclination flip are presented in 
Fig.\ \ref{Ftest1}. The switching of the solution between components
of the leading unstable left eigenvector is apparent.
Finally, we remark that the Newton step in the dummy 
parameter {\tt PAR(20)} performed above is crucial
to obtain convergence. Indeed, if instead we try to continue the
homoclinic orbit and the solution of the adjoint equation directly by
setting
\begin{verbatim}
  ITWIST = 1   IRS = 2   NMX = 50   ICP(1) = 4   NPUSZR = 0
\end{verbatim}
(as saved in {\tt c.san.4}) and running
\begin{center}
{\it make fourth}
\end{center}
we obtain a no convergence error.

\section{Non-orientable Resonant Eigenvalues.}
Inspecting the output saved in the third run,
we observe the existence of a non-orientable homoclinic orbit at label 
{\tt 7} corresponding to {\tt N=40}. We restart at this label, with
the first continuation parameter being once again $a=${\tt PAR(1)}, 
by changing constants and storing them in {\tt c.san.5} according to 
\begin{verbatim}
   IRS = 7     DS = -0.05D0    NMX = 20    ICP(1) = 1
\end{verbatim}
Running, 
 \begin{center}
{\it make fifth}\\
\end{center}
the output at label {\tt 10}
\begin{verbatim}
  BR    PT  TY LAB    PAR(1)           PAR(8)        PAR(10)       PAR(21)       
  1     8  UZ  10 -1.304570E-07  ... 3.874816E-12 -1.468457E-09 -2.609139E-07 
\end{verbatim}
indicates that {\cal AUTO} has detected a zero of
{\tt PAR(21)}, implying that a non-orientable resonant bifurcation occurred at that
point.

\section{Orbit Flip.}
In this section we compute an orbit flip. To this end we restart
from the original explicit solution, without computing the orientation. We 
begin by separately performing continuation in $(\alpha,\ti \mu)$, 
$(\beta,\ti \mu)$, $(a,\ti \mu)$, $(b,\ti \mu)$ and $(\mu, \ti \mu)$
in order to reach the parameter values 
$(a,b,\alpha,\beta, \mu)=(0.5,3,1,0,0.25)$.
The sequence of continuations up to the desired parameter values 
are run via
\begin{center}
\it make sixth\\
make seventh\\
make eighth\\
make ninth\\ 
make tenth\\
\end{center}
with appropriate continuation parameters and user output values
set in the corresponding files {\tt c.san.xx}. 
All the output is saved to {\tt s.6}.

The final saved point {\tt LAB=10} contains a homoclinic solution at
the desired parameter values. From here we perform continuation in
the negative direction of $(\mu,\ti \mu)=$ ({\tt PAR(7),PAR(8)}) with
the test function $\psi_{11}$ for orbit flips with respect to the
stable manifold activated.
\begin{center}
\it make eleventh
\end{center}
The output detects an inclination flip (by a zero of {\tt PAR(31)}) 
at {\tt PAR(7)=0} 
\begin{verbatim}
  BR    PT  TY LAB    PAR(7)      ...    PAR(8)        PAR(31)    
  1     5  UZ  12  2.394737E-07   ...  6.434492E-08 -4.133994E-06
\end{verbatim}
at which parameter value the homoclinic orbit is contained in the $(x,y)$-plane
(see Fig.\ \ref{Ftest2}).

%------------------------------------------------------
\begin{figure}[t]
\epsfysize 9.0cm
\centerline{\epsffile{include/san2.ps}}
\caption{Orbits on either side of the orbit flip bifurcation. The critical
orbit is contained in the $(x,y)$-plane}
\label{Ftest2}
\end{figure}
%------------------------------------------------------

Finally, we demonstrate that the orbit flip can be continued as 
three parameters ({\tt PAR(6), PAR(7), PAR(8)}) are varied. 
\begin{center}
\it make twelfth
\end{center}
\begin{verbatim}
 BR    PT  TY LAB    PAR(7)       ...    PAR(8)        PAR(6)     
   1     5      14 -5.374538E-19  ... -1.831991E-10 -3.250000E-01
   1    10      15 -6.145911E-19  ... -2.628607E-10 -8.250001E-01
   1    15      16 -4.947133E-19  ... -2.361151E-10 -1.325000E+00
   1    20  EP  17 -5.792940E-19  ... -3.075527E-10 -1.825000E+00
\end{verbatim}
The orbit flip continues to be defined by a planar homoclinic orbit
at {\tt PAR(7)=PAR(8)=0}.

\newpage
\section{ Detailed {\cal AUTO}-Commands.}

\begin{table}[htbp]
\begin{center}
\begin{tabular}{| l | l |}
\hline
  COMMAND  & ACTION \\
\hline
%==============================================================================
  {\it mkdir san} & create an empty work directory \\ 
  {\it cd san} & change directory \\
  {\it @dm san} & copy the demo files to the work directory \\
\hline
%==============================================================================
  {\it cp c.san.1 c.san} & get the AUTO constants-file \\ 
  {\it cp s.san.1 s.san} & get the HomCont constants-file \\ 
  {\it @h san} &  continuation in {\tt PAR(1)} \\ 
  {\it @sv 1} & save output-files as {\tt b.1, s.1, d.1} \\ 
\hline
%==============================================================================
  {\it cp c.san.2 c.san} & get the AUTO constants-file \\ 
  {\it cp s.san.2 s.san} & get the HomCont constants-file \\ 
  {\it @h san 1} & generate adjoint variables; restart from {\tt s.1} \\ 
  {\it @sv 2} & save output-files as {\tt b.2, s.2, d.2} \\ 
\hline
%=============================================================================
  {\it cp c.san.3 c.san} & get the AUTO constants-file \\ 
  {\it cp s.san.3 s.san} & get the HomCont constants-file \\ 
  {\it @h san 2} & continue homoclinic orbit and adjoint; restart from {\tt s.2} \\ 
  {\it @sv 3} & save output-files as {\tt b.3, s.3, d.3} \\ 
\hline
%==============================================================================
  {\it cp c.san.4 c.san} & get the AUTO constants-file \\ 
  {\it cp s.san.4 s.san} & get the HomCont constants-file \\ 
  {\it @h san 1} & no convergence without dummy step; restart from {\tt s.1} \\ 
  {\it @sv 4} &  save output-files as {\tt b.4, s.4, d.4} \\ 
\hline
%=============================================================================
  {\it cp c.san.5 c.san} & get the AUTO constants-file \\ 
  {\it cp s.san.5 s.san} & get the HomCont constants-file \\
  {\it @h san 3} & continue non-orientable orbit; restart from {\tt s.3} \\
  {\it @sv 5} & save output-files as {\tt b.5, s.5, d.5} \\ 
\hline
%==============================================================================
\end{tabular}
\caption{Detailed {\cal AUTO}-Commands for running demo {\tt san}.}
\label{tbl:demo_san_1}
\end{center}
\end{table}


\begin{table}[htbp]
\begin{center}
\begin{tabular}{| l | l |}
\hline
  COMMAND  & ACTION \\
\hline
%==============================================================================
  {\it cp c.san.6 c.san} & get the AUTO constants-file \\ 
  {\it cp s.san.6 s.san} & get the HomCont constants-file \\ 
  {\it @h san} & restart and homotopy to {\tt PAR(4)}=1.0 \\ 
  {\it @sv 6} & save output-files as {\tt b.6, s.6, d.6} \\ 
\hline
%==============================================================================
  {\it cp c.san.7 c.san} & get the AUTO constants-file \\ 
  {\it cp s.san.7 s.san} & get the HomCont constants-file \\ 
  {\it @h san 6} & homotopy to {\tt PAR(5)}=0.0; restart from {\tt s.6} \\ 
  {\it @ap 6} & append output-files to {\tt b.6, s.6, d.6} \\ 
\hline
%==============================================================================
  {\it cp c.san.8 c.san} & get the AUTO constants-file \\ 
  {\it cp s.san.8 s.san} & get the HomCont constants-file \\ 
  {\it @h san 6} & homotopy to {\tt PAR(1)}=0.5; restart from {\tt s.6} \\ 
  {\it @ap 6} & append output-files to {\tt b.6, s.6, d.6} \\ 
\hline
  {\it cp c.san.9 c.san} & get the AUTO constants-file \\ 
  {\it cp s.san.9 s.san} & get the HomCont constants-file \\ 
  {\it @h san 6} & homotopy to {\tt PAR(2)}=3.0; restart from {\tt s.6} \\ 
  {\it @ap 6} & append output-files to {\tt b.6, s.6, d.6} \\ 
\hline
%==============================================================================
  {\it cp c.san.10 c.san} & get the AUTO constants-file \\ 
  {\it cp s.san.10 s.san} & get the HomCont constants-file \\ 
  {\it @h san 6} & homotopy to {\tt PAR(7)}=0.25; restart from {\tt s.6} \\ 
  {\it @ap 6} & append output-files to {\tt b.6, s.6, d.6} \\ 
\hline
%==============================================================================
  {\it cp c.san.11 c.san} & get the AUTO constants-file \\ 
  {\it cp s.san.11 s.san} & get the HomCont constants-file \\ 
  {\it @h san 6} & continue in {\tt PAR(7)} to detect orbit flip; restart from {\tt s.6} \\ 
  {\it @sv 11} & save output-files as {\tt b.11, s.11, d.11} \\ 
\hline
%==============================================================================
  {\it cp c.san.12 c.san} & get the AUTO constants-file \\ 
  {\it cp s.san.12 s.san} & get the HomCont constants-file \\ 
  {\it @h san 11} & three-parameter continuation of orbit flip; restart from {\tt s.11} \\ 
  {\it @sv 12} & save output-files as {\tt b.12, s.12, d.12} \\ 
\hline
%==============================================================================
\end{tabular}
\caption{Detailed {\cal AUTO}-Commands for running demo {\tt san}.}
\label{tbl:demo_san_2}
\end{center}
\end{table}




%==============================================================================
%==============================================================================
\chapter{ {\cal HomCont} Demo : mtn.} \label{ch:HomCont_mtn}
%==============================================================================
%==============================================================================

%==============================================================================
%DEMO=mtn======================================================================
%==============================================================================
\section{ A Predator-Prey Model with Immigration.}
Consider the following system of two equations \cite{Sc:95}
\begin{equation} \label{sn.1} \begin{array}{rcl}
\dot{X} & = & RX\left(1-{\frac{X}{K}}\right) - 
{\frac{A_1XY}{B_1+X}} + D_0K \\
\dot{Y} & = & E_1 { \frac{A_1XY}{B_1+X}} - D_1Y - 
{\frac{A_2ZY^2}{B_2^2+Y^2}}.
\end{array} \end{equation}
%------------------------------------------------------
\begin{figure}[b]
\epsfysize 10.0cm
\centerline{\epsffile{include/mtn1.ps}}
\caption{Parametric portrait of the predator-prey system }
\label{SNF.1}
\end{figure}
%------------------------------------------------------
The values of all parameters except $(K,Z)$ are set as follows~:
$$
R=0.5,\ A_1=0.4,\ B_1=0.6,\ D_0=0.01,\ E_1=0.6,\ A_2=1.0,\ B_2=0.5,\ D_1=0.15.
$$
\par
\noindent
The parametric portrait of the system (\ref{sn.1}) on the
$(Z,K)$-plane is presented in Figure \ref{SNF.1}. It contains fold
($t_{1,2}$) and Hopf ($H$) bifurcation curves, as well as a homoclinic
bifurcation curve $P$. The fold curves meet at a cusp singular point
$C$, while the Hopf and the homoclinic curves originate at a
Bogdanov-Takens point $BT$. Only the homoclinic curve $P$ will be 
considered here, the other bifurcation curves can be computed using
{\tt AUTO} or,
for example, {\cal locbif} \cite{KhKuLeNi:93}.

\section{Continuation of Central Saddle-Node Homoclinics.}
Local bifurcation analysis shows that at $K=6.0,\ Z=0.06729762\ldots$,
the system has a saddle-node equilibrium 
$$
(X^0,Y^0) = (5.738626\ldots,0.5108401\ldots),
$$
with one zero and one negative eigenvalue. Direct simulations reveal a
homoclinic 
orbit to this saddle-node, departing and returning along its central
direction (i.e., tangent to the null-vector).
\par
Starting from this solution, stored in the file {\tt mtn.dat}, we
continue the saddle-node central homoclinic orbit 
with respect to the parameters $K$ and $Z$ by copying the
demo and running it
\begin{center}
{\it @dm mtn}\\
{\it make first}
\end{center}
The file {\tt mtn.f} contains approximate
parameter values
$$
K={\tt PAR(1)}=6.0,\ Z={\tt PAR(2)}=0.06729762,
$$
as well as the coordinates of the saddle-node
$$
X^0={\tt PAR(12)}=5.738626,\ Y^0={\tt PAR(13)}=0.5108401,
$$
and the length of the truncated time-interval
$$
T_0={\tt PAR(11)} = 1046.178 \: .
$$
Since a homoclinic orbit to a saddle-node is being followed, we have also
made the choices
$$
{\tt IEQUIB = 2 \quad NUNSTAB =0 \quad NSTAB = 1   }
$$
in {\tt s.mtn.1}. The two test-functions, $\psi_{15}$ and $\psi_{16}$, 
to detect non-central saddle-node homoclinic
orbits are also activated, which must be specified in three ways. 
Firstly, in {\tt s.mtn.1}, {\tt NPSI} is
set to 2 and the active test functions {\tt IPSI(I),I=1,2}
are chosen as 15 and 16. This sets up the monitoring of these
test functions. Secondly, in {\tt c.mtn.1} user-defined functions
({\tt NUZR=2}) are set up to look for zeros of the parameters
corresponding to these test functions. Recall that the
parameters to be zeroed are always the test functions plus 20.
Finally, these parameters are included in the list of continuation
parameters ({\tt NICP,(ICP(I),I=1 NICP)}).

Among the output there is a line 
\begin{verbatim}
  BR    PT  TY LAB    PAR(1)    ...     PAR(2)        PAR(35)       PAR(36)    
   1    27  UZ   5  6.10437E+00 ...   6.932475E-02 -6.782898E-07  8.203437E-02
\end{verbatim}
indicating that a zero of the test function {\tt IPSI(1)=15} 
This means that at
$$
D_1=(K^1,Z^1)=(6.6104\ldots, 0.069325\ldots)
$$
the homoclinic orbit to the saddle-node becomes {\it non-central}, namely,
it returns to the equilibrium along the stable eigenvector, forming a
non-smooth loop. The output is saved in {\tt b.1}, {\tt s.1} and {\tt d.1}. 
Repeating computations in the opposite direction along the curve, 
{\tt IRS=1, DS=-0.01} in {\tt c.mtn.2}, 
\begin{center}
{\it make second}
\end{center}
one obtains 
\begin{verbatim}
  BR    PT  TY LAB    PAR(1)     ...    PAR(2)        PAR(35)       PAR(36)  
   1    34  UZ   9  5.180323E+00 ...  6.385506E-02  3.349720E-09  9.361957E-02
\end{verbatim}
which means another non-central saddle-node homoclinic bifurcation occurs
at
$$
D_2=(K^2,Z^2)=(5.1803\ldots,0.063855\ldots).
$$
Note that these data were obtained using a smaller value of {\tt NTST} than
the original computation (compare {\tt c.mtn.1} with {\tt c.mtn.2}). The
high original value of {\tt NTST} was only necessary for the first few steps
because the original solution is specified on a uniform mesh. 

\section{Switching between Saddle-Node and Saddle Homoclinic Orbits.}
Now we can switch to continuation of saddle homoclinic orbits at the
located codim 2 points $D_1$ and $D_2$. 
\begin{center}
{\it make third}
\end{center}
starts from $D_1$. Note that now 
\begin{center}
{\tt 
NUNSTAB = 1 \quad IEQUIB = 1}  
\end{center}
has been specified in {\tt s.mtn.3}. Also, test functions $\psi_9$
and $\psi_{10}$ have been activated in order
to monitor for non-hyperbolic equilibria along the homoclinic locus. 
We get the following output
\begin{verbatim}
  BR    PT  TY LAB    PAR(1)     ...    PAR(2)        PAR(29)       PAR(30)    
   1    10      11  7.114523E+00 ...  7.081751E-02 -4.649861E-01  3.183429E-03
   1    20      12  9.176810E+00 ...  7.678731E-02 -4.684912E-01  1.609294E-02
   1    30      13  1.210834E+01 ...  8.543468E-02 -4.718871E-01  3.069638E-02
   1    40  EP  14  1.503788E+01 ...  9.428036E-02 -4.743794E-01  4.144558E-02
\end{verbatim}
The fact that {\tt PAR(29)} and {\tt PAR(30)} do not change sign indicates 
that there are no further non-hyperbolic equilibria
along this family. Note that restarting in the opposite direction with {\tt IRS=11,
DS=-0.02} 
\begin{center}
{\it make fourth}
\end{center}
will detect the same codim 2 point $D_1$ but now as a zero
of the test-function $\psi_{10}$
\begin{verbatim}
  BR    PT  TY LAB    PAR(1)     ...    PAR(2)        PAR(29)       PAR(30)
  1    10  UZ  15  6.610459E+00  ...  6.932482E-02 -4.636603E-01  1.725013E-09    
\end{verbatim}
Note that the values of {\tt PAR(1)} and {\tt PAR(2)} differ from that at label {\tt 4} 
only in the sixth significant figure. 

Actually, the program runs further and
eventually computes the point $D_2$ and the whole lower family of $P$
emanating from it, however, the solutions between $D_1$ and $D_2$
should be considered as spurious\footnote{\label{ft1} The program actually
computes the saddle-saddle heteroclinic orbit bifurcating from the
non-central saddle-node homoclinic at the point $D_1$, see
\citeasnoun[Fig. 2]{ChKuSa:95}, and continues it to the one emanating from
$D_2$.}, therefore we do not save these data.
The reliable way to compute the lower family of $P$ is to restart computation
of saddle homoclinic orbits in the other direction from the point $D_2$
\begin{center}
{\it make fifth}
\end{center}
This gives the lower family of $P$ approaching the BT point
(see Figure \ref{SNF.1})
\begin{verbatim}
  BR    PT  TY LAB    PAR(1)     ...   PAR(2)        PAR(29)       PAR(30)    
   1    10      15  4.966429E+00 ... 6.298418E-02 -4.382426E-01  4.946824E-03
   1    20      16  4.925379E+00 ... 7.961214E-02 -3.399102E-01  3.288447E-02
   1    30      17  7.092267E+00 ... 1.587114E-01 -1.692842E-01  3.876291E-02
   1    40  EP  18  1.101819E+01 ... 2.809825E-01 -3.482651E-02  2.104384E-02
\end{verbatim}
The data are appended to the stored results in {\tt b.1}, {\tt s.1} and
{\tt d.1}. One could now display all data using the {\cal AUTO}
command {\it @p 1} to reproduce the curve $P$ shown in Figure
\ref{SNF.1}.
\par
%------------------------------------------------------
\begin{figure}[p]
\epsfysize 10.0cm
\centerline{\epsffile{include/mtn2.ps}}
\caption{Approximation by a large-period cycle}
\label{SNF.2}
\end{figure}
%------------------------------------------------------
%------------------------------------------------------
\begin{figure}[p]
\epsfysize 9.0cm
\centerline{\epsffile{include/mtn3.ps}}
\caption{Projection onto the ($K,D_0$)-plane of the 
three-parameter curve of non-central  saddle-node homoclinic orbit}
\label{SNF.3}
\end{figure}
%------------------------------------------------------
%
It is worthwhile to compare the homoclinic curves computed above with
a curve $T_0=const$ along which the system has a limit cycle of constant large
period $T_0=1046.178$, which can easily be computed using {\cal AUTO} or
{\cal locbif}. Such a curve is plotted in Figure \ref{SNF.2}. 
It obviously approximates well the saddle homoclinic loci of $P$, but 
demonstrates much bigger
deviation from the saddle-node homoclinic segment $D_1D_2$. This happens
because the period of the limit cycle grows to infinity while approaching both
types of homoclinic orbit, but with {\it different asymptotics}: 
as $-\ln\|\alpha-\alpha^*\|$, in the saddle homoclinic case, and 
as $\|\alpha-\alpha^*\|^{-1}$ in the saddle-node case. 

\section{Three-Parameter Continuation.}
Finally, we can follow the curve of non-central saddle-node homoclinic
orbits in three parameters. The extra continuation parameter is
$D_0$={\tt PAR(3)}.  To achieve this we restart at label {\tt 4},
corresponding to the codim 2 point $D_1$. We return to continuation of
saddle-node homoclinics, {\tt NUNSTAB=0},{\tt IEQUIB=2}, but append the
defining equation $\psi_{15}=0$ to the continuation problem
(via {\tt NFIXED=1}, {\tt IFIXED(1)=15}). The new
continuation problem is specified in {\tt c.mtn.6} and {\tt s.mtn.6}.
\begin{center}
{\it make sixth}
\end{center}
Notice that we set {\tt ILP=1} and choose {\tt PAR(3)} as the first 
continuation parameter so that {\cal AUTO} can detect limit points 
with respect to this parameter. We also make a user-defined function
({\tt NUZR=1})
to detect intersections with the plane $D_0=0.01$.
We get among other output
\begin{verbatim}
  BR    PT  TY LAB    PAR(3)        L2-NORM    ...    PAR(1)        PAR(2)
   1    22  LP  19  1.081212E-02  5.325894E+00 ...  5.673631E+00  6.608184E-02
   1    31  UZ  20  1.000000E-02  4.819681E+00 ...  5.180317E+00  6.385503E-02
\end{verbatim}
the first line of which represents the $D_0$ value at which 
the homoclinic curve $P$ has a tangency with the family $t_2$ 
of fold bifurcations. Beyond this value of $D_0$,
$P$ consists entirely of saddle homoclinic orbits. The data at label {\tt 20} 
reproduce the coordinates of the point $D_2$. The results of this
computation and a similar one starting from $D_1$ in the opposite direction
(with {\tt DS=-0.01}) are displayed in Figure \ref{SNF.3}.
%

\newpage
\section{ Detailed {\cal AUTO}-Commands.}
\begin{table}[htbp]
\begin{center}
\begin{tabular}{| l | l |}
\hline
  COMMAND  & ACTION \\
\hline
%==============================================================================
  {\it mkdir mtn} & create an empty work directory \\ 
  {\it cd mtn} & change directory \\
  {\it @dm mtn} & copy the demo files to the work directory \\
\hline
%==============================================================================
  {\it cp c.mtn.1 c.mtn} & get the AUTO constants-file \\ 
  {\it cp s.mtn.1 s.mtn} & get the HomCont constants-file \\ 
  {\it @fc mtn} & use the starting data in {\tt mtn.dat} to create {\tt s.dat} \\ 
  {\it @h mtn dat} &  continue saddle-node homoclinic orbit\\
  {\it @sv 1} & save output-files as {\tt b.1, s.1, d.1} \\ 
\hline
%==============================================================================
  {\it cp c.mtn.2 c.mtn} & get the AUTO constants-file \\ 
  {\it cp s.mtn.2 s.mtn} & get the HomCont constants-file \\ 
  {\it @h mtn 1} & continue in opposite direction; restart from {\tt s.1} \\ 
  {\it @ap 1} & append output-files to {\tt b.1, s.1, d.1} \\ 
\hline
%=============================================================================
  {\it cp c.mtn.3 c.mtn} & get the AUTO constants-file \\ 
  {\it cp s.mtn.3 s.mtn} & get the HomCont constants-file \\ 
  {\it @h mtn 1} & switch to saddle homoclinic orbit  ; restart from {\tt s.1} \\ 
  {\it @ap 1} & append output-files to {\tt b.1, s.1, d.1} \\ 
\hline
%==============================================================================
  {\it cp c.mtn.4 c.mtn} & get the AUTO constants-file \\ 
  {\it cp s.mtn.4 s.mtn} & get the HomCont constants-file \\ 
  {\it @h mtn 1} & continue in reverse direction; restart from {\tt s.1} \\ 
  {\it @sv 4} & save output-files as {\tt b.4, s.4, d.4} \\ 
\hline
%=============================================================================
  {\it cp c.mtn.5 c.mtn} & get the AUTO constants-file \\ 
  {\it cp s.mtn.5 s.mtn} & get the HomCont constants-file \\
  {\it @h mtn 1} & other saddle homoclinic orbit family; restart from {\tt s.1} \\
  {\it @ap 1} & append output-files to {\tt b., s.1, d.1} \\ 
\hline
%==============================================================================
  {\it cp c.mtn.6 c.mtn} & get the AUTO constants-file \\ 
  {\it cp s.mtn.6 s.mtn} & get the HomCont constants-file \\ 
  {\it @h mtn 1} & 3-parameter non-central saddle-node homoclinic. \\ 
  {\it @sv 6} & save output-files as {\tt b.6, s.6, d.6} \\ 
\hline
%==============================================================================
\end{tabular}
\caption{Detailed {\cal AUTO}-Commands for running demo {\tt mtn}.}
\label{tbl:demo_mtn_1}
\end{center}
\end{table}



%==============================================================================
%==============================================================================
\chapter{ {\cal HomCont} Demo : kpr.} \label{ch:HomCont_kpr}
%==============================================================================
%==============================================================================

%==============================================================================
%DEMO=kpr======================================================================
%==============================================================================
\section{ Koper's Extended Van der Pol Model.}
%
The equation-file {\tt kpr.f} contains the equations
\begin{equation} \label{ko} \begin{array}{rcl}
\dot{x} & = & \eps_1^{-1}\:(k\: y - x^3 +3\:x - \lambda) \\
\dot{y} & = & x - 2\: y + z \\
\dot{z} & = & \eps_2(y-z), 
\end{array} \end{equation}
with $\eps_1 =0.1$ and $\eps_2=1$ \cite{Ko:95}.

To copy across the demo {\tt kpr} and compile we type
\begin{center}
{\it @dm kpr} \\
\end{center}

\section{The Primary family of Homoclinics.}
First, we locate a homoclinic orbit using 
the homotopy method. The file {\tt kpr.f} 
already contains 
approximate parameter values for a homoclinic orbit, 
namely $\lambda=${\tt PAR(1)=-1.851185}, $k=${\tt PAR(2)=-0.15}. 
The files {\tt c.kpr.1} and {\tt s.kpr.1} specify the appropriate
constants for continuation in $2T${\tt =PAR(11)} (also referred
to as {\tt PERIOD}) and the dummy parameter $\omega_1$={\tt PAR(17)}
starting
from a small solution in the local unstable manifold; 
\begin{center}
{\it make first}
\end{center}
Among the output there is the line
\begin{verbatim}
     BR    PT  TY LAB    PERIOD        L2-NORM     ...    PAR(17)    ...
      1    29  UZ   2  1.900184E+01  1.693817E+00  ...  4.433433E-09 ... 
\end{verbatim}
which indicates that a zero of the artificial parameter $\omega_1$
has been located. This means that the right-hand end point of the solution
belongs to the plane that is tangent to the stable manifold at the saddle. 
The output is stored in files {\tt b.1, s.1, d.1}. 
Upon plotting the data at label {\tt 2} (see Figure \ref{kf.1a})
it can be noted that although the right-hand projection boundary
condition is satisfied, the solution is still quite away from the
equilibrium. 

The right-hand endpoint can be made to approach the
equilibrium by performing a further continuation in $T$ with the
right-hand projection condition satisfied ({\tt PAR(17)} fixed) but
with $\lambda$ allowed to vary. 
%
\begin{figure}[p]
\epsfysize 9.0cm
\centerline{\epsffile{include/kpr1.ps}}
\caption{Projection on the $(x,y)$-plane of solutions 
of the boundary value 
problem with $2T=19.08778$.}
\label{kf.1a}
\end{figure}
\begin{figure}[p]
\epsfysize 9.0cm
\centerline{\epsffile{include/kpr2.ps}}
\caption{Projection on the $(x,y)$-plane of solutions of the 
boundary value problem with $2T = 60.0$.}
\label{kf.1b}
\end{figure}
%
\begin{center}
{\it make second}
\end{center}
the output at label {\tt 4}, stored in {\tt kpr.2},
\begin{verbatim}
   BR   PT TY   LAB    PERIOD       L2-NORM     ...    PAR(1)     ...
   1    35  UZ   4  6.000000E+01  1.672806E+00  ... -1.851185E+00 ...
\end{verbatim}
provides a good approximation to a homoclinic solution (see Figure
\ref{kf.1b}). 

The second stage to obtain a starting solution 
is to add a solution to the modified adjoint
variational equation. This is achieved by setting both 
{\tt ITWIST} and {\tt ISTART} to 1 (in {\tt s.kpr.3}), which generates
a trivial guess for the adjoint equations. Because the adjoint
equations are linear, only a single
Newton step (by continuation in a trivial parameter) 
is required to provide a solution.
Rather than choose a parameter that might be used internally
by {\cal AUTO}, in {\tt c.kpr.3} we take the continuation parameter
to be {\tt PAR(11)}, which is not quite a trivial parameter
but whose affect upon the solution is mild.
\begin{center}
{\it make third}
\end{center}
The output at the second point (label {\tt 6}) 
contains the converged homoclinic
solution (variables ({\tt U(1), U(2), U(3)}) and the adjoint ({\tt
U(4), U(5), U(6)})). We now have a starting solution 
and are ready to perform two-parameter continuation.

The fourth run
\begin{center}
{\it make fourth}
\end{center}
continues the homoclinic orbit in {\tt PAR(1)} and {\tt PAR(2)}. 
%
%------------------------------------------------------
\begin{figure}[p]
\epsfysize 9.0cm
\centerline{\epsffile{include/kpr4.ps}}
\caption{Projection on the $(x,y)$-plane of solutions $\phi(t)$
at {\tt 1} ($\lambda=-1.825470, k=-0.1760749$) and
{\tt 2} ($\lambda=-1.686154, k=-0.3183548$).}
\label{kf.2a}
\end{figure}
%------------------------------------------------------
%------------------------------------------------------
\begin{figure}[p]
\epsfysize 8.0cm
\centerline{\epsffile{include/kpr5.ps}}
\caption{Three-dimensional blow-up of the solution curves
 $\phi(t)$ 
at labels {\tt 1} (dotted) and {\tt 2} (solid line) from Figure 3.8.}
\label{kf.2b}
\end{figure}
%------------------------------------------------------
%
Note that several other parameters appear in
the output. {\tt PAR(10)} is a dummy parameter
that should be zero when the adjoint is being computed correctly;
{\tt PAR(29)}, {\tt PAR(30)}, {\tt PAR(33)} correspond to the
test functions $\psi_9$,$\psi_{10}$ and $\psi_{13}$. 
That these test functions were activated is specified
in three places in {\tt c.kpr.4} and {\tt s.kpr.4} 
as described in Section~\ref{sec:HomCont_Test_functions}.  

Note that at the end-point of
the family (reached when after {\tt NMX=50} steps) {\tt PAR(29)} is
approximately zero which corresponds to a zero of $\psi_9$, a 
non-central saddle-node homoclinic orbit. We shall return to the computation of
this codimension-two point later. Before reaching this point,
among the output we find two zeroes of {\tt PAR(33)}
(test function $\psi_{13}$) which gives the accurate
location of two inclination-flip bifurcations,
\begin{verbatim}
 BR  PT  TY LAB    PAR(1)     ...     PAR(2)        PAR(10)   ...    PAR(33)  
  1   6  UZ  10 -1.801662E+00 ... -2.002660E-01 -7.255434E-07 ... -1.425714E-04
  1  12  UZ  11 -1.568756E+00 ... -4.395468E-01 -2.156353E-07 ...  4.514073E-07
\end{verbatim}
That the test function really does have a regular zero at this point can
be checked from the data saved in {\tt b.3}, plotting {\tt PAR(33)} as
a function of {\tt PAR(1)} or {\tt PAR(2)}. 
Figure \ref{kf.2a} presents solutions $\phi(t)$ of the modified adjoint 
variational equation (for details see \citeasnoun{ChKuSa:95})
at parameter values on the homoclinic 
family before and after the first detected inclination flip. 
Note that these solutions were obtained by choosing a smaller
step {\tt DS} and more output (smaller {\tt NPR}) in
{\tt c.kpr.4}.
A blow-up of the region close to the origin of this 
figure is shown in Figure \ref{kf.2b}.
It illustrates the flip of the solutions of the adjoint equation while
moving through the bifurcation point. Note that the data in this
figure were plotted after first performing an additional
continuation of the solutions with respect to {\tt PAR(11)}. 

Continuing in the other direction 
\begin{center}
{\it make fifth}
\end{center}
we approach a Bogdanov-Takens point
\begin{verbatim}  
 BR    PT  TY LAB    PAR(1)     ...    PAR(10)    ...    PAR(33)    
  1    50  EP  13 -1.938276E+00 ... -7.523344E+00 ...  6.310810E+01
\end{verbatim}
%------------------------------------------------------
\begin{figure}[t]
\epsfysize 9.0cm
\centerline{\epsffile{include/kpr6.ps}}
\caption{Computed homoclinic orbits approaching the BT point}
\label{kp.6}
\end{figure}
%------------------------------------------------------
Note that the numerical approximation has ceased to become reliable, since 
{\tt PAR(10)} has now become large. 
Phase portraits of homoclinic orbits between the BT point and the first
inclination flip 
are depicted in Figure \ref{kp.6}. Note how the computed homoclinic orbits
approaching the BT point have their endpoints well away from the equilibrium.
To follow the homoclinic orbit to 
the BT point with more precision, we would need to first perform continuation 
in $T$ ({\tt PAR(11)}) to obtain a more accurate homoclinic solution.


\section{More Accuracy and Saddle-Node Homoclinic Orbits.}
Continuation in $T$ 
in order to obtain an approximation of the homoclinic orbit over a
longer interval is necessary for parameter values near a non-hyperbolic
equilibrium (either a saddle-node or BT) where the convergence
to the equilibrium is slower. 
First, we start from the original homoclinic orbit computed
via the homotopy method, label {\tt 4}, which is well away from
the non-hyperbolic equilibrium.
Also, we shall no longer be interested in
in inclination flips so we set {\tt ITWIST=0} in {\tt c.kpr.6},
and in order to compute up to {\tt PAR(11)=1000}, we set up a
user-defined function for this. Running {\cal AUTO} with {\tt PAR(11)} and 
{\tt PAR(2)} as free parameters
\begin{center}
{\it make sixth}
\end{center}
we obtain among the output
\begin{verbatim}
  BR    PT  TY LAB     PERIOD       L2-NORM    ...    PAR(2)     
   1    35  UZ   6  1.000000E+03  1.661910E+00 ... -1.500000E-01
\end{verbatim}

We can now repeat the computation of the family of saddle homoclinic
orbits in {\tt PAR(1)} and {\tt PAR(2)} from this point with
the test functions $\psi_9$ and $\psi_{10}$ for non-central
saddle-node homoclinic orbits activated 
\begin{center}
{\it make seventh}
\end{center}
The saddle-node point is now detected at 
\begin{verbatim} 
  BR    PT  TY LAB    PAR(1)     ...    PAR(2)        PAR(29)       PAR(30)
   1    30  UZ   8  1.765003E-01 ... -2.405345E+00  2.743361E-06  2.309317E+01
\end{verbatim}
which is stored in {\tt s.7}.
That {\tt PAR(29)} ($\psi_9$) is zeroed shows that this
is a non-central saddle-node connecting the centre manifold to the strong stable
manifold. Note that all output beyond this point, although a well-posed
solution to the boundary-value problem, is spurious in that it no longer
represents a homoclinic orbit to a saddle equilibrium (see
\citeasnoun{ChKuSa:95}). If we had chosen
to, we could continue in the other direction in order to
approach the BT point more accurately by reversing the sign of
{\tt DS} in {\tt c.kpr.7}.
 
The files {\tt c.kpr.9} and {\tt s.kpr.9} contain the constants necessary 
for switching to continuation of the central saddle-node homoclinic curve 
in two parameters starting from the non-central saddle-node homoclinic orbit
stored as label {\tt 8} in {\tt s.7}.
\begin{center}
{\it make eighth}
\end{center}
In this run we have activated the test functions for saddle to saddle-node
transition points along curves of saddle homoclinic orbits ($\psi_{15}$ and 
$\psi_{16}$). Among the output we find
\begin{verbatim}
  BR    PT  TY LAB    PAR(1)     ...    PAR(2)        PAR(35)       PAR(36)    
   1    38  UZ  13  1.765274E-01 ... -2.405284E+00  9.705426E-03 -5.464784E-07
\end{verbatim}
%------------------------------------------------------
\begin{figure}[p]
\epsfysize 9.0cm
\centerline{\epsffile{include/kpr7.ps}}
\caption{Two non-central saddle-node homoclinic orbits, {\tt 1} and {\tt 3};
and, {\tt 2}, a central saddle-node homoclinic orbit between
these two points \label{kf.7}}
\end{figure}
%------------------------------------------------------
%------------------------------------------------------
\begin{figure}[p]
\epsfysize 9.0cm
\centerline{\epsffile{include/kpr8.ps}}
\caption{The big homoclinic orbit approaching a figure-of-eight}
\label{kp.8}
\end{figure}
%------------------------------------------------------
%
which corresponds to the family of homoclinic orbits leaving
the locus of saddle-nodes in a second non-central saddle-node
homoclinic bifurcation (a zero of $\psi_{16}$). 

Note that the parameter values do not vary much between the
two codimension-two non-central saddle-node points (labels {\tt 8} and {\tt 13}).
However, Figure \ref{kf.7} shows clearly that between the two
codimension-two points 
the homoclinic orbit
rotates between the two components of the 1D stable manifold, i.e.\
between the two boundaries of the center-stable manifold of the saddle
node. The overall effect of this process is the transformation of a
nearby ``small'' saddle homoclinic orbit to a ``big'' saddle
homoclinic orbit (i.e.\ with two extra turning points in phase space).  

Finally, we can switch to continuation of the big saddle homoclinic orbit 
from the new codim 2 point at label {\tt 13}. 
\begin{center}
{\it make ninth}
\end{center}
Note that {\cal AUTO} takes a large number of steps near the line  
{\tt PAR(1)=0}, while {\tt PAR(2)} approaches $-2.189\ldots$  
(which is why we chose such a large value {\tt NMX=500} in {\tt c.kpr.9}). This
particular computation ends at 
\begin{verbatim}  
  BR    PT  TY LAB    PAR(1)        L2-NORM    ...    PAR(2)   
   1   500  EP  24 -1.218988E-05  2.181205E-01 ... -2.189666E+00
\end{verbatim}
By plotting phase portraits of orbits approaching this end point (see Figure
\ref{kp.8}) we see a ``canard-like'' like transformation of the big homoclinic
orbit to a pair of homoclinic orbits in a figure-of-eight configuration.
That we get a figure-of-eight is not a surprise because {\tt PAR(1)=0}
corresponds to a symmetry in the differential equations \cite{Ko:94};
note also that the equilibrium, stored as ({\tt PAR(12), PAR(13), PAR(14)}) in
{\tt d.9}, approaches the origin as we approach the figure-of-eight homoclinic.

\section{Three-Parameter Continuation.}
We now consider curves in three parameters of each of
the codimension-two points encountered in this model, by
freeing the parameter $\eps=$ {\tt PAR(3)}.
First we continue the first inclination flip stored at label
{\tt 7} in {\tt s.3}
\begin{center}
\it make tenth
\end{center}
Note that {\tt ITWIST=1} in {\tt s.kpr.10}, so that the adjoint is also
continued, and there is one fixed condition {\tt IFIXED(1)=13} so that
test function $\psi_{13}$ has been frozen.
Among the output there is a codimension-three point (zero of $\psi_9$)
where the neutrally twisted homoclinic orbit collides with the saddle-node
curve
\begin{verbatim}
 BR  PT  TY LAB    PAR(1)     ...   PAR(2)        PAR(3)        PAR(29)    ...     
  1  28  UZ  14  1.282702E-01 ... -2.519325E+00 5.744770E-01 -4.347113E-09 ...
\end{verbatim}
The other detected inclination flip (at label {\tt 8} in {\tt s.3}) is continued
similarly
\begin{center}
\it make eleventh
\end{center}
giving among its output another codim 3 saddle-node inclination-flip point
\begin{verbatim} 
 BR  PT  TY LAB    PAR(1)     ...   PAR(2)        PAR(3)        PAR(29)    ...  
  1  27  UZ  14  1.535420E-01 ... -2.458100E+00 1.171705E+00 -1.933188E-07 ... 
\end{verbatim}
Output beyond both of these codim 3 points is spurious and both computations end in
an {\tt MX} point (no convergence).

To continue the non-central saddle-node homoclinic orbits it is
necessary to work on the data without the solution $\phi(t)$. We
restart from the data saved at {\tt LAB=8} and {\tt LAB=13} in
{\tt s.7} and {\tt s.8} respectively. We could continue these codim 2 points in two
ways, either by appending the defining condition $\psi_{16} =0$ to
the continuation of saddle-node homoclinic orbits (with {\tt IEQUIB=2},
etc.), or by appending $\psi_{9} =0$ to the continuation 
of a saddle homoclinic orbit (with {\tt IEQUIB=1}. 
The first approach is used in the example {\tt mtn},  
for contrast we shall adopt the second approach here.
%------------------------------------------------------
\begin{figure}[p]
\epsfysize 9.0cm
\centerline{\epsffile{include/kpr10.ps}}
\caption{Projection onto the {\tt (PAR(3),PAR(2))}-plane of the non-central
saddle-node homoclinic orbit curves (labeled {\tt 1} and {\tt 2}) and the 
inclination-flip curves (labeled {\tt 3} and {\tt 4})}.
\label{kp.10}
\end{figure}
%------------------------------------------------------
%
\begin{center}
{\it make twelfth}\\
{\it make thirteenth}
\end{center}
The projection onto the $(\eps,k)$-plane of all four of these
codimension-two curves is given in Figure \ref{kp.10}. 
The intersection of the inclination-flip lines with one of the
non-central saddle-node homoclinic lines is apparent. Note that the two
non-central saddle-node homoclinic orbit curves are almost overlaid, but
that as in Figure \ref{kf.7} the orbits look quite distinct in phase space.

\section{ Detailed {\cal AUTO}-Commands.}
\begin{table}[htbp]
\begin{center}
\begin{tabular}{| l | l |}
\hline
  COMMAND  & ACTION \\
\hline
%==============================================================================
  {\it mkdir kpr} & create an empty work directory \\ 
  {\it cd kpr} & change directory \\
  {\it @dm kpr} & copy the demo files to the work directory \\
\hline
%==============================================================================
  {\it cp c.kpr.1 c.kpr} & get the AUTO constants-file \\ 
  {\it cp s.kpr.1 s.kpr} & get the HomCont constants-file \\ 
  {\it @h kpr} &  continuation in the time-length parameter {\tt PAR(11)} \\ 
  {\it @sv 1} & save output-files as {\tt b.1, s.1, d.1} \\ 
\hline
%==============================================================================
  {\it cp c.kpr.2 c.kpr} & get the AUTO constants-file \\ 
  {\it cp s.kpr.2 s.kpr} & get the HomCont constants-file \\ 
  {\it @h kpr 1} & locate the homoclinic orbit; restart from {\tt s.1} \\ 
  {\it @sv 2} & save output-files as {\tt b.2, s.2, d.2} \\ 
\hline
%=============================================================================
  {\it cp c.kpr.3 c.kpr} & get the AUTO constants-file \\ 
  {\it cp s.kpr.3 s.kpr} & get the HomCont constants-file \\ 
  {\it @h kpr 2} & generate adjoint variables  ; restart from {\tt s.2} \\ 
  {\it @sv 3} & save output-files as {\tt b.3, s.3, d.3} \\ 
\hline
%==============================================================================
  {\it cp c.kpr.4 c.kpr} & get the AUTO constants-file \\ 
  {\it cp s.kpr.4 s.kpr} & get the HomCont constants-file \\ 
  {\it @h kpr 3} & continue the homoclinic orbit; restart from {\tt s.3} \\ 
  {\it @ap 3} & append output-files to {\tt b.3, s.3, d.3} \\ 
\hline
%=============================================================================
  {\it cp c.kpr.5 c.kpr} & get the AUTO constants-file \\ 
  {\it cp s.kpr.5 s.kpr} & get the HomCont constants-file \\
  {\it @h kpr 3} & continue in reverse direction; restart from {\tt s.3} \\
  {\it @ap 3} & append output-files to {\tt b.3, s.3, d.3} \\ 
\hline
%==============================================================================
  {\it cp c.kpr.6 c.kpr} & get the AUTO constants-file \\ 
  {\it cp s.kpr.6 s.kpr} & get the HomCont constants-file \\ 
  {\it @h kpr 2} & increase the period; restart from {\tt s.2} \\ 
  {\it @sv 6} & save output-files as {\tt b.6, s.6, d.6} \\ 
\hline
%==============================================================================
\end{tabular}
\caption{Detailed {\cal AUTO}-Commands for running demo {\tt kpr}.}
\label{tbl:demo_kpr_1}
\end{center}
\end{table}


\begin{table}[htbp]
\begin{center}
\begin{tabular}{| l | l |}
\hline
  COMMAND  & ACTION \\
\hline
%==============================================================================
  {\it cp c.kpr.7 c.kpr} & get the AUTO constants-file \\ 
  {\it cp s.kpr.7 s.kpr} & get the HomCont constants-file \\ 
  {\it @h kpr 6} & recompute the family of homoclinic orbits; restart from {\tt s.6} \\ 
  {\it @sv 7} & save output-files as {\tt b.7, s.7, d.7} \\ 
\hline
%==============================================================================
  {\it cp c.kpr.8 c.kpr} & get the AUTO constants-file \\ 
  {\it cp s.kpr.8 s.kpr} & get the HomCont constants-file \\ 
  {\it @h kpr 7} & continue central saddle-node homoclinics; restart from {\tt s.7} \\ 
  {\it @sv 8} & save output-files as {\tt b.8, s.8, d.8} \\ 
\hline
%==============================================================================
  {\it cp c.kpr.9 c.kpr} & get the AUTO constants-file \\ 
  {\it cp s.kpr.9 s.kpr} & get the HomCont constants-file \\ 
  {\it @h kpr 8} & continue homoclinics from codim-2 point; restart from {\tt s.8} \\ 
  {\it @sv 9} & save output-files as {\tt b.9, s.9, d.9} \\ 
\hline
%==============================================================================
  {\it cp c.kpr.10 c.kpr} & get the AUTO constants-file \\ 
  {\it cp s.kpr.10 s.kpr} & get the HomCont constants-file \\ 
  {\it @h kpr 3} & 3-parameter curve of inclination-flips; restart from {\tt s.3} \\ 
  {\it @sv 10} & save output-files as {\tt b.10, s.10, d.10} \\ 
\hline
%==============================================================================
  {\it cp c.kpr.11 c.kpr} & get the AUTO constants-file \\ 
  {\it cp s.kpr.11 s.kpr} & get the HomCont constants-file \\ 
  {\it @h kpr 3} & another curve of inclination-flips; restart from {\tt s.3} \\ 
  {\it @sv 11} & save output-files as {\tt b.11, s.11, d.11} \\ 
\hline
%==============================================================================
  {\it cp c.kpr.12 c.kpr} & get the AUTO constants-file \\ 
  {\it cp s.kpr.12 s.kpr} & get the HomCont constants-file \\ 
  {\it @h kpr 7} & continue non-central saddle-node homoclinics; restart from {\tt s.7} \\ 
  {\it @sv 12} & save output-files as {\tt b.12, s.12, d.12} \\ 
\hline
%==============================================================================
  {\it cp c.kpr.13 c.kpr} & get the AUTO constants-file \\ 
  {\it cp s.kpr.13 s.kpr} & get the HomCont constants-file \\ 
  {\it @h kpr 8} & continue non-central saddle-node homoclinics; restart from {\tt s.8} \\ 
  {\it @ap 12} & append output-files to {\tt b.12, s.12, d.12} \\ 
\hline
%==============================================================================
\end{tabular}
\caption{Detailed {\cal AUTO}-Commands for running demo {\tt kpr}.}
\label{tbl:demo_kpr_2}
\end{center}
\end{table}



%==============================================================================
%==============================================================================
\chapter{ {\cal HomCont} Demo : cir.} \label{ch:HomCont_cir}
%==============================================================================
%==============================================================================

%==============================================================================
%DEMO=cir======================================================================
%==============================================================================
\section{ Electronic Circuit of Freire {\it et al}.}
Consider the following model of a three-variable electronic circuit
\cite{FrRLuGaPo:93}
 \begin{equation}
\left \{ 
\begin{array}{rcl}
\dot{x} & = & \left [-(\beta+\nu) x + \beta y -a_3 x^3 
+b_3(y-x)^3\right ]/r, \\
\dot{y} & = & \beta x -(\beta+\gamma)y -z -b_3(y-x)^3, \\
\dot{z} & = & y.
\end{array}
\right.  
\label{5.fr1}
\end{equation}
These autonomous equations are also considered in the {\cal AUTO} demo {\tt tor}.

First, we copy the demo into a new directory and compile
\begin{center}
{\it @dm cir} \\
\end{center}
The system is contained in 
the equation-file {\tt cir.f} and the initial run-time constants
are stored in {\tt c.cir.1} and {\tt s.cir.1}. We begin by starting from
the data from {\tt cir.dat} for a saddle-focus homoclinic orbit 
at 
$\nu=-0.721309$, $\beta=0.6$, $\gamma=0$, $r=0.6$, $A_3=0.328578$ 
and $B_3=0.933578$, which was obtained by shooting over 
the time interval $2T=${\tt PAR(11)}$=36.13$.
We wish to follow the family in the $(\beta,\nu)$-plane, but 
first we perform continuation in $(T,\nu)$ to obtain a better 
approximation to a homoclinic orbit.
\begin{center}
{\it make first}
\end{center} 
yields the output
\begin{verbatim}
 BR  PT  TY LAB     PERIOD       L2-NORM    ...   PAR(1)     
  1  21  UZ   2  1.000000E+02  1.286637E-01 ... -7.213093E-01
  1  42  UZ   3  2.000000E+02  9.097899E-02 ... -7.213093E-01
  1  50  EP   4  2.400000E+02  8.305208E-02 ... -7.213093E-01
\end{verbatim}
Note that $\nu=${\tt PAR(1)} remains constant during the continuation
as the parameter values do not change, only the the length of
the interval over which the approximate homoclinic solution is computed.
Note from the eigenvalues, stored in {\tt d.1} that this is a homoclinic
orbit to a saddle-focus with a one-dimensional unstable manifold.

We now restart at {\tt LAB=3}, corresponding to a time interval $2T=200$,
and change the principal continuation parameters to be $(\nu,\beta)$.
The new constants defining the continuation are given in {\tt c.cir.2}
and {\tt s.cir.2}.
We also activate the test functions pertinent to codimension-two
singularities which may be encountered along a family of saddle-focus
homoclinic orbits, viz.\ $\psi_2$, $\psi_4$, $\psi_5$, $\psi_9$ and $\psi_{10}$.
This must be specified in three ways: by choosing {\tt NPSI=5} and appropriate
{\tt IPSI(I)} in {\tt s.cir.2}, by adding the corresponding parameter labels
to the list of continuation parameters {\tt ICP(I)} in {\tt c.cir.2}
(recall that these parameter indices are 20 more than the corresponding
$\psi$ indices), and finally adding USZR functions defining zeros of
these parameters in {\tt c.cir.2}. Running 
\begin{center}  
{\it make second}
\end{center} 
results in
\begin{verbatim}
BR  PT  TY LAB    PAR(1)     ...    PAR(2)     ...    PAR(25)       PAR(29)    
1   17  UZ   5 -7.256925E-01 ...  4.535645E-01 ... -1.765251E-05 -2.888436E-01
1   75  UZ   6 -1.014704E+00 ...  9.998966E-03 ...  1.664509E+00 -5.035997E-03
1   78  UZ   7 -1.026445E+00 ... -2.330391E-05 ...  1.710804E+00  1.165176E-05
1   81  UZ   8 -1.038012E+00 ... -1.000144E-02 ...  1.756690E+00  4.964621E-03  
1  100  EP   9 -1.164160E+00 ... -1.087732E-01 ...  2.230329E+00  5.042736E-02
\end{verbatim}
with results saved in {\tt b.2, s.2, d.2}.
%------------------------------------------------------
\begin{figure}[p]
\epsfysize 9.0cm
\centerline{\epsffile{include/cir1.ps}}
\caption{Solutions of the boundary value problem at labels 6 and 8, 
either side of the Shil'nikov-Hopf bifurcation}
\label{Fcircuit1}
\end{figure}
%------------------------------------------------------
%------------------------------------------------------
\begin{figure}[p]
\epsfysize 9.0cm
\centerline{\epsffile{include/cir2.ps}}
\caption{Phase portraits of three homoclinic orbits 
on the family, showing the saddle-focus to saddle transition}
\label{Fcircuit2}
\end{figure}
%------------------------------------------------------
Upon inspection of the output, note that label 5, where {\tt PAR(25)}$\approx 0$, 
corresponds to a neutrally-divergent saddle-focus, $\psi_5=0$. 
Label 7, where {\tt PAR(29)}$\approx 0$ corresponds to a local bifurcation, $\psi_9=0$, 
which we note from the eigenvalues stored in {\tt d.2} corresponds to a {\it
Shil'nikov-Hopf} bifurcation. Note that {\tt PAR(2)} is also approximately zero
at label 7, which accords with the analytical observation that the origin of
(\ref{5.fr1}) undergoes a Hopf bifurcation when $\beta=0$.
Labels 6 and 8 are the user-defined output
points, the solutions at which are plotted in Fig.\ \ref{Fcircuit1}.
Note that solutions beyond label 7 (e.g., the plotted solution
at label 8) do not correspond to homoclinic orbits, but to 
{\it point-to-cycle} heteroclinic orbits (c.f.\ Section~2.2.1 of
\citeasnoun{ChKuSa:95}).

We now continue in the other direction along the family. It turns out
that starting from the initial point in the other direction results in
missing a codim 2 point which is close to the starting point. Instead we
start from the first saved point from the previous computation
(label 5 in {\tt s.2}):
\begin{center}
\it make third
\end{center}
The output
\begin{verbatim}
 BR  PT  TY LAB    PAR(1)     ...    PAR(2)        PAR(22)       PAR(24)    
  1   9  UZ  10 -7.204001E-01 ...  5.912315E-01 -1.725669E+00 -3.295862E-05
  1  18  UZ  11 -7.590583E-01 ...  7.428734E-01  3.432139E-05 -2.822988E-01
  1  26  UZ  12 -7.746686E-01 ...  7.746147E-01  5.833163E-01  1.637611E-07
  1  28  EP  13 -7.746628E-01 ...  7.746453E-01  5.908902E-01  1.426214E-04
\end{verbatim}
contains a neutral saddle-focus (a {\it Belyakov} transition) 
at {\tt LAB=10} ($\psi_4=0$), a double real leading eigenvalue 
(saddle-focus to saddle transition) at {\tt LAB =11} ($\psi_2=0$) 
and a neutral saddle at {\tt LAB=12} ($\psi_4=0$). Data at several
points on the complete family are plotted in Fig.\ \ref{Fcircuit2}.
If we had continued further (by increasing {\tt NMX}), 
the computation would end at a no convergence error {\tt TY=MX} owing 
to the homoclinic family approaching a Bogdanov-Takens singularity 
at small amplitude. To compute further towards the BT point 
we would first need to continue to a higher value of {\tt PAR(11)}.

\section{ Detailed {\cal AUTO}-Commands.}
\begin{table}[htbp]
\begin{center}
\begin{tabular}{| l | l |}
\hline
  COMMAND  & ACTION \\
\hline
%==============================================================================
  {\it mkdir cir} & create an empty work directory \\ 
  {\it cd cir} & change directory \\
  {\it @dm cir} & copy the demo files to the work directory \\
\hline
%==============================================================================
  {\it cp c.cir.1 c.cir} & get the AUTO constants-file \\ 
  {\it cp s.cir.1 s.cir} & get the HomCont constants-file \\ 
  {\it @fc cir} & use the starting data in {\tt cir.dat} to create {\tt s.dat} \\ 
  {\it @h cir dat} &  increase the truncation interval; restart from {\tt s.dat}\\ 
  {\it @sv 1} & save output-files as {\tt b.1, s.1, d.1} \\ 
\hline
%==============================================================================
  {\it cp c.cir.2 c.cir} & get the AUTO constants-file \\ 
  {\it cp s.cir.2 s.cir} & get the HomCont constants-file \\ 
  {\it @h cir 1} &  continue saddle-focus homoclinic orbit; restart from {\tt s.1} \\ 
  {\it @sv 2} & save output-files as {\tt b.2, s.2, d.2} \\ 
\hline
%=============================================================================
  {\it cp c.cir.3 c.cir} & get the AUTO constants-file \\ 
  {\it cp s.cir.3 s.cir} & get the HomCont constants-file \\ 
  {\it @h cir 2} & generate adjoint variables  ; restart from {\tt s.2} \\ 
  {\it @ap 2} & append output-files as {\tt b.2, s.2, d.2} \\ 
\hline
%==============================================================================
\end{tabular}
\caption{Detailed {\cal AUTO}-Commands for running demo {\tt cir}.}
\label{tbl:demo_cir_1}
\end{center}
\end{table}





%==============================================================================
%==============================================================================
\chapter{ {\cal HomCont} Demo : she.} \label{ch:HomCont_she}
%==============================================================================
%==============================================================================

%==============================================================================
%DEMO=she======================================================================
%==============================================================================
\section{ A Heteroclinic Example.}
The following system of five equations \citeasnoun{RuMa:95}
\begin{equation} \label{sh1} \begin{array}{rcl}
\dot{x} & = & \mu \, x + x\, y - z\, u, \\
\dot{y} & = & -y - x^2, \\
\dot{z} & = & (4\sigma\, x\, u + 4\sigma\, \mu\, z -9\sigma\, z 
+ 4 x\, u + 4\mu\, z) / 4(1+\sigma)  \\
\dot{u} & = & - \sigma u / 4 - \sigma Q v / 4\pi^2
+ 3(1 + \sigma) x z / 4\sigma \\
\dot{v} & = & \zeta u / 4  - \zeta v / 4
\end{array} 
\end{equation}
has been used to describe shearing instabilities in fluid convection.
The equations possess a rich structure of local and global bifurcations.
Here we shall reproduce a single curve in the $(\sigma,\mu)$-plane
of codimension-one heteroclinic orbits connecting a non-trivial 
equilibrium to the origin for $Q=0$ and $\zeta=4$. The defining
problem is contained in equation-file 
{\tt she.f}\footnote{The last parameter used to store the equilibria ({\tt PAR(21)}) is
overlaped here with the first test-function. In this example, it is harmless since the test functions are 
irrelevant for heteroclinic continuation.}, and starting data for the orbit at 
$(\sigma,\mu)=(0.5,0.163875)$ are stored in {\tt she.dat},
with a truncation interval of {\tt PAR(11)=85.07}.

We begin by computing towards $\mu=0$ with the option {\tt IEQUIB=-2}
which means that both equilibria are solved for as part of
the continuation process.
\begin{center}
{\it @dm she} \\
{\it make first}
\end{center} 
This yields the output
\begin{verbatim}
  BR    PT  TY LAB    PAR(3)        L2-NORM    ...    PAR(1)     
   1     5       2  4.528332E-01  3.726787E-01 ...  1.364973E-01
   1    10       3  3.943370E-01  3.303798E-01 ...  1.044119E-01
   1    15       4  3.358942E-01  2.873213E-01 ...  7.515570E-02
   1    20       5  2.772726E-01  2.433403E-01 ...  4.952636E-02
   1    25       6  2.181955E-01  1.981358E-01 ...  2.845849E-02
   1    30  EP   7  1.581633E-01  1.512340E-01 ...  1.292975E-02
\end{verbatim}
Alternatively, for this problem there exists an analytic expression for
the two equilibria. This is specified in the routine {\tt PVLS} of
{\tt she.f}. Re-running with {\tt IEQUIB=-1}
\begin{center}
\it make second
\end{center} 
we obtain the output
\begin{verbatim}
  BR    PT  TY LAB    PAR(3)        L2-NORM    ...    PAR(1)     
   1     5       2  4.432015E-01  3.657716E-01 ...  1.310559E-01
   1    10       3  3.723085E-01  3.142439E-01 ...  9.300982E-02
   1    15       4  3.008842E-01  2.611556E-01 ...  5.933966E-02
   1    20       5  2.286652E-01  2.062194E-01 ...  3.179939E-02
   1    25       6  1.555409E-01  1.491652E-01 ...  1.239897E-02
   1    30  EP   7  8.107462E-02  9.143108E-02 ...  2.386616E-03
\end{verbatim}
This output is similar to that above, but note that it is obtained slightly
more efficiently because the extra parameters {\tt PAR(12-21)} representing the
coordinates of the equilibria are no longer
part of the continuation problem. Also note that {\cal AUTO} has chosen to take
slightly larger steps along the family. Finally, we can continue in the opposite
direction along the family from the original starting point (again with {\tt IEQUIB=-1}).
\begin{center}
\it make third
\end{center}
%
%------------------------------------------------------
\begin{figure}[b]
\epsfysize 9.0cm
\centerline{\epsffile{include/she1.ps}}
\caption{Projections into $(x,y,z)$-space of the family of heteroclinic
orbits.}
\label{Fshear}
\end{figure}
%------------------------------------------------------
%
\begin{verbatim}
  BR    PT  TY LAB    PAR(3)        L2-NORM    ...    PAR(1)     
   1     5       8  4.997590E-01  4.060153E-01 ...  1.637322E-01
   1    10       9  5.705299E-01  4.551872E-01 ...  2.065264E-01
   1    15      10  6.416439E-01  5.031844E-01 ...  2.507829E-01
   1    20      11  7.133301E-01  5.500668E-01 ...  2.959336E-01
   1    25      12  7.857688E-01  5.958712E-01 ...  3.415492E-01
   1    30      13  8.590970E-01  6.406182E-01 ...  3.872997E-01
   1    35  EP  14  9.334159E-01  6.843173E-01 ...  4.329270E-01
\end{verbatim}
The results of both computations are presented in Figure \ref{Fshear}, 
from which we see that the orbit shrinks to zero as
{\tt PAR(1)=}$\mu \to 0$.

\newpage   
\section{ Detailed {\cal AUTO}-Commands.}
\begin{table}[htbp]
\begin{center}
\begin{tabular}{| l | l |}
\hline
  COMMAND  & ACTION \\
\hline
%==============================================================================
  {\it mkdir she} & create an empty work directory \\ 
  {\it cd she} & change directory \\
  {\it @dm she} & copy the demo files to the work directory \\
\hline
%==============================================================================
  {\it cp c.she.1 c.she} & get the AUTO constants-file \\ 
  {\it cp s.she.1 s.she} & get the HomCont constants-file \\ 
  {\it @fc she} & use the starting data in {\tt she.dat} to create {\tt s.dat} \\ 
  {\it @h she dat} &  continue heteroclinic orbit; restart from {\tt s.dat}\\ 
  {\it @sv 1} & save output-files as {\tt b.1, s.1, d.1} \\ 
\hline
%==============================================================================
  {\it cp c.she.2 c.she} & get the AUTO constants-file \\ 
  {\it cp s.she.2 s.she} & get the HomCont constants-file \\ 
  {\it @h she dat} &  repeat with IEQUIB=-1 \\ 
  {\it @sv 2} & save output-files as {\tt b.2, s.2, d.2} \\ 
\hline
%=============================================================================
  {\it cp c.she.3 c.she} & get the AUTO constants-file \\ 
  {\it cp s.she.3 s.she} &  get the HomCont constants-file \\ 
  {\it @h she 2} & continue in reverse direction ; restart from {\tt s.2} \\ 
  {\it @ap 2} & append output-files to {\tt b.2, s.2, d.2} \\ 
\hline
%=============================================================================
\end{tabular}
\caption{Detailed {\cal AUTO}-Commands for running demo {\tt she}.}
\label{tbl:demo_she_1}
\end{center}
\end{table}





%==============================================================================
%==============================================================================
\chapter{ {\cal HomCont} Demo : rev.} \label{ch:HomCont_rev}
%==============================================================================
%==============================================================================

%==============================================================================
%DEMO=rev======================================================================
%==============================================================================
\section{ A Reversible System.}
The fourth-order differential equation
$$
u'''' + P u'' + u -u^3 =0
$$
arises in a number of contexts, e.g., as the travelling-wave
equation for a nonlinear-Schr\"{o}dinger equation with fourth-order
dissipation \cite{BuAk:95} and as a model of a strut on a symmetric 
nonlinear elastic foundation \cite{HuBoTh:89}. It may be expressed as
a system
\begin{equation}
\left \{ 
\begin{array}{rcl}
\dot{u_1} & = & u_2 \\
\dot{u_2} & = & u_3 \\
\dot{u_3} & = & u_4 \\
\dot{u_4} & = & -P u_3 - u_1 + u_1^3
\end{array}
\right.  
\label{6.1}
\end{equation}
Note that (\ref{6.1}) is invariant under two separate reversibilities
\begin{equation}
R_1: (u_1,u_2,u_3,u_4,t) \mapsto (u_1,-u_2,u_3,-u_4,-t)  
\label{6.R1}
\end{equation}
and 
\begin{equation}
R_2: (u_1,u_2,u_3,u_4,t) \mapsto (-u_1,u_2,-u_3,u_4,-t)  
\label{6.R2}
\end{equation}
First, we copy the demo into a new directory 
\begin{center}
{\it @dm rev }
\end{center}
For this example, we shall make two separate starts
from data stored in equation and data files {\tt rev.f.1,
rev.dat.1} and {\tt rev.f.3, rev.dat.3} respectively. The first
of these contains initial data for a solution that is reversible
under $R_1$ and the second for data that is reversible under $R_2$. 
%
%Note that {\it make} or {\it make all} will only run the
%first of these. To make the output starting from the
%$R_2$-reversible solution we need to {\it make run2}. As before,
%though we illustrate here the step by step approach.


\section{An $R_1$-Reversible Homoclinic Solution.}

The first run
\begin{center}
\it make first
\end{center}
starts by 
copying the files {\tt rev.f.1} and {\tt rev.dat.1} to 
{\tt rev.f} and {\tt rev.dat}. The orbit contained in
the data file is a ``primary'' homoclinic solution for $P=1.6$, with
truncation (half-)interval {\tt PAR(11) = 39.0448429}.
which is reversible under $R_1$. Note that this reversibility is
specified in {\tt s.rev.1} via {\tt NREV=1}, 
{\tt (IREV(I), I=1,NDIM)} {\tt = 0 1 0 1}. Note also, from
{\tt c.rev.1} that we only have one free parameter {\tt PAR(1)}
because symmetric homoclinic orbits in reversible systems are
generic rather than of codimension one.
The first run  results in the output
\begin{verbatim}
  BR    PT  TY LAB    PAR(1)        L2-NORM       MAX U(1)   ...   
   1     7  UZ   2  1.700002E+00  2.633353E-01  4.179794E-01
   1    12  UZ   3  1.800000E+00  2.682659E-01  4.806063E-01
   1    15  UZ   4  1.900006E+00  2.493415E-01  4.429364E-01
   1    20  EP   5  1.996247E+00  1.111306E-01  1.007111E-01
\end{verbatim}
which is consistent with the theoretical result that the solution
tends uniformly to zero as $P\to 0$. Note, by plotting the data
saved in {\tt s.1} that only ``half'' of the 
homoclinic orbit is computed up to its point of symmetry. See Figure
\ref{Frev1}.

%------------------------------------------------------
\begin{figure}[p]
\epsfysize 9.0cm
\centerline{\epsffile{include/rev1.ps}}
\caption{$R_1$-Reversible homoclinic solutions on the half-interval
$x/T \in [0,1]$ where $T=39.0448429$ for $P$ approaching $2$ (solutions
with labels {\tt 1-5} respectively have decreasing amplitude)}
\label{Frev1}
\end{figure}
%------------------------------------------------------
%------------------------------------------------------
\begin{figure}[p]
\epsfysize 9.0cm
\centerline{\epsffile{include/rev2.ps}}
\caption{$R_1$-reversible homoclinic orbits with oscillatory decay 
as $x \to -\infty$ (corresponding to label {\tt 6}) and monotone decay 
(at label {\tt 10})}
\label{Frev2}
\end{figure}
%------------------------------------------------------

The second run continues in the other direction of {\tt PAR(1)}, with
the test function $\psi_2$ activated 
for the detection of saddle to saddle-focus transition points
\begin{center}
\it make second
\end{center}
The output
\begin{verbatim}
 BR  PT  TY LAB    PAR(1)        L2-NORM       MAX U(1)   ...    PAR(22)    
  1  11  UZ   6  1.000005E+00  2.555446E-01  1.767149E-01 ... -3.000005E+00
  1  22  UZ   7 -1.198325E-07  2.625491E-01  4.697314E-02 ... -2.000000E+00
  1  33  UZ   8 -1.000000E+00  2.741483E-01  4.316007E-03 ... -1.000000E+00
  1  44  UZ   9 -2.000000E+00  2.873838E-01  1.245735E-11 ...  2.318248E-08
  1  55  EP  10 -3.099341E+00  3.020172E-01 -2.749454E-11 ...  1.099341E+00
\end{verbatim}
shows a saddle to saddle-focus transition 
(indicated by a zero of {\tt PAR(22)}) at {\tt PAR(1)=-2}. Beyond
that label the first component of the solution is negative and (up to the
point of symmetry) monotone decreasing. See Figure \ref{Frev2}.

\section{An $R_2$-Reversible Homoclinic Solution.}

\begin{center}
{\it make third}
\end{center}
Copies the files {\tt rev.f.3} and {\tt rev.dat.3} to 
{\tt rev.f} and {\tt rev.dat}, and runs them with the
constants stored in {\tt c.rev.3} and {\tt s.rev.3}. 
The orbit contained in
the data file is a ``multi-pulse'' homoclinic solution for $P=1.6$, with
truncation (half-)interval {\tt PAR(11) = 47.4464189}.
which is reversible under $R_2$. This reversibility is
specified in {\tt s.rev.1} via {\tt NREV=1}, 
{\tt (IREV(I), I=1,NDIM)} {\tt = 1 0 1 0}. The output 
\begin{verbatim}
  BR    PT  TY LAB    PAR(1)        L2-NORM       MAX U(1)   ...
   1    15  UZ   2  1.700000E+00  3.836401E-01  4.890015E-01  
   1    16  LP   3  1.711574E+00  3.922135E-01  5.442385E-01  
   1    19  UZ   4  1.600000E+00  4.329404E-01  7.769491E-01  
   1    31  UZ   5  1.000000E+00  4.808488E-01  1.083298E+00  
   1    86  UZ   6 -9.664802E-10  5.158463E-01  1.258650E+00  
\end{verbatim}
contains the label of a limit point ({\tt ILP} was set to {\tt 1} in
{\tt c.rev.3}, which corresponds to a ``coalescence'' of two reversible
homoclinic orbits. The two solutions on either side of this limit point are
displayed in Figure \ref{Frev3}. The computation ends in a no-convergence
point. The solution here is depicted in Figure \ref{Frev4}. The lack of
convergence is due to the large peak and trough of the solution rapidly
moving to the left as $P \to -2$ (cf. \citeasnoun{ChSp:93}).

%------------------------------------------------------
\begin{figure}[p]
\epsfysize 9.0cm
\centerline{\epsffile{include/rev3.ps}}
\caption{Two $R_2$-reversible homoclinic orbits at $P=1.6$ 
corresponding to labels {\tt 1} (smaller amplitude) and {\tt 5} (larger amplitude)}
\label{Frev3}
\end{figure}
%------------------------------------------------------
%------------------------------------------------------
\begin{figure}[p]
\epsfysize 9.0cm
\centerline{\epsffile{include/rev4.ps}}
\caption{An $R_2$-reversible homoclinic orbit at label {\tt 8}}
\label{Frev4}
\end{figure}
%------------------------------------------------------

Continuing from the initial solution in the other parameter direction
\begin{center}
\it make fourth
\end{center}
we obtain the output
\begin{verbatim}
  BR    PT  TY LAB    PAR(1)        L2-NORM       MAX U(1)   ...
   1     7  UZ   8  1.600000E+00  3.701709E-01  3.836833E-01  
   1    33  UZ   9  9.999980E-01  3.614405E-01  1.775035E-01  
   1    93  UZ  10 -7.819855E-06  3.713007E-01  4.698309E-02  
\end{verbatim}
which again ends at a no convergence error for similar reasons.




\newpage
\section{ Detailed {\cal AUTO}-Commands.}
\begin{table}[htbp]
\begin{center}
\begin{tabular}{| l | l |}
\hline
  COMMAND  & ACTION \\
\hline
%==============================================================================
  {\it mkdir rev} & create an empty work directory \\ 
  {\it cd rev} & change directory \\
  {\it @dm rev} & copy the demo files to the work directory \\
\hline
%==============================================================================
  {\it cp rev.f.1 rev.f} &  get equations file to {\tt rev.f}\\
  {\it cp rev.dat.1 rev.dat} & get the starting data to {\tt rev.dat} \\ 
  {\it cp c.rev.1 c.rev} & get the AUTO constants-file \\ 
  {\it cp s.rev.1 s.rev} & get the HomCont constants-file \\ 
  {\it @fc rev} & use the starting data in {\tt rev.dat} to create {\tt s.dat} \\ 
  {\it @h rev dat} &  increase {\tt PAR(1)} \\ 
  {\it @sv 1} & save output-files as {\tt b.1, s.1, d.1} \\ 
\hline
%==============================================================================
  {\it cp c.rev.2 c.rev} & get the AUTO constants-file \\ 
  {\it cp s.rev.2 s.rev} & get the HomCont constants-file \\ 
  {\it @h rev 1} &  continue in reverse direction; restart from {\tt s.1} \\ 
  {\it @ap 1} & append output-files to {\tt b.1, s.1, d.1} \\ 
\hline
%=============================================================================
  {\it cp rev.f.3 rev.f} & get equations file with new value of {\tt PAR(11)}\\
  {\it cp rev.dat.3 rev.dat} & get starting data with different reversibility\\
  {\it cp c.rev.3 c.rev} & get the AUTO constants-file \\ 
  {\it cp s.rev.3 s.rev} & get the HomCont constants-file \\ 
  {\it @fc rev} & use the starting data in {\tt rev.dat} to create {\tt s.dat} \\ 
  {\it @h rev dat} & restart with different reversibility \\ 
  {\it @sv 3} & save output-files as {\tt b.3, s.3, d.3} \\ 
\hline
%==============================================================================
  {\it cp c.rev.4 c.rev} & get the AUTO constants-file \\ 
  {\it cp s.rev.4 s.rev} & get the HomCont constants-file \\ 
  {\it @h rev 3} & continue in reverse direction; restart from {\tt s.3} \\ 
  {\it @ap 3} & append output-files to {\tt b.3, s.3, d.3} \\ 
\hline
%=============================================================================
\end{tabular}
\caption{Detailed {\cal AUTO}-Commands for running demo {\tt rev}.}
\label{tbl:demo_rev_1}
\end{center}
\end{table}



%==============================================================================
%==============================================================================


\bibliography{include/auto} \label{sec:bibliography}


\end{document}
